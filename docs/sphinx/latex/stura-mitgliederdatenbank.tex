%% Generated by Sphinx.
\def\sphinxdocclass{report}
\documentclass[letterpaper,10pt,english]{sphinxmanual}
\ifdefined\pdfpxdimen
   \let\sphinxpxdimen\pdfpxdimen\else\newdimen\sphinxpxdimen
\fi \sphinxpxdimen=.75bp\relax

\PassOptionsToPackage{warn}{textcomp}
\usepackage[utf8]{inputenc}
\ifdefined\DeclareUnicodeCharacter
% support both utf8 and utf8x syntaxes
  \ifdefined\DeclareUnicodeCharacterAsOptional
    \def\sphinxDUC#1{\DeclareUnicodeCharacter{"#1}}
  \else
    \let\sphinxDUC\DeclareUnicodeCharacter
  \fi
  \sphinxDUC{00A0}{\nobreakspace}
  \sphinxDUC{2500}{\sphinxunichar{2500}}
  \sphinxDUC{2502}{\sphinxunichar{2502}}
  \sphinxDUC{2514}{\sphinxunichar{2514}}
  \sphinxDUC{251C}{\sphinxunichar{251C}}
  \sphinxDUC{2572}{\textbackslash}
\fi
\usepackage{cmap}
\usepackage[T1]{fontenc}
\usepackage{amsmath,amssymb,amstext}
\usepackage{babel}



\usepackage{times}
\expandafter\ifx\csname T@LGR\endcsname\relax
\else
% LGR was declared as font encoding
  \substitutefont{LGR}{\rmdefault}{cmr}
  \substitutefont{LGR}{\sfdefault}{cmss}
  \substitutefont{LGR}{\ttdefault}{cmtt}
\fi
\expandafter\ifx\csname T@X2\endcsname\relax
  \expandafter\ifx\csname T@T2A\endcsname\relax
  \else
  % T2A was declared as font encoding
    \substitutefont{T2A}{\rmdefault}{cmr}
    \substitutefont{T2A}{\sfdefault}{cmss}
    \substitutefont{T2A}{\ttdefault}{cmtt}
  \fi
\else
% X2 was declared as font encoding
  \substitutefont{X2}{\rmdefault}{cmr}
  \substitutefont{X2}{\sfdefault}{cmss}
  \substitutefont{X2}{\ttdefault}{cmtt}
\fi


\usepackage[Bjarne]{fncychap}
\usepackage{sphinx}

\fvset{fontsize=\small}
\usepackage{geometry}


% Include hyperref last.
\usepackage{hyperref}
% Fix anchor placement for figures with captions.
\usepackage{hypcap}% it must be loaded after hyperref.
% Set up styles of URL: it should be placed after hyperref.
\urlstyle{same}

\addto\captionsenglish{\renewcommand{\contentsname}{Contents:}}

\usepackage{sphinxmessages}
\setcounter{tocdepth}{2}



\title{StuRa \sphinxhyphen{} Mitgliederdatenbank}
\date{Aug 13, 2020}
\release{}
\author{Team T8}
\newcommand{\sphinxlogo}{\vbox{}}
\renewcommand{\releasename}{}
\makeindex
\begin{document}

\pagestyle{empty}
\sphinxmaketitle
\pagestyle{plain}
\sphinxtableofcontents
\pagestyle{normal}
\phantomsection\label{\detokenize{index::doc}}



\chapter{User Documentation}
\label{\detokenize{masterUserDoc:user-documentation}}\label{\detokenize{masterUserDoc::doc}}

\section{Working with Members (Mitglieder)}
\label{\detokenize{masterUserDoc:working-with-members-mitglieder}}

\subsection{Create a Member}
\label{\detokenize{masterUserDoc:create-a-member}}
Pre\sphinxhyphen{}conditions:
\begin{itemize}
\item {} 
login as Admin

\end{itemize}

To manage Members of the StuRa\sphinxhyphen{}Mitgliederdatenbank you first have to navigate
to the Member section of the application.

\noindent\sphinxincludegraphics{{navigateMitglieder}.png}

Now you have to click on the Button “+ Hinzufügen”

\noindent\sphinxincludegraphics{{createMitglied1}.png}

\noindent\sphinxincludegraphics{{createMitglied2}.png}

With that example form you see above you can fill in personal data for
each member of the StuRa. You can add multiple functions the Member belongs to
and the legislative period for that function. You can also add multiple
E\sphinxhyphen{}Mails for the User. You can save the new Member by clicking on the Button
“Speichern”. You will then be redirected to the Member\sphinxhyphen{}View and your
new Member should appear in the table.


\subsection{Alter a Member}
\label{\detokenize{masterUserDoc:alter-a-member}}
Pre\sphinxhyphen{}conditions:
\begin{itemize}
\item {} 
login as Admin

\end{itemize}

If you are in the Member section you can simply modify a Member by
clicking on the pen on the right in the row of a Member.

\noindent\sphinxincludegraphics{{alterMitglieder}.png}

Now you can modify all data just like if you would create a Member.
To save your changes click the Button “Speichern”.


\subsection{Delete a Member}
\label{\detokenize{masterUserDoc:delete-a-member}}
Pre\sphinxhyphen{}conditions:
\begin{itemize}
\item {} 
login as Admin

\end{itemize}

If you made a mistake or a Member is leaving the StuRa you can delete
the Member in a few steps.
\begin{enumerate}
\sphinxsetlistlabels{\arabic}{enumi}{enumii}{}{.}%
\item {} 
Check the Members you want to delete

\item {} 
Click on the Button “Entfernen”

\item {} 
Check the Dialog and accept it

\item {} 
Check success prompt

\end{enumerate}

\noindent\sphinxincludegraphics{{deleteMember}.png}

\noindent\sphinxincludegraphics{{deleteMember2}.png}

\noindent\sphinxincludegraphics{{deleteMember3}.png}

Now you successfully deleted the Member/s.


\section{Working with Organizational units (Organisationseinheiten)}
\label{\detokenize{masterUserDoc:working-with-organizational-units-organisationseinheiten}}

\subsection{Add a new Organizational unit}
\label{\detokenize{masterUserDoc:add-a-new-organizational-unit}}
Pre\sphinxhyphen{}conditions:
\begin{itemize}
\item {} 
login as Admin

\end{itemize}

First step is to get into the admin section of the application.

\noindent\sphinxincludegraphics{{navigateAdminPanel}.png}

Then we have to move to the correct section to manage our Organizational units.

\noindent\sphinxincludegraphics{{navigateOrganisationseinheiten}.png}

Now to add a new Organizational unit we need to click on the Button
“Organisationseinheit hinzufügen”.

\noindent\sphinxincludegraphics{{alterOrganisationseinheit}.png}

You can fill in the appearing form and add on new functions and/or
subsections of this Organizational unit. At last you need to Save everything
with the Button “Sichern” on the lower right of the form.

\noindent\sphinxincludegraphics{{OrganisationseinheitSuccess}.png}

If you are Prompted
“Organisationseinheit „\textless{}Your Organisational unit\textgreater{}“ wurde erfolgreich hinzugefügt.”
then everything succeeded and you are finished.


\subsection{Alter an Organizational unit}
\label{\detokenize{masterUserDoc:alter-an-organizational-unit}}
Pre\sphinxhyphen{}conditions:
\begin{itemize}
\item {} 
login as Admin

\end{itemize}

First step is to get into the admin section of the application and
then we have to move to the correct section to manage our Organizational units.

\noindent\sphinxincludegraphics{{navigateOrganisationseinheiten}.png}

Now we have to click on the Organizational unit we want to alter, in the
following form you can modify everything you want corresponding to the
Organizational unit.

\noindent\sphinxincludegraphics{{alterOrganisationseinheit}.png}

To confirm your modifications you have to click on the Button “sichern”.
If you are Prompted
“Organisationseinheit „\textless{}Your Organisational unit\textgreater{}“ wurde erfolgreich geändert.”
then everything succeded and you are finished.


\subsection{Delete an Organizational unit}
\label{\detokenize{masterUserDoc:delete-an-organizational-unit}}
Pre\sphinxhyphen{}conditions:
\begin{itemize}
\item {} 
login as Admin

\end{itemize}

First step is to get into the admin section of the application and
then we have to move to the correct section to manage our Organizational units.

\noindent\sphinxincludegraphics{{navigateOrganisationseinheiten}.png}

Now we have to check the Checkbox of the Organizational unit we want to delete.
After this we need to select in Aktion “Ausgewählte Organisationseinheiten löschen”
and click on the Button “Ausführen”.

\noindent\sphinxincludegraphics{{OrganisationseinheitDelete}.png}

To confirm our delete command you have to click on the Button “Ja, ich bin sicher”.
If you are prompted “Erfolgreich \textless{}Number of Organizational units\textgreater{} Organisationseinheit gelöscht.”
everything succeeded.


\section{Working with Subsections (Unterbereiche)}
\label{\detokenize{masterUserDoc:working-with-subsections-unterbereiche}}

\subsection{Create a new Subsection}
\label{\detokenize{masterUserDoc:create-a-new-subsection}}
Pre\sphinxhyphen{}conditions:
\begin{itemize}
\item {} 
login as Admin

\end{itemize}

First step is to get into the admin section of the application.
Now we have 2 Options to create a new Subsection of an Organizational unit.

\noindent\sphinxincludegraphics{{createUnterbereiche2Options}.png}


\subsubsection{First Option via Organizational units}
\label{\detokenize{masterUserDoc:first-option-via-organizational-units}}
Navigate and click on the Organizational unit the Subsection belongs to.
Now we can start to add a new Subsection, first we click on the Button
“Unterbereich hinzufügen”. Second we need to name the new Subsection and
save our changes with the Button “Sichern und weiter bearbeiten”.
If you are prompted with a success message everything succeded.

\noindent\sphinxincludegraphics{{createUnterbereicheOrganisationseinheiten}.png}


\subsubsection{Second Option via Subsections}
\label{\detokenize{masterUserDoc:second-option-via-subsections}}
Navigate to the Subsection section and then click on the Button “Unterbereich hinzufügen”.

\noindent\sphinxincludegraphics{{createUnterbereiche}.png}

Now we can give our new Subsection a name and choose the Organizational unit
the Subsection belongs to.

\noindent\sphinxincludegraphics{{createUnterbereicheForm}.png}

If you are prompted with a success message everything succeded.


\subsection{Alter a Subsection}
\label{\detokenize{masterUserDoc:alter-a-subsection}}
Pre\sphinxhyphen{}conditions:
\begin{itemize}
\item {} 
login as Admin

\end{itemize}

To Alter a Subsection you can navigate to the Subsection section. There you can choose
the Subsection you want to modify. Now you can:
\begin{enumerate}
\sphinxsetlistlabels{\arabic}{enumi}{enumii}{}{.}%
\item {} 
Change the name of the subsection

\item {} 
Change the Organizational unit the subsection belongs to

\end{enumerate}

\noindent\sphinxincludegraphics{{alterUnterbereich}.png}

To save your modifications click the Button “Sichern und weiter bearbeiten”
and don’t forget to check the success message.


\subsection{Delete a Subsection}
\label{\detokenize{masterUserDoc:delete-a-subsection}}
Pre\sphinxhyphen{}conditions:
\begin{itemize}
\item {} 
login as Admin

\end{itemize}

First step is to get into the admin section of the application.
Navigate and click on the Organizational unit the Subsection you want to delete
belongs to. Now we can check the subsections we want to delete from the
organizational unit. At last we can save the modifications with the Button
“Sichern und weiter bearbeiten”.

\noindent\sphinxincludegraphics{{deleteUnterbereiche}.png}

Check the success message to ensure that the Subsection was deleted successfully.

\noindent\sphinxincludegraphics{{deleteUnterbereiche2}.png}


\section{Working with Functions (Funktionen)}
\label{\detokenize{masterUserDoc:working-with-functions-funktionen}}

\subsection{Add a new Function}
\label{\detokenize{masterUserDoc:add-a-new-function}}
Pre\sphinxhyphen{}conditions:
\begin{itemize}
\item {} 
login as Admin

\end{itemize}

First step is to get into the admin section of the application.
Now we have 2 different Options to create a new Function.


\subsubsection{Create a Function that belongs directly to an Organizational unit}
\label{\detokenize{masterUserDoc:create-a-function-that-belongs-directly-to-an-organizational-unit}}\label{\detokenize{masterUserDoc:createfunctionorganizationalunit}}
To create a Function of a Organizational unit you first have to navigate
to the Organizational unit you want to add the Function to. Then follow these
steps to create a new Function:
\begin{enumerate}
\sphinxsetlistlabels{\arabic}{enumi}{enumii}{}{.}%
\item {} 
Click on the Button “Funktion hinzufügen”

\item {} 
Name the new Function

\item {} 
Set a Workload for the Function

\item {} 
Set the maximum of allowed Members in the Function

\item {} 
Save your changes with the Button “Sichern und weiter bearbeiten”

\item {} 
Check the success message

\end{enumerate}

\noindent\sphinxincludegraphics{{createFunktionOrganisationseinheiten}.png}


\subsubsection{Create a Function that belongs to a Subsection of an Organizational unit}
\label{\detokenize{masterUserDoc:create-a-function-that-belongs-to-a-subsection-of-an-organizational-unit}}
To create a Function of a Subsection you first have to navigate to the Subsection
you want to add the Function to. Then follow these steps to create a new Function:

You can follow the same steps like in \DUrole{xref,std,std-ref}{\_createFunctionOrganizationalUnit}.

\noindent\sphinxincludegraphics{{createFunktionOrganisationseinheiten}.png}


\subsection{Alter a Function}
\label{\detokenize{masterUserDoc:alter-a-function}}
Pre\sphinxhyphen{}conditions:
\begin{itemize}
\item {} 
login as Admin

\end{itemize}

To alter a Function you first need to know which Organizational unit or
Subsection the Function belongs to. If you found it you can navigate to it
through the admin section of the application. Now you can:
\begin{enumerate}
\sphinxsetlistlabels{\arabic}{enumi}{enumii}{}{.}%
\item {} 
Change the name of the Function

\item {} 
Change the workload of a Function or

\item {} 
Change the maximum ammount of Members in the Function

\end{enumerate}

To save your modifications click the button “Sichern und weiter bearbeiten”
and don’t forget to check the success message.


\subsection{Delete a Function}
\label{\detokenize{masterUserDoc:delete-a-function}}
Pre\sphinxhyphen{}conditions:
\begin{itemize}
\item {} 
login as Admin

\end{itemize}

To alter a Function you first need to know which Organizational unit or
Subsection the Function belongs to. If you found it you can navigate to it
through the admin section of the application. Now we can check the Functions
we want to delete from the Organizational unit or Subsection.
At last we can save the modifications with the Button
“Sichern und weiter bearbeiten”.
Check the success message to see the changes have been made successfully


\section{Working with Checklists (Checklisten)}
\label{\detokenize{masterUserDoc:working-with-checklists-checklisten}}

\subsection{Create a new General Task}
\label{\detokenize{masterUserDoc:create-a-new-general-task}}
Pre\sphinxhyphen{}conditions:
\begin{itemize}
\item {} 
login as Admin

\end{itemize}

First step is to get into the admin section of the application.

\noindent\sphinxincludegraphics{{navigateAdminPanel}.png}

Now you have to move to the Tasks section of the application.

\noindent\sphinxincludegraphics{{Aufgaben}.png}

If you now click on the Button “Aufgabe hinzufügen” you can add the new Task.
You have to name the new Task and save it with the Button “Sichern”

\noindent\sphinxincludegraphics{{newAufgaben}.png}

At last check the success message.


\subsection{Create Right for a Function}
\label{\detokenize{masterUserDoc:create-right-for-a-function}}
Pre\sphinxhyphen{}conditions:
\begin{itemize}
\item {} 
login as Admin

\end{itemize}

First step is to get into the admin section of the application.

\noindent\sphinxincludegraphics{{navigateAdminPanel}.png}

Now you have to move to the rights section of the application.
You can simply create a new right like in “Create a new General Task”.
If you created a right you now need to link it with a Function.
To do it, first you need to navigate in the admin section to the Function\sphinxhyphen{}Right
Subsection.

\noindent\sphinxincludegraphics{{Funktion-Recht}.png}

If you are there you can add a new link with the Button
“Zuordnung Funktion\sphinxhyphen{}Recht hinzufügen”.

\noindent\sphinxincludegraphics{{createFunktion-Recht}.png}

There you have to select a Function and a right you want to link with and save
by clicking the button “Sichern”.


\subsection{Create a new Checklist}
\label{\detokenize{masterUserDoc:create-a-new-checklist}}
Pre\sphinxhyphen{}conditions:
\begin{itemize}
\item {} 
login as Admin

\item {} 
Rights for a Function

\item {} 
\_\_optional:\_\_ General Tasks have been created through the admin panel

\end{itemize}

To manage Checklisten you have to go to the Checklisten\sphinxhyphen{}View of the Application.

\noindent\sphinxincludegraphics{{Checklisten}.png}

If you are there click on the big panel with the label “+” to add a new
checklist.

\noindent\sphinxincludegraphics{{createChecklisten}.png}

You have to select the Member which belongs to the Checklist and optionally a
Function the Member is associated with. The checkbox General
Tasks can only be deselected if you choose a function. Create a new Checklist by
clicking on the Button “Erstellen”.

\noindent\sphinxincludegraphics{{createdChecklisten}.png}


\subsection{Delete a Checklist}
\label{\detokenize{masterUserDoc:delete-a-checklist}}
Pre\sphinxhyphen{}conditions:
\begin{itemize}
\item {} 
login as Admin

\end{itemize}

To delete a Checklist you need to click on the garbage can on the lower left
of a checklist and accept the dialog.

\noindent\sphinxincludegraphics{{deleteChecklisten}.png}


\section{Working with Users and Admins}
\label{\detokenize{masterUserDoc:working-with-users-and-admins}}

\subsection{Create an User}
\label{\detokenize{masterUserDoc:create-an-user}}
Pre\sphinxhyphen{}conditions:
\begin{itemize}
\item {} 
login as Admin

\end{itemize}

First step is to get into the admin section of the application.
Now you have to go to the User section of the admin page.

\noindent\sphinxincludegraphics{{UserSection}.png}

Now you see a overview of all Users of the Software. To add a
new User you can click on the Button “Benutzer hinzufügen” on the
upper right of the page. You have to fill in the form and accept your input
with the Button “Sichern und weiter bearbeiten”

\noindent\sphinxincludegraphics{{createUser}.png}

Now you can add some more personal information to the user account if you want
and you can give the user some more rights for the Page. (More in the section
create an admin)

\noindent\sphinxincludegraphics{{createUserMore}.png}


\subsection{Create a User (with Admin rights)}
\label{\detokenize{masterUserDoc:create-a-user-with-admin-rights}}
Pre\sphinxhyphen{}conditions:
\begin{itemize}
\item {} 
login as Admin

\item {} 
successfully created User

\end{itemize}

If you successfully created a User with the section above you can Upgrade
the User account to an Admin account. First you need to navigate to the
User you want to give administrative privileges and then you have to simply check
the Checkboxes “Mitarbeiter\sphinxhyphen{}Status” and “Administrator\sphinxhyphen{}Status”

\noindent\sphinxincludegraphics{{createUserAdmin}.png}

Finally you need to save your changes with a click on the Button
“Sichern und weiter bearbeiten”


\chapter{Admin Documentation}
\label{\detokenize{masterAdminDoc:admin-documentation}}\label{\detokenize{masterAdminDoc::doc}}

\section{Deployment}
\label{\detokenize{masterAdminDoc:deployment}}
This section contains a step\sphinxhyphen{}by\sphinxhyphen{}step guidance how to deploy this software.
It\textasciigrave{}s based on a deployment of Django with Apache2 and mod\_wsgi on a raspberry pi 4.


\subsection{Sources:}
\label{\detokenize{masterAdminDoc:sources}}\begin{itemize}
\item {} 
\sphinxurl{https://docs.djangoproject.com/en/3.0/howto/deployment/}

\item {} 
\sphinxurl{https://wiki.ubuntuusers.de/Apache\_2.4/}

\item {} 
\sphinxurl{https://www.digitalocean.com/community/tutorials/how-to-serve-django-applications-with-apache-and-mod\_wsgi-on-ubuntu-16-04}

\end{itemize}


\subsection{Requirements:}
\label{\detokenize{masterAdminDoc:requirements}}\begin{itemize}
\item {} 
a web server

\item {} 
basic knowledge of bash

\item {} 
Python 3

\end{itemize}


\subsection{Basic configuration:}
\label{\detokenize{masterAdminDoc:basic-configuration}}
Update your distro
\begin{quote}

\sphinxcode{\sphinxupquote{sudo apt update \&\& sudo apt upgrade}}
\end{quote}

Install Dependencies for the deployment. (We use Python3)
\begin{quote}

\sphinxcode{\sphinxupquote{sudo apt\sphinxhyphen{}get install python3\sphinxhyphen{}pip apache2 libapache2\sphinxhyphen{}mod\sphinxhyphen{}wsgi\sphinxhyphen{}py3}}
\end{quote}

You also need to install virtualenv, to seperate Python from your system’s python.
Important is that you use python 3 because we use the Apache mod\_wsgi for
python 3.
\begin{quote}

\sphinxcode{\sphinxupquote{sudo pip3 install virtualenv}}
\end{quote}

Now we need to clone the Git\sphinxhyphen{}Repository and setup the virtualenv for Python.
First you need to change to the directory that you want to clone this web application to.
Then:
\begin{quote}

\sphinxcode{\sphinxupquote{git clone https://github.com/mribrgr/StuRa\sphinxhyphen{}Mitgliederdatenbank.git}}
\sphinxcode{\sphinxupquote{cd StuRa\sphinxhyphen{}Mitgliederdatenbank/}}
\end{quote}

Now create a virtual environment and activate it.
\begin{quote}

\begin{DUlineblock}{0em}
\item[] \sphinxcode{\sphinxupquote{virtualenv venv}}
\item[] \sphinxcode{\sphinxupquote{source venv/bin/activate}}
\end{DUlineblock}
\end{quote}

If you have successfully activated your virtual ennvironment than your prompt should
look something like this \sphinxcode{\sphinxupquote{(venv) user@host:StuRa\sphinxhyphen{}Mitgliederdatenbank}}. The last
step is to install the requirements.txt in the virtual environement.
\begin{quote}

\sphinxcode{\sphinxupquote{pip install \sphinxhyphen{}r requirements.txt}}
\end{quote}


\subsection{Adjust Django Project Settings:}
\label{\detokenize{masterAdminDoc:adjust-django-project-settings}}
First we need to configure \sphinxcode{\sphinxupquote{StuRa\sphinxhyphen{}Mitgliederdatenbank/bin/settings.py}}:
We open the file first (based on the previous chapter)
\begin{quote}

\sphinxcode{\sphinxupquote{nano bin/settings.py}}
\end{quote}

For a productive enviroment set the debug output to false.
\begin{quote}

\sphinxcode{\sphinxupquote{DEBUG = False}}
\end{quote}

Here we need to register our server’s public IP address or domain name.
Replace “IP\_or\_DOMAIN” with your personal IP address or domain name.
\begin{quote}

\sphinxcode{\sphinxupquote{ALLOWED\_HOSTS = {[}"IP\_or\_DOMAIN"{]}}}
\end{quote}

At the bottom of the settings.py we need to define a static directory for our static html files.
\begin{quote}

\sphinxcode{\sphinxupquote{STATIC\_ROOT = os.path.join(BASE\_DIR, \textquotesingle{}mystatic/\textquotesingle{})}}
\end{quote}

Now we can close and save the file.
After this you need to create the folder static in the directory StuRa\sphinxhyphen{}Mitgliederdatenbank
with the command.
\begin{quote}

\sphinxcode{\sphinxupquote{mkdir static}}
\end{quote}

The last step is to do initial commands:
\begin{quote}

\begin{DUlineblock}{0em}
\item[] \sphinxcode{\sphinxupquote{python ./manage.py makemigrations}}
\item[] \sphinxcode{\sphinxupquote{python ./manage.py migrate}}
\item[] \sphinxcode{\sphinxupquote{python ./manage.py collectstatic}}
\item[] \sphinxcode{\sphinxupquote{python ./manage.py createsuperuser}}
\end{DUlineblock}
\end{quote}

An optional step that can be done is to fill in some functions that are common
to the StuRa of the HTW\sphinxhyphen{}Dresden.
\begin{quote}

\begin{DUlineblock}{0em}
\item[] \sphinxcode{\sphinxupquote{cd importscripts}}
\item[] \sphinxcode{\sphinxupquote{python main.py}}
\end{DUlineblock}
\end{quote}

Now wait a little moment and then you can change to the parent directory.
\begin{quote}

\sphinxcode{\sphinxupquote{cd ..}}
\end{quote}

And deactivate the virtual environment:
\begin{quote}

\sphinxcode{\sphinxupquote{deactivate}}
\end{quote}


\subsection{Configure Apache2:}
\label{\detokenize{masterAdminDoc:configure-apache2}}
To enable Apache2 as front end we need to configure WSGI pass.
To achieve this we need to edit the default virtual host file:
\begin{quote}

\sphinxcode{\sphinxupquote{sudo nano /etc/apache2/sites\sphinxhyphen{}available/000\sphinxhyphen{}default.conf}}
\end{quote}

We can keep everything that is present in this file and add our config above
the last \sphinxcode{\sphinxupquote{\textless{}/VirtualHost\textgreater{}}} tag. What we first specify is the static directory
and the path to the wsgi.py.
(In this Example I have cloned the directory in \textasciitilde{}/StuRa\sphinxhyphen{}Mitgliederdatenbank)
(pi is my username change it to yours)
\sphinxSetupCaptionForVerbatim{/etc/apache2/sites\sphinxhyphen{}available/000\sphinxhyphen{}default.conf}
\def\sphinxLiteralBlockLabel{\label{\detokenize{masterAdminDoc:id1}}}
\begin{sphinxVerbatim}[commandchars=\\\{\}]
\PYGZlt{}VirtualHost *:80\PYGZgt{}

  . . .

  Alias /static /home/pi/StuRa\PYGZhy{}Mitgliederdatenbank/mystatic
  \PYGZlt{}Directory /home/pi/StuRa\PYGZhy{}Mitgliederdatenbank/mystatic\PYGZgt{}
      Require all granted
  \PYGZlt{}/Directory\PYGZgt{}

  \PYGZlt{}Directory /home/pi/StuRa\PYGZhy{}Mitgliederdatenbank/bin\PYGZgt{}
    \PYGZlt{}Files wsgi.py\PYGZgt{}
      Require all granted
    \PYGZlt{}/Files\PYGZgt{}
  \PYGZlt{}/Directory\PYGZgt{}

\PYGZlt{}/VirtualHost\PYGZgt{}
\end{sphinxVerbatim}

Now we add the recommended deamon mode to the WSGI process.
To do it you need to append the following lines to the Apache config.
\sphinxSetupCaptionForVerbatim{/etc/apache2/sites\sphinxhyphen{}available/000\sphinxhyphen{}default.conf}
\def\sphinxLiteralBlockLabel{\label{\detokenize{masterAdminDoc:id2}}}
\begin{sphinxVerbatim}[commandchars=\\\{\}]
\PYGZlt{}VirtualHost *:80\PYGZgt{}

  . . .

  WSGIDaemonProcess StuRa\PYGZhy{}Mitgliederdatenbank python\PYGZhy{}home\PYG{o}{=}/home/pi/StuRa\PYGZhy{}Mitgliederdatenbank/venv python\PYGZhy{}path\PYG{o}{=}/home/pi/StuRa\PYGZhy{}Mitgliederdatenbank
  WSGIProcessGroup StuRa\PYGZhy{}Mitgliederdatenbank
  WSGIScriptAlias / /home/pi/StuRa\PYGZhy{}Mitgliederdatenbank/bin/wsgi.py

\PYGZlt{}/VirtualHost\PYGZgt{}
\end{sphinxVerbatim}


\subsection{Solve some Permission Issues:}
\label{\detokenize{masterAdminDoc:solve-some-permission-issues}}
The first step is to change the permissions of the database, so that group owner
can read and write. Then we need to transfer the ownership of some files to Apache2
group and user \sphinxcode{\sphinxupquote{www\sphinxhyphen{}data}}.
\begin{quote}

\begin{DUlineblock}{0em}
\item[] \sphinxcode{\sphinxupquote{chmod 664 \textasciitilde{}/StuRa\sphinxhyphen{}Mitgliederdatenbank/db.sqlite3}}
\item[] \sphinxcode{\sphinxupquote{sudo chown www\sphinxhyphen{}data:www\sphinxhyphen{}data \textasciitilde{}/StuRa\sphinxhyphen{}Mitgliederdatenbank/db.sqlite3}}
\item[] \sphinxcode{\sphinxupquote{sudo chown www\sphinxhyphen{}data:www\sphinxhyphen{}data \textasciitilde{}/StuRa\sphinxhyphen{}Mitgliederdatenbank}}
\end{DUlineblock}
\end{quote}

If you got firewall issues, allow Apache to acces the firewall example:
\begin{quote}

\sphinxcode{\sphinxupquote{sudo ufw allow \textquotesingle{}Apache Full\textquotesingle{}}}
\end{quote}

Last but not least check the Apache files if everything is correct:
\begin{quote}

\sphinxcode{\sphinxupquote{sudo apache2ctl configtest}}
\end{quote}

If the output looks like \sphinxcode{\sphinxupquote{Syntax OK}} you are done and can restart your apache2
service:
\begin{quote}

\sphinxcode{\sphinxupquote{sudo systemctl restart apache2}}
\end{quote}


\section{Cronjobs}
\label{\detokenize{masterAdminDoc:cronjobs}}
In the following lines there is an explanation how to create cronjobs to
schedule tasks. These tasks help to cleanup the database of the application.

First make sure you have installed cron than you can add cronjobs
with the command:
\begin{quote}

\sphinxcode{\sphinxupquote{crontab \sphinxhyphen{}e}}
\end{quote}

Now you can see a file like this:

At the bottom of the file you can add your personal cronjobs.
To keep it simple our recommendation is to create a script with all commands you
want (described in the commands section of the code documentation).
You can easily schedule this script to run once a week with cronjob.
\begin{quote}

\sphinxcode{\sphinxupquote{0 0 * * 0 bash /path/to/script}}
\end{quote}


\section{Update the Application}
\label{\detokenize{masterAdminDoc:update-the-application}}
This section describes how to update the application from an existing deployment.


\subsection{Update From GitHub with no changes in migrations:}
\label{\detokenize{masterAdminDoc:update-from-github-with-no-changes-in-migrations}}
First you need to get the ownership back from www\sphinxhyphen{}data.
In my example my user is \sphinxstyleemphasis{pi} and the application is locatetd
in pi’s home directory.
\begin{quote}

\begin{DUlineblock}{0em}
\item[] \sphinxcode{\sphinxupquote{sudo chown pi:pi \textasciitilde{}/StuRa\sphinxhyphen{}Mitgliederdatenbank/db.sqlite3}}
\item[] \sphinxcode{\sphinxupquote{sudo chown pi:pi \textasciitilde{}/StuRa\sphinxhyphen{}Mitgliederdatenbank}}
\end{DUlineblock}
\end{quote}

Now we need to stash our changes we have done during our config.
\begin{quote}

\sphinxcode{\sphinxupquote{git stash}}
\end{quote}

We can now pull the latest version from the git Repository
\begin{quote}

\sphinxcode{\sphinxupquote{git pull}}
\end{quote}

To apply our configs back we need to get them back from the stash
\begin{quote}

\sphinxcode{\sphinxupquote{git stash pop}}
\end{quote}
\begin{description}
\item[{At last you need to give the ownership back to www\sphinxhyphen{}data.}] \leavevmode
\begin{DUlineblock}{0em}
\item[] \sphinxcode{\sphinxupquote{sudo chown www\sphinxhyphen{}data:www\sphinxhyphen{}data \textasciitilde{}/StuRa\sphinxhyphen{}Mitgliederdatenbank/db.sqlite3}}
\item[] \sphinxcode{\sphinxupquote{sudo chown www\sphinxhyphen{}data:www\sphinxhyphen{}data \textasciitilde{}/StuRa\sphinxhyphen{}Mitgliederdatenbank}}
\end{DUlineblock}

\end{description}


\chapter{Developer Documentation}
\label{\detokenize{masterDeveloperDoc:developer-documentation}}\label{\detokenize{masterDeveloperDoc::doc}}

\section{Introduction Development}
\label{\detokenize{masterDeveloperDoc:introduction-development}}
This and the following sections deal with development specific topics.


\subsection{Clone the Git\sphinxhyphen{}Repository, install the requirements and fire up the Webserver for developement}
\label{\detokenize{masterDeveloperDoc:clone-the-git-repository-install-the-requirements-and-fire-up-the-webserver-for-developement}}
Requirements:
\begin{itemize}
\item {} 
A installation of git

\item {} 
A Installation of Python (we used 3.8.2)

\item {} 
(optional) a installation of virtualenviroment

\item {} 
A Webbrowser

\end{itemize}

First step is to pull the Git\sphinxhyphen{}Repository
\begin{quote}

\sphinxcode{\sphinxupquote{git clone https://github.com/mribrgr/StuRa\sphinxhyphen{}Mitgliederdatenbank.git}}
\end{quote}

Next we change the directory to the pulled Git\sphinxhyphen{}Repository with:
\begin{quote}

\sphinxcode{\sphinxupquote{cd StuRa\sphinxhyphen{}Mitgliederdatenbank/}}
\end{quote}

Now we need to install all python requirements (optional: install it in a virtualenviroment).
You can use the requirements.txt, it contains all requirements we used.
\begin{quote}

\sphinxcode{\sphinxupquote{pip install \sphinxhyphen{}r requirements.txt}}
\end{quote}

The Basics are done, now we need to create a Database and apply all migrations to it. We use the
makemigrations and the migrate command from the manage.py.
\begin{quote}

\begin{DUlineblock}{0em}
\item[] \sphinxcode{\sphinxupquote{python3 ./manage.py makemigrations}}
\item[] \sphinxcode{\sphinxupquote{python3 ./manage.py migrate}}
\end{DUlineblock}
\end{quote}

If the Database was successfully created there must be a db.sqlite3 in the directory.
As last step, before we can fire up the server, we need to create an admin account
using the command superuser so that we can login into the application.
\begin{quote}

\sphinxcode{\sphinxupquote{python3 ./manage.py createsuperuser}}
\end{quote}

Now we can start the server by typing:
\begin{quote}

\sphinxcode{\sphinxupquote{python3 ./manage.py runserver}}
\end{quote}


\section{System Architecture}
\label{\detokenize{masterDeveloperDoc:system-architecture}}
The architecture of the Django Webframework is a Standard MVC architecture.
Below is an example of the basic workflow of the Django Framework:

\sphinxhref{\_images/diagramm\_django\_flow.png}{\sphinxincludegraphics{{diagramm_django_flow}.png}}

The layers of the Application look like this:

\sphinxhref{\_images/diagramm\_layers.png}{\sphinxincludegraphics{{diagramm_layers}.png}}


\section{The V in MVC}
\label{\detokenize{masterDeveloperDoc:the-v-in-mvc}}
This section gives an overview over the function of the \sphinxstyleemphasis{views.py} in each
module.

A generic Example:

\sphinxhref{\_images/diagramm\_activity\_django\_view.png}{\sphinxincludegraphics{{diagramm_activity_django_view}.png}}

The generic sequence diagram from the Example above:

\sphinxhref{\_images/diagramm\_sequence\_django\_view.png}{\sphinxincludegraphics{{diagramm_sequence_django_view}.png}}

A example for the \sphinxstyleemphasis{mitglieder/views.py}:

\sphinxhref{\_images/diagramm\_activity\_django\_view\_example.png}{\sphinxincludegraphics{{diagramm_activity_django_view_example}.png}}


\section{Components}
\label{\detokenize{masterDeveloperDoc:components}}
A brief overview over the Packages

\sphinxhref{\_images/diagramm\_package\_classes.png}{\sphinxincludegraphics{{diagramm_package_classes}.png}}

A brief overview over the Components of a Package

\sphinxhref{\_images/diagramm\_package\_components.png}{\sphinxincludegraphics{{diagramm_package_components}.png}}

A brief overview over the connections between the Classes

\sphinxhref{\_images/diagramm\_class.png}{\sphinxincludegraphics{{diagramm_class}.png}}


\section{Database Structure}
\label{\detokenize{masterDeveloperDoc:database-structure}}
The Database is a SQLite3 Database, the Tables are specified in each Packages
model.py and in a scheme it looks like:

\sphinxhref{\_images/diagramm\_domain.png}{\sphinxincludegraphics{{diagramm_domain}.png}}


\section{Activity Diagrams}
\label{\detokenize{masterDeveloperDoc:activity-diagrams}}
Here you can see two examples for activity diagrams. The first one is based
on all workflows we wanted to cover with our software.

\sphinxhref{\_images/diagramm\_activity\_mitglieder\_verwalten.png}{\sphinxincludegraphics{{diagramm_activity_mitglieder_verwalten}.png}}

The second one describes the workflow Django covers with its admin panel.

\sphinxhref{\_images/diagramm\_activity\_nutzer\_verwalten.png}{\sphinxincludegraphics{{diagramm_activity_nutzer_verwalten}.png}}


\chapter{Code Documentation}
\label{\detokenize{masterCodeDoc:code-documentation}}\label{\detokenize{masterCodeDoc::doc}}

\section{Login}
\label{\detokenize{masterCodeDoc:login}}
This app is responsible for authenticating users.


\subsection{Views}
\label{\detokenize{masterCodeDoc:module-login.views}}\label{\detokenize{masterCodeDoc:views}}\index{module@\spxentry{module}!login.views@\spxentry{login.views}}\index{login.views@\spxentry{login.views}!module@\spxentry{module}}\index{logout\_request() (in module login.views)@\spxentry{logout\_request()}\spxextra{in module login.views}}

\begin{fulllineitems}
\phantomsection\label{\detokenize{masterCodeDoc:login.views.logout_request}}\pysiglinewithargsret{\sphinxcode{\sphinxupquote{login.views.}}\sphinxbfcode{\sphinxupquote{logout\_request}}}{\emph{\DUrole{n}{request}}}{}
This view processes all logout requests made by navigating to /logout. It logs the user out and displays a goodbye message.
\begin{quote}\begin{description}
\item[{Parameters}] \leavevmode
\sphinxstyleliteralstrong{\sphinxupquote{request}} \textendash{} The HTTP request that triggered the view.

\item[{Returns}] \leavevmode
A redirect to the app’s root URL (i.e. the login form).

\end{description}\end{quote}

\end{fulllineitems}

\index{main\_screen() (in module login.views)@\spxentry{main\_screen()}\spxextra{in module login.views}}

\begin{fulllineitems}
\phantomsection\label{\detokenize{masterCodeDoc:login.views.main_screen}}\pysiglinewithargsret{\sphinxcode{\sphinxupquote{login.views.}}\sphinxbfcode{\sphinxupquote{main\_screen}}}{\emph{\DUrole{n}{request}}}{}
This view processes all login requests made via the form found at the app’s root URL.
If a user who is already logged in tries to access the page, they will automatically be redirected to /mitglieder.
If the form was submitted, the view gets all data from the submitted form and tries to authenticate the user using that data. 
If authentication is sucessful, the user will be logged in and shown an appropriate welcome message. Otherwise, or if the submitted form is invalid, the user will be shown an error message.
If the user navigates to the login form (i.e. is not submitting any data), the AuthenticationForm provided by Django will be rendered.
\begin{quote}\begin{description}
\item[{Parameters}] \leavevmode
\sphinxstyleliteralstrong{\sphinxupquote{request}} \textendash{} The HTTP request that triggered the view.

\item[{Returns}] \leavevmode
The rendered AuthenticationForm if no data was submitted, or a redirect to /mitglieder if the user was logged in successfully.

\end{description}\end{quote}

\end{fulllineitems}



\subsection{Templates}
\label{\detokenize{masterCodeDoc:templates}}
All templates can be found under \sphinxcode{\sphinxupquote{/login/templates/login.}}


\subsubsection{login.html}
\label{\detokenize{masterCodeDoc:login-html}}
Contains the login form using Materialize’s form components. Variables are defined by the AuthenticationForm supplied by Django.


\section{Ämter}
\label{\detokenize{masterCodeDoc:amter}}

\subsection{Models}
\label{\detokenize{masterCodeDoc:module-aemter.models}}\label{\detokenize{masterCodeDoc:models}}\index{module@\spxentry{module}!aemter.models@\spxentry{aemter.models}}\index{aemter.models@\spxentry{aemter.models}!module@\spxentry{module}}\index{Funktion (class in aemter.models)@\spxentry{Funktion}\spxextra{class in aemter.models}}

\begin{fulllineitems}
\phantomsection\label{\detokenize{masterCodeDoc:aemter.models.Funktion}}\pysiglinewithargsret{\sphinxbfcode{\sphinxupquote{class }}\sphinxcode{\sphinxupquote{aemter.models.}}\sphinxbfcode{\sphinxupquote{Funktion}}}{\emph{\DUrole{o}{*}\DUrole{n}{args}}, \emph{\DUrole{o}{**}\DUrole{n}{kwargs}}}{}
Datenbankmodel Funktion

Felder:
\begin{itemize}
\item {} 
bezeichnung

\item {} 
workload

\item {} 
max\_members (Maximale Anzahl an Mitgliedern in der Funktion)

\item {} 
organisationseinheit (Referenziert zugehörige Organisationseinheit)

\item {} \begin{description}
\item[{unterbereich (Referenziert zugehörigen Unterbereich)}] \leavevmode
Unterbereich kann null sein

\end{description}

\item {} 
history

\end{itemize}
\index{Funktion.DoesNotExist@\spxentry{Funktion.DoesNotExist}}

\begin{fulllineitems}
\phantomsection\label{\detokenize{masterCodeDoc:aemter.models.Funktion.DoesNotExist}}\pysigline{\sphinxbfcode{\sphinxupquote{exception }}\sphinxbfcode{\sphinxupquote{DoesNotExist}}}
\end{fulllineitems}

\index{Funktion.MultipleObjectsReturned@\spxentry{Funktion.MultipleObjectsReturned}}

\begin{fulllineitems}
\phantomsection\label{\detokenize{masterCodeDoc:aemter.models.Funktion.MultipleObjectsReturned}}\pysigline{\sphinxbfcode{\sphinxupquote{exception }}\sphinxbfcode{\sphinxupquote{MultipleObjectsReturned}}}
\end{fulllineitems}

\index{bezeichnung (aemter.models.Funktion attribute)@\spxentry{bezeichnung}\spxextra{aemter.models.Funktion attribute}}

\begin{fulllineitems}
\phantomsection\label{\detokenize{masterCodeDoc:aemter.models.Funktion.bezeichnung}}\pysigline{\sphinxbfcode{\sphinxupquote{bezeichnung}}}
A wrapper for a deferred\sphinxhyphen{}loading field. When the value is read from this
object the first time, the query is executed.

\end{fulllineitems}

\index{funktionrecht\_set (aemter.models.Funktion attribute)@\spxentry{funktionrecht\_set}\spxextra{aemter.models.Funktion attribute}}

\begin{fulllineitems}
\phantomsection\label{\detokenize{masterCodeDoc:aemter.models.Funktion.funktionrecht_set}}\pysigline{\sphinxbfcode{\sphinxupquote{funktionrecht\_set}}}
Accessor to the related objects manager on the reverse side of a
many\sphinxhyphen{}to\sphinxhyphen{}one relation.

In the example:

\begin{sphinxVerbatim}[commandchars=\\\{\}]
\PYG{k}{class} \PYG{n+nc}{Child}\PYG{p}{(}\PYG{n}{Model}\PYG{p}{)}\PYG{p}{:}
    \PYG{n}{parent} \PYG{o}{=} \PYG{n}{ForeignKey}\PYG{p}{(}\PYG{n}{Parent}\PYG{p}{,} \PYG{n}{related\PYGZus{}name}\PYG{o}{=}\PYG{l+s+s1}{\PYGZsq{}}\PYG{l+s+s1}{children}\PYG{l+s+s1}{\PYGZsq{}}\PYG{p}{)}
\end{sphinxVerbatim}

\sphinxcode{\sphinxupquote{Parent.children}} is a \sphinxcode{\sphinxupquote{ReverseManyToOneDescriptor}} instance.

Most of the implementation is delegated to a dynamically defined manager
class built by \sphinxcode{\sphinxupquote{create\_forward\_many\_to\_many\_manager()}} defined below.

\end{fulllineitems}

\index{history (aemter.models.Funktion attribute)@\spxentry{history}\spxextra{aemter.models.Funktion attribute}}

\begin{fulllineitems}
\phantomsection\label{\detokenize{masterCodeDoc:aemter.models.Funktion.history}}\pysigline{\sphinxbfcode{\sphinxupquote{history}}\sphinxbfcode{\sphinxupquote{ = \textless{}simple\_history.manager.HistoryManager object\textgreater{}}}}
\end{fulllineitems}

\index{id (aemter.models.Funktion attribute)@\spxentry{id}\spxextra{aemter.models.Funktion attribute}}

\begin{fulllineitems}
\phantomsection\label{\detokenize{masterCodeDoc:aemter.models.Funktion.id}}\pysigline{\sphinxbfcode{\sphinxupquote{id}}}
A wrapper for a deferred\sphinxhyphen{}loading field. When the value is read from this
object the first time, the query is executed.

\end{fulllineitems}

\index{max\_members (aemter.models.Funktion attribute)@\spxentry{max\_members}\spxextra{aemter.models.Funktion attribute}}

\begin{fulllineitems}
\phantomsection\label{\detokenize{masterCodeDoc:aemter.models.Funktion.max_members}}\pysigline{\sphinxbfcode{\sphinxupquote{max\_members}}}
A wrapper for a deferred\sphinxhyphen{}loading field. When the value is read from this
object the first time, the query is executed.

\end{fulllineitems}

\index{mitgliedamt\_set (aemter.models.Funktion attribute)@\spxentry{mitgliedamt\_set}\spxextra{aemter.models.Funktion attribute}}

\begin{fulllineitems}
\phantomsection\label{\detokenize{masterCodeDoc:aemter.models.Funktion.mitgliedamt_set}}\pysigline{\sphinxbfcode{\sphinxupquote{mitgliedamt\_set}}}
Accessor to the related objects manager on the reverse side of a
many\sphinxhyphen{}to\sphinxhyphen{}one relation.

In the example:

\begin{sphinxVerbatim}[commandchars=\\\{\}]
\PYG{k}{class} \PYG{n+nc}{Child}\PYG{p}{(}\PYG{n}{Model}\PYG{p}{)}\PYG{p}{:}
    \PYG{n}{parent} \PYG{o}{=} \PYG{n}{ForeignKey}\PYG{p}{(}\PYG{n}{Parent}\PYG{p}{,} \PYG{n}{related\PYGZus{}name}\PYG{o}{=}\PYG{l+s+s1}{\PYGZsq{}}\PYG{l+s+s1}{children}\PYG{l+s+s1}{\PYGZsq{}}\PYG{p}{)}
\end{sphinxVerbatim}

\sphinxcode{\sphinxupquote{Parent.children}} is a \sphinxcode{\sphinxupquote{ReverseManyToOneDescriptor}} instance.

Most of the implementation is delegated to a dynamically defined manager
class built by \sphinxcode{\sphinxupquote{create\_forward\_many\_to\_many\_manager()}} defined below.

\end{fulllineitems}

\index{objects (aemter.models.Funktion attribute)@\spxentry{objects}\spxextra{aemter.models.Funktion attribute}}

\begin{fulllineitems}
\phantomsection\label{\detokenize{masterCodeDoc:aemter.models.Funktion.objects}}\pysigline{\sphinxbfcode{\sphinxupquote{objects}}\sphinxbfcode{\sphinxupquote{ = \textless{}django.db.models.manager.Manager object\textgreater{}}}}
\end{fulllineitems}

\index{organisationseinheit (aemter.models.Funktion attribute)@\spxentry{organisationseinheit}\spxextra{aemter.models.Funktion attribute}}

\begin{fulllineitems}
\phantomsection\label{\detokenize{masterCodeDoc:aemter.models.Funktion.organisationseinheit}}\pysigline{\sphinxbfcode{\sphinxupquote{organisationseinheit}}}
Accessor to the related object on the forward side of a many\sphinxhyphen{}to\sphinxhyphen{}one or
one\sphinxhyphen{}to\sphinxhyphen{}one (via ForwardOneToOneDescriptor subclass) relation.

In the example:

\begin{sphinxVerbatim}[commandchars=\\\{\}]
\PYG{k}{class} \PYG{n+nc}{Child}\PYG{p}{(}\PYG{n}{Model}\PYG{p}{)}\PYG{p}{:}
    \PYG{n}{parent} \PYG{o}{=} \PYG{n}{ForeignKey}\PYG{p}{(}\PYG{n}{Parent}\PYG{p}{,} \PYG{n}{related\PYGZus{}name}\PYG{o}{=}\PYG{l+s+s1}{\PYGZsq{}}\PYG{l+s+s1}{children}\PYG{l+s+s1}{\PYGZsq{}}\PYG{p}{)}
\end{sphinxVerbatim}

\sphinxcode{\sphinxupquote{Child.parent}} is a \sphinxcode{\sphinxupquote{ForwardManyToOneDescriptor}} instance.

\end{fulllineitems}

\index{organisationseinheit\_id (aemter.models.Funktion attribute)@\spxentry{organisationseinheit\_id}\spxextra{aemter.models.Funktion attribute}}

\begin{fulllineitems}
\phantomsection\label{\detokenize{masterCodeDoc:aemter.models.Funktion.organisationseinheit_id}}\pysigline{\sphinxbfcode{\sphinxupquote{organisationseinheit\_id}}}
\end{fulllineitems}

\index{save\_without\_historical\_record() (aemter.models.Funktion method)@\spxentry{save\_without\_historical\_record()}\spxextra{aemter.models.Funktion method}}

\begin{fulllineitems}
\phantomsection\label{\detokenize{masterCodeDoc:aemter.models.Funktion.save_without_historical_record}}\pysiglinewithargsret{\sphinxbfcode{\sphinxupquote{save\_without\_historical\_record}}}{\emph{\DUrole{o}{*}\DUrole{n}{args}}, \emph{\DUrole{o}{**}\DUrole{n}{kwargs}}}{}
Save model without saving a historical record

Make sure you know what you’re doing before you use this method.

\end{fulllineitems}

\index{tostring() (aemter.models.Funktion method)@\spxentry{tostring()}\spxextra{aemter.models.Funktion method}}

\begin{fulllineitems}
\phantomsection\label{\detokenize{masterCodeDoc:aemter.models.Funktion.tostring}}\pysiglinewithargsret{\sphinxbfcode{\sphinxupquote{tostring}}}{}{}
\end{fulllineitems}

\index{unterbereich (aemter.models.Funktion attribute)@\spxentry{unterbereich}\spxextra{aemter.models.Funktion attribute}}

\begin{fulllineitems}
\phantomsection\label{\detokenize{masterCodeDoc:aemter.models.Funktion.unterbereich}}\pysigline{\sphinxbfcode{\sphinxupquote{unterbereich}}}
Accessor to the related object on the forward side of a many\sphinxhyphen{}to\sphinxhyphen{}one or
one\sphinxhyphen{}to\sphinxhyphen{}one (via ForwardOneToOneDescriptor subclass) relation.

In the example:

\begin{sphinxVerbatim}[commandchars=\\\{\}]
\PYG{k}{class} \PYG{n+nc}{Child}\PYG{p}{(}\PYG{n}{Model}\PYG{p}{)}\PYG{p}{:}
    \PYG{n}{parent} \PYG{o}{=} \PYG{n}{ForeignKey}\PYG{p}{(}\PYG{n}{Parent}\PYG{p}{,} \PYG{n}{related\PYGZus{}name}\PYG{o}{=}\PYG{l+s+s1}{\PYGZsq{}}\PYG{l+s+s1}{children}\PYG{l+s+s1}{\PYGZsq{}}\PYG{p}{)}
\end{sphinxVerbatim}

\sphinxcode{\sphinxupquote{Child.parent}} is a \sphinxcode{\sphinxupquote{ForwardManyToOneDescriptor}} instance.

\end{fulllineitems}

\index{unterbereich\_id (aemter.models.Funktion attribute)@\spxentry{unterbereich\_id}\spxextra{aemter.models.Funktion attribute}}

\begin{fulllineitems}
\phantomsection\label{\detokenize{masterCodeDoc:aemter.models.Funktion.unterbereich_id}}\pysigline{\sphinxbfcode{\sphinxupquote{unterbereich\_id}}}
\end{fulllineitems}

\index{workload (aemter.models.Funktion attribute)@\spxentry{workload}\spxextra{aemter.models.Funktion attribute}}

\begin{fulllineitems}
\phantomsection\label{\detokenize{masterCodeDoc:aemter.models.Funktion.workload}}\pysigline{\sphinxbfcode{\sphinxupquote{workload}}}
A wrapper for a deferred\sphinxhyphen{}loading field. When the value is read from this
object the first time, the query is executed.

\end{fulllineitems}


\end{fulllineitems}

\index{FunktionRecht (class in aemter.models)@\spxentry{FunktionRecht}\spxextra{class in aemter.models}}

\begin{fulllineitems}
\phantomsection\label{\detokenize{masterCodeDoc:aemter.models.FunktionRecht}}\pysiglinewithargsret{\sphinxbfcode{\sphinxupquote{class }}\sphinxcode{\sphinxupquote{aemter.models.}}\sphinxbfcode{\sphinxupquote{FunktionRecht}}}{\emph{\DUrole{o}{*}\DUrole{n}{args}}, \emph{\DUrole{o}{**}\DUrole{n}{kwargs}}}{}
Datenbankmodell FunktionRecht

Felder:
\begin{itemize}
\item {} 
funktion

\item {} 
recht

\item {} 
history

\end{itemize}
\index{FunktionRecht.DoesNotExist@\spxentry{FunktionRecht.DoesNotExist}}

\begin{fulllineitems}
\phantomsection\label{\detokenize{masterCodeDoc:aemter.models.FunktionRecht.DoesNotExist}}\pysigline{\sphinxbfcode{\sphinxupquote{exception }}\sphinxbfcode{\sphinxupquote{DoesNotExist}}}
\end{fulllineitems}

\index{FunktionRecht.MultipleObjectsReturned@\spxentry{FunktionRecht.MultipleObjectsReturned}}

\begin{fulllineitems}
\phantomsection\label{\detokenize{masterCodeDoc:aemter.models.FunktionRecht.MultipleObjectsReturned}}\pysigline{\sphinxbfcode{\sphinxupquote{exception }}\sphinxbfcode{\sphinxupquote{MultipleObjectsReturned}}}
\end{fulllineitems}

\index{funktion (aemter.models.FunktionRecht attribute)@\spxentry{funktion}\spxextra{aemter.models.FunktionRecht attribute}}

\begin{fulllineitems}
\phantomsection\label{\detokenize{masterCodeDoc:aemter.models.FunktionRecht.funktion}}\pysigline{\sphinxbfcode{\sphinxupquote{funktion}}}
Accessor to the related object on the forward side of a many\sphinxhyphen{}to\sphinxhyphen{}one or
one\sphinxhyphen{}to\sphinxhyphen{}one (via ForwardOneToOneDescriptor subclass) relation.

In the example:

\begin{sphinxVerbatim}[commandchars=\\\{\}]
\PYG{k}{class} \PYG{n+nc}{Child}\PYG{p}{(}\PYG{n}{Model}\PYG{p}{)}\PYG{p}{:}
    \PYG{n}{parent} \PYG{o}{=} \PYG{n}{ForeignKey}\PYG{p}{(}\PYG{n}{Parent}\PYG{p}{,} \PYG{n}{related\PYGZus{}name}\PYG{o}{=}\PYG{l+s+s1}{\PYGZsq{}}\PYG{l+s+s1}{children}\PYG{l+s+s1}{\PYGZsq{}}\PYG{p}{)}
\end{sphinxVerbatim}

\sphinxcode{\sphinxupquote{Child.parent}} is a \sphinxcode{\sphinxupquote{ForwardManyToOneDescriptor}} instance.

\end{fulllineitems}

\index{funktion\_id (aemter.models.FunktionRecht attribute)@\spxentry{funktion\_id}\spxextra{aemter.models.FunktionRecht attribute}}

\begin{fulllineitems}
\phantomsection\label{\detokenize{masterCodeDoc:aemter.models.FunktionRecht.funktion_id}}\pysigline{\sphinxbfcode{\sphinxupquote{funktion\_id}}}
\end{fulllineitems}

\index{history (aemter.models.FunktionRecht attribute)@\spxentry{history}\spxextra{aemter.models.FunktionRecht attribute}}

\begin{fulllineitems}
\phantomsection\label{\detokenize{masterCodeDoc:aemter.models.FunktionRecht.history}}\pysigline{\sphinxbfcode{\sphinxupquote{history}}\sphinxbfcode{\sphinxupquote{ = \textless{}simple\_history.manager.HistoryManager object\textgreater{}}}}
\end{fulllineitems}

\index{id (aemter.models.FunktionRecht attribute)@\spxentry{id}\spxextra{aemter.models.FunktionRecht attribute}}

\begin{fulllineitems}
\phantomsection\label{\detokenize{masterCodeDoc:aemter.models.FunktionRecht.id}}\pysigline{\sphinxbfcode{\sphinxupquote{id}}}
A wrapper for a deferred\sphinxhyphen{}loading field. When the value is read from this
object the first time, the query is executed.

\end{fulllineitems}

\index{objects (aemter.models.FunktionRecht attribute)@\spxentry{objects}\spxextra{aemter.models.FunktionRecht attribute}}

\begin{fulllineitems}
\phantomsection\label{\detokenize{masterCodeDoc:aemter.models.FunktionRecht.objects}}\pysigline{\sphinxbfcode{\sphinxupquote{objects}}\sphinxbfcode{\sphinxupquote{ = \textless{}django.db.models.manager.Manager object\textgreater{}}}}
\end{fulllineitems}

\index{recht (aemter.models.FunktionRecht attribute)@\spxentry{recht}\spxextra{aemter.models.FunktionRecht attribute}}

\begin{fulllineitems}
\phantomsection\label{\detokenize{masterCodeDoc:aemter.models.FunktionRecht.recht}}\pysigline{\sphinxbfcode{\sphinxupquote{recht}}}
Accessor to the related object on the forward side of a many\sphinxhyphen{}to\sphinxhyphen{}one or
one\sphinxhyphen{}to\sphinxhyphen{}one (via ForwardOneToOneDescriptor subclass) relation.

In the example:

\begin{sphinxVerbatim}[commandchars=\\\{\}]
\PYG{k}{class} \PYG{n+nc}{Child}\PYG{p}{(}\PYG{n}{Model}\PYG{p}{)}\PYG{p}{:}
    \PYG{n}{parent} \PYG{o}{=} \PYG{n}{ForeignKey}\PYG{p}{(}\PYG{n}{Parent}\PYG{p}{,} \PYG{n}{related\PYGZus{}name}\PYG{o}{=}\PYG{l+s+s1}{\PYGZsq{}}\PYG{l+s+s1}{children}\PYG{l+s+s1}{\PYGZsq{}}\PYG{p}{)}
\end{sphinxVerbatim}

\sphinxcode{\sphinxupquote{Child.parent}} is a \sphinxcode{\sphinxupquote{ForwardManyToOneDescriptor}} instance.

\end{fulllineitems}

\index{recht\_id (aemter.models.FunktionRecht attribute)@\spxentry{recht\_id}\spxextra{aemter.models.FunktionRecht attribute}}

\begin{fulllineitems}
\phantomsection\label{\detokenize{masterCodeDoc:aemter.models.FunktionRecht.recht_id}}\pysigline{\sphinxbfcode{\sphinxupquote{recht\_id}}}
\end{fulllineitems}

\index{save\_without\_historical\_record() (aemter.models.FunktionRecht method)@\spxentry{save\_without\_historical\_record()}\spxextra{aemter.models.FunktionRecht method}}

\begin{fulllineitems}
\phantomsection\label{\detokenize{masterCodeDoc:aemter.models.FunktionRecht.save_without_historical_record}}\pysiglinewithargsret{\sphinxbfcode{\sphinxupquote{save\_without\_historical\_record}}}{\emph{\DUrole{o}{*}\DUrole{n}{args}}, \emph{\DUrole{o}{**}\DUrole{n}{kwargs}}}{}
Save model without saving a historical record

Make sure you know what you’re doing before you use this method.

\end{fulllineitems}


\end{fulllineitems}

\index{HistoricalFunktion (class in aemter.models)@\spxentry{HistoricalFunktion}\spxextra{class in aemter.models}}

\begin{fulllineitems}
\phantomsection\label{\detokenize{masterCodeDoc:aemter.models.HistoricalFunktion}}\pysiglinewithargsret{\sphinxbfcode{\sphinxupquote{class }}\sphinxcode{\sphinxupquote{aemter.models.}}\sphinxbfcode{\sphinxupquote{HistoricalFunktion}}}{\emph{\DUrole{n}{id}}, \emph{\DUrole{n}{bezeichnung}}, \emph{\DUrole{n}{workload}}, \emph{\DUrole{n}{max\_members}}, \emph{\DUrole{n}{organisationseinheit}}, \emph{\DUrole{n}{unterbereich}}, \emph{\DUrole{n}{history\_id}}, \emph{\DUrole{n}{history\_date}}, \emph{\DUrole{n}{history\_change\_reason}}, \emph{\DUrole{n}{history\_type}}, \emph{\DUrole{n}{history\_user}}}{}~\index{HistoricalFunktion.DoesNotExist@\spxentry{HistoricalFunktion.DoesNotExist}}

\begin{fulllineitems}
\phantomsection\label{\detokenize{masterCodeDoc:aemter.models.HistoricalFunktion.DoesNotExist}}\pysigline{\sphinxbfcode{\sphinxupquote{exception }}\sphinxbfcode{\sphinxupquote{DoesNotExist}}}
\end{fulllineitems}

\index{HistoricalFunktion.MultipleObjectsReturned@\spxentry{HistoricalFunktion.MultipleObjectsReturned}}

\begin{fulllineitems}
\phantomsection\label{\detokenize{masterCodeDoc:aemter.models.HistoricalFunktion.MultipleObjectsReturned}}\pysigline{\sphinxbfcode{\sphinxupquote{exception }}\sphinxbfcode{\sphinxupquote{MultipleObjectsReturned}}}
\end{fulllineitems}

\index{bezeichnung (aemter.models.HistoricalFunktion attribute)@\spxentry{bezeichnung}\spxextra{aemter.models.HistoricalFunktion attribute}}

\begin{fulllineitems}
\phantomsection\label{\detokenize{masterCodeDoc:aemter.models.HistoricalFunktion.bezeichnung}}\pysigline{\sphinxbfcode{\sphinxupquote{bezeichnung}}}
A wrapper for a deferred\sphinxhyphen{}loading field. When the value is read from this
object the first time, the query is executed.

\end{fulllineitems}

\index{get\_default\_history\_user() (aemter.models.HistoricalFunktion static method)@\spxentry{get\_default\_history\_user()}\spxextra{aemter.models.HistoricalFunktion static method}}

\begin{fulllineitems}
\phantomsection\label{\detokenize{masterCodeDoc:aemter.models.HistoricalFunktion.get_default_history_user}}\pysiglinewithargsret{\sphinxbfcode{\sphinxupquote{static }}\sphinxbfcode{\sphinxupquote{get\_default\_history\_user}}}{\emph{\DUrole{n}{instance}}}{}
Returns the user specified by \sphinxtitleref{get\_user} method for manually creating
historical objects

\end{fulllineitems}

\index{get\_history\_type\_display() (aemter.models.HistoricalFunktion method)@\spxentry{get\_history\_type\_display()}\spxextra{aemter.models.HistoricalFunktion method}}

\begin{fulllineitems}
\phantomsection\label{\detokenize{masterCodeDoc:aemter.models.HistoricalFunktion.get_history_type_display}}\pysiglinewithargsret{\sphinxbfcode{\sphinxupquote{get\_history\_type\_display}}}{\emph{*}, \emph{field=\textless{}django.db.models.fields.CharField: history\_type\textgreater{}}}{}
\end{fulllineitems}

\index{get\_next\_by\_history\_date() (aemter.models.HistoricalFunktion method)@\spxentry{get\_next\_by\_history\_date()}\spxextra{aemter.models.HistoricalFunktion method}}

\begin{fulllineitems}
\phantomsection\label{\detokenize{masterCodeDoc:aemter.models.HistoricalFunktion.get_next_by_history_date}}\pysiglinewithargsret{\sphinxbfcode{\sphinxupquote{get\_next\_by\_history\_date}}}{\emph{*}, \emph{field=\textless{}django.db.models.fields.DateTimeField: history\_date\textgreater{}}, \emph{is\_next=True}, \emph{**kwargs}}{}
\end{fulllineitems}

\index{get\_previous\_by\_history\_date() (aemter.models.HistoricalFunktion method)@\spxentry{get\_previous\_by\_history\_date()}\spxextra{aemter.models.HistoricalFunktion method}}

\begin{fulllineitems}
\phantomsection\label{\detokenize{masterCodeDoc:aemter.models.HistoricalFunktion.get_previous_by_history_date}}\pysiglinewithargsret{\sphinxbfcode{\sphinxupquote{get\_previous\_by\_history\_date}}}{\emph{*}, \emph{field=\textless{}django.db.models.fields.DateTimeField: history\_date\textgreater{}}, \emph{is\_next=False}, \emph{**kwargs}}{}
\end{fulllineitems}

\index{history\_change\_reason (aemter.models.HistoricalFunktion attribute)@\spxentry{history\_change\_reason}\spxextra{aemter.models.HistoricalFunktion attribute}}

\begin{fulllineitems}
\phantomsection\label{\detokenize{masterCodeDoc:aemter.models.HistoricalFunktion.history_change_reason}}\pysigline{\sphinxbfcode{\sphinxupquote{history\_change\_reason}}}
A wrapper for a deferred\sphinxhyphen{}loading field. When the value is read from this
object the first time, the query is executed.

\end{fulllineitems}

\index{history\_date (aemter.models.HistoricalFunktion attribute)@\spxentry{history\_date}\spxextra{aemter.models.HistoricalFunktion attribute}}

\begin{fulllineitems}
\phantomsection\label{\detokenize{masterCodeDoc:aemter.models.HistoricalFunktion.history_date}}\pysigline{\sphinxbfcode{\sphinxupquote{history\_date}}}
A wrapper for a deferred\sphinxhyphen{}loading field. When the value is read from this
object the first time, the query is executed.

\end{fulllineitems}

\index{history\_id (aemter.models.HistoricalFunktion attribute)@\spxentry{history\_id}\spxextra{aemter.models.HistoricalFunktion attribute}}

\begin{fulllineitems}
\phantomsection\label{\detokenize{masterCodeDoc:aemter.models.HistoricalFunktion.history_id}}\pysigline{\sphinxbfcode{\sphinxupquote{history\_id}}}
A wrapper for a deferred\sphinxhyphen{}loading field. When the value is read from this
object the first time, the query is executed.

\end{fulllineitems}

\index{history\_object (aemter.models.HistoricalFunktion attribute)@\spxentry{history\_object}\spxextra{aemter.models.HistoricalFunktion attribute}}

\begin{fulllineitems}
\phantomsection\label{\detokenize{masterCodeDoc:aemter.models.HistoricalFunktion.history_object}}\pysigline{\sphinxbfcode{\sphinxupquote{history\_object}}}
\end{fulllineitems}

\index{history\_type (aemter.models.HistoricalFunktion attribute)@\spxentry{history\_type}\spxextra{aemter.models.HistoricalFunktion attribute}}

\begin{fulllineitems}
\phantomsection\label{\detokenize{masterCodeDoc:aemter.models.HistoricalFunktion.history_type}}\pysigline{\sphinxbfcode{\sphinxupquote{history\_type}}}
A wrapper for a deferred\sphinxhyphen{}loading field. When the value is read from this
object the first time, the query is executed.

\end{fulllineitems}

\index{history\_user (aemter.models.HistoricalFunktion attribute)@\spxentry{history\_user}\spxextra{aemter.models.HistoricalFunktion attribute}}

\begin{fulllineitems}
\phantomsection\label{\detokenize{masterCodeDoc:aemter.models.HistoricalFunktion.history_user}}\pysigline{\sphinxbfcode{\sphinxupquote{history\_user}}}
Accessor to the related object on the forward side of a many\sphinxhyphen{}to\sphinxhyphen{}one or
one\sphinxhyphen{}to\sphinxhyphen{}one (via ForwardOneToOneDescriptor subclass) relation.

In the example:

\begin{sphinxVerbatim}[commandchars=\\\{\}]
\PYG{k}{class} \PYG{n+nc}{Child}\PYG{p}{(}\PYG{n}{Model}\PYG{p}{)}\PYG{p}{:}
    \PYG{n}{parent} \PYG{o}{=} \PYG{n}{ForeignKey}\PYG{p}{(}\PYG{n}{Parent}\PYG{p}{,} \PYG{n}{related\PYGZus{}name}\PYG{o}{=}\PYG{l+s+s1}{\PYGZsq{}}\PYG{l+s+s1}{children}\PYG{l+s+s1}{\PYGZsq{}}\PYG{p}{)}
\end{sphinxVerbatim}

\sphinxcode{\sphinxupquote{Child.parent}} is a \sphinxcode{\sphinxupquote{ForwardManyToOneDescriptor}} instance.

\end{fulllineitems}

\index{history\_user\_id (aemter.models.HistoricalFunktion attribute)@\spxentry{history\_user\_id}\spxextra{aemter.models.HistoricalFunktion attribute}}

\begin{fulllineitems}
\phantomsection\label{\detokenize{masterCodeDoc:aemter.models.HistoricalFunktion.history_user_id}}\pysigline{\sphinxbfcode{\sphinxupquote{history\_user\_id}}}
\end{fulllineitems}

\index{id (aemter.models.HistoricalFunktion attribute)@\spxentry{id}\spxextra{aemter.models.HistoricalFunktion attribute}}

\begin{fulllineitems}
\phantomsection\label{\detokenize{masterCodeDoc:aemter.models.HistoricalFunktion.id}}\pysigline{\sphinxbfcode{\sphinxupquote{id}}}
A wrapper for a deferred\sphinxhyphen{}loading field. When the value is read from this
object the first time, the query is executed.

\end{fulllineitems}

\index{instance() (aemter.models.HistoricalFunktion property)@\spxentry{instance()}\spxextra{aemter.models.HistoricalFunktion property}}

\begin{fulllineitems}
\phantomsection\label{\detokenize{masterCodeDoc:aemter.models.HistoricalFunktion.instance}}\pysigline{\sphinxbfcode{\sphinxupquote{property }}\sphinxbfcode{\sphinxupquote{instance}}}
\end{fulllineitems}

\index{instance\_type (aemter.models.HistoricalFunktion attribute)@\spxentry{instance\_type}\spxextra{aemter.models.HistoricalFunktion attribute}}

\begin{fulllineitems}
\phantomsection\label{\detokenize{masterCodeDoc:aemter.models.HistoricalFunktion.instance_type}}\pysigline{\sphinxbfcode{\sphinxupquote{instance\_type}}}
alias of {\hyperref[\detokenize{masterCodeDoc:aemter.models.Funktion}]{\sphinxcrossref{\sphinxcode{\sphinxupquote{Funktion}}}}}

\end{fulllineitems}

\index{max\_members (aemter.models.HistoricalFunktion attribute)@\spxentry{max\_members}\spxextra{aemter.models.HistoricalFunktion attribute}}

\begin{fulllineitems}
\phantomsection\label{\detokenize{masterCodeDoc:aemter.models.HistoricalFunktion.max_members}}\pysigline{\sphinxbfcode{\sphinxupquote{max\_members}}}
A wrapper for a deferred\sphinxhyphen{}loading field. When the value is read from this
object the first time, the query is executed.

\end{fulllineitems}

\index{next\_record() (aemter.models.HistoricalFunktion property)@\spxentry{next\_record()}\spxextra{aemter.models.HistoricalFunktion property}}

\begin{fulllineitems}
\phantomsection\label{\detokenize{masterCodeDoc:aemter.models.HistoricalFunktion.next_record}}\pysigline{\sphinxbfcode{\sphinxupquote{property }}\sphinxbfcode{\sphinxupquote{next\_record}}}
Get the next history record for the instance. \sphinxtitleref{None} if last.

\end{fulllineitems}

\index{objects (aemter.models.HistoricalFunktion attribute)@\spxentry{objects}\spxextra{aemter.models.HistoricalFunktion attribute}}

\begin{fulllineitems}
\phantomsection\label{\detokenize{masterCodeDoc:aemter.models.HistoricalFunktion.objects}}\pysigline{\sphinxbfcode{\sphinxupquote{objects}}\sphinxbfcode{\sphinxupquote{ = \textless{}django.db.models.manager.Manager object\textgreater{}}}}
\end{fulllineitems}

\index{organisationseinheit (aemter.models.HistoricalFunktion attribute)@\spxentry{organisationseinheit}\spxextra{aemter.models.HistoricalFunktion attribute}}

\begin{fulllineitems}
\phantomsection\label{\detokenize{masterCodeDoc:aemter.models.HistoricalFunktion.organisationseinheit}}\pysigline{\sphinxbfcode{\sphinxupquote{organisationseinheit}}}
Accessor to the related object on the forward side of a many\sphinxhyphen{}to\sphinxhyphen{}one or
one\sphinxhyphen{}to\sphinxhyphen{}one (via ForwardOneToOneDescriptor subclass) relation.

In the example:

\begin{sphinxVerbatim}[commandchars=\\\{\}]
\PYG{k}{class} \PYG{n+nc}{Child}\PYG{p}{(}\PYG{n}{Model}\PYG{p}{)}\PYG{p}{:}
    \PYG{n}{parent} \PYG{o}{=} \PYG{n}{ForeignKey}\PYG{p}{(}\PYG{n}{Parent}\PYG{p}{,} \PYG{n}{related\PYGZus{}name}\PYG{o}{=}\PYG{l+s+s1}{\PYGZsq{}}\PYG{l+s+s1}{children}\PYG{l+s+s1}{\PYGZsq{}}\PYG{p}{)}
\end{sphinxVerbatim}

\sphinxcode{\sphinxupquote{Child.parent}} is a \sphinxcode{\sphinxupquote{ForwardManyToOneDescriptor}} instance.

\end{fulllineitems}

\index{organisationseinheit\_id (aemter.models.HistoricalFunktion attribute)@\spxentry{organisationseinheit\_id}\spxextra{aemter.models.HistoricalFunktion attribute}}

\begin{fulllineitems}
\phantomsection\label{\detokenize{masterCodeDoc:aemter.models.HistoricalFunktion.organisationseinheit_id}}\pysigline{\sphinxbfcode{\sphinxupquote{organisationseinheit\_id}}}
\end{fulllineitems}

\index{prev\_record() (aemter.models.HistoricalFunktion property)@\spxentry{prev\_record()}\spxextra{aemter.models.HistoricalFunktion property}}

\begin{fulllineitems}
\phantomsection\label{\detokenize{masterCodeDoc:aemter.models.HistoricalFunktion.prev_record}}\pysigline{\sphinxbfcode{\sphinxupquote{property }}\sphinxbfcode{\sphinxupquote{prev\_record}}}
Get the previous history record for the instance. \sphinxtitleref{None} if first.

\end{fulllineitems}

\index{revert\_url() (aemter.models.HistoricalFunktion method)@\spxentry{revert\_url()}\spxextra{aemter.models.HistoricalFunktion method}}

\begin{fulllineitems}
\phantomsection\label{\detokenize{masterCodeDoc:aemter.models.HistoricalFunktion.revert_url}}\pysiglinewithargsret{\sphinxbfcode{\sphinxupquote{revert\_url}}}{}{}
URL for this change in the default admin site.

\end{fulllineitems}

\index{unterbereich (aemter.models.HistoricalFunktion attribute)@\spxentry{unterbereich}\spxextra{aemter.models.HistoricalFunktion attribute}}

\begin{fulllineitems}
\phantomsection\label{\detokenize{masterCodeDoc:aemter.models.HistoricalFunktion.unterbereich}}\pysigline{\sphinxbfcode{\sphinxupquote{unterbereich}}}
Accessor to the related object on the forward side of a many\sphinxhyphen{}to\sphinxhyphen{}one or
one\sphinxhyphen{}to\sphinxhyphen{}one (via ForwardOneToOneDescriptor subclass) relation.

In the example:

\begin{sphinxVerbatim}[commandchars=\\\{\}]
\PYG{k}{class} \PYG{n+nc}{Child}\PYG{p}{(}\PYG{n}{Model}\PYG{p}{)}\PYG{p}{:}
    \PYG{n}{parent} \PYG{o}{=} \PYG{n}{ForeignKey}\PYG{p}{(}\PYG{n}{Parent}\PYG{p}{,} \PYG{n}{related\PYGZus{}name}\PYG{o}{=}\PYG{l+s+s1}{\PYGZsq{}}\PYG{l+s+s1}{children}\PYG{l+s+s1}{\PYGZsq{}}\PYG{p}{)}
\end{sphinxVerbatim}

\sphinxcode{\sphinxupquote{Child.parent}} is a \sphinxcode{\sphinxupquote{ForwardManyToOneDescriptor}} instance.

\end{fulllineitems}

\index{unterbereich\_id (aemter.models.HistoricalFunktion attribute)@\spxentry{unterbereich\_id}\spxextra{aemter.models.HistoricalFunktion attribute}}

\begin{fulllineitems}
\phantomsection\label{\detokenize{masterCodeDoc:aemter.models.HistoricalFunktion.unterbereich_id}}\pysigline{\sphinxbfcode{\sphinxupquote{unterbereich\_id}}}
\end{fulllineitems}

\index{workload (aemter.models.HistoricalFunktion attribute)@\spxentry{workload}\spxextra{aemter.models.HistoricalFunktion attribute}}

\begin{fulllineitems}
\phantomsection\label{\detokenize{masterCodeDoc:aemter.models.HistoricalFunktion.workload}}\pysigline{\sphinxbfcode{\sphinxupquote{workload}}}
A wrapper for a deferred\sphinxhyphen{}loading field. When the value is read from this
object the first time, the query is executed.

\end{fulllineitems}


\end{fulllineitems}

\index{HistoricalFunktionRecht (class in aemter.models)@\spxentry{HistoricalFunktionRecht}\spxextra{class in aemter.models}}

\begin{fulllineitems}
\phantomsection\label{\detokenize{masterCodeDoc:aemter.models.HistoricalFunktionRecht}}\pysiglinewithargsret{\sphinxbfcode{\sphinxupquote{class }}\sphinxcode{\sphinxupquote{aemter.models.}}\sphinxbfcode{\sphinxupquote{HistoricalFunktionRecht}}}{\emph{\DUrole{n}{id}}, \emph{\DUrole{n}{funktion}}, \emph{\DUrole{n}{recht}}, \emph{\DUrole{n}{history\_id}}, \emph{\DUrole{n}{history\_date}}, \emph{\DUrole{n}{history\_change\_reason}}, \emph{\DUrole{n}{history\_type}}, \emph{\DUrole{n}{history\_user}}}{}~\index{HistoricalFunktionRecht.DoesNotExist@\spxentry{HistoricalFunktionRecht.DoesNotExist}}

\begin{fulllineitems}
\phantomsection\label{\detokenize{masterCodeDoc:aemter.models.HistoricalFunktionRecht.DoesNotExist}}\pysigline{\sphinxbfcode{\sphinxupquote{exception }}\sphinxbfcode{\sphinxupquote{DoesNotExist}}}
\end{fulllineitems}

\index{HistoricalFunktionRecht.MultipleObjectsReturned@\spxentry{HistoricalFunktionRecht.MultipleObjectsReturned}}

\begin{fulllineitems}
\phantomsection\label{\detokenize{masterCodeDoc:aemter.models.HistoricalFunktionRecht.MultipleObjectsReturned}}\pysigline{\sphinxbfcode{\sphinxupquote{exception }}\sphinxbfcode{\sphinxupquote{MultipleObjectsReturned}}}
\end{fulllineitems}

\index{funktion (aemter.models.HistoricalFunktionRecht attribute)@\spxentry{funktion}\spxextra{aemter.models.HistoricalFunktionRecht attribute}}

\begin{fulllineitems}
\phantomsection\label{\detokenize{masterCodeDoc:aemter.models.HistoricalFunktionRecht.funktion}}\pysigline{\sphinxbfcode{\sphinxupquote{funktion}}}
Accessor to the related object on the forward side of a many\sphinxhyphen{}to\sphinxhyphen{}one or
one\sphinxhyphen{}to\sphinxhyphen{}one (via ForwardOneToOneDescriptor subclass) relation.

In the example:

\begin{sphinxVerbatim}[commandchars=\\\{\}]
\PYG{k}{class} \PYG{n+nc}{Child}\PYG{p}{(}\PYG{n}{Model}\PYG{p}{)}\PYG{p}{:}
    \PYG{n}{parent} \PYG{o}{=} \PYG{n}{ForeignKey}\PYG{p}{(}\PYG{n}{Parent}\PYG{p}{,} \PYG{n}{related\PYGZus{}name}\PYG{o}{=}\PYG{l+s+s1}{\PYGZsq{}}\PYG{l+s+s1}{children}\PYG{l+s+s1}{\PYGZsq{}}\PYG{p}{)}
\end{sphinxVerbatim}

\sphinxcode{\sphinxupquote{Child.parent}} is a \sphinxcode{\sphinxupquote{ForwardManyToOneDescriptor}} instance.

\end{fulllineitems}

\index{funktion\_id (aemter.models.HistoricalFunktionRecht attribute)@\spxentry{funktion\_id}\spxextra{aemter.models.HistoricalFunktionRecht attribute}}

\begin{fulllineitems}
\phantomsection\label{\detokenize{masterCodeDoc:aemter.models.HistoricalFunktionRecht.funktion_id}}\pysigline{\sphinxbfcode{\sphinxupquote{funktion\_id}}}
\end{fulllineitems}

\index{get\_default\_history\_user() (aemter.models.HistoricalFunktionRecht static method)@\spxentry{get\_default\_history\_user()}\spxextra{aemter.models.HistoricalFunktionRecht static method}}

\begin{fulllineitems}
\phantomsection\label{\detokenize{masterCodeDoc:aemter.models.HistoricalFunktionRecht.get_default_history_user}}\pysiglinewithargsret{\sphinxbfcode{\sphinxupquote{static }}\sphinxbfcode{\sphinxupquote{get\_default\_history\_user}}}{\emph{\DUrole{n}{instance}}}{}
Returns the user specified by \sphinxtitleref{get\_user} method for manually creating
historical objects

\end{fulllineitems}

\index{get\_history\_type\_display() (aemter.models.HistoricalFunktionRecht method)@\spxentry{get\_history\_type\_display()}\spxextra{aemter.models.HistoricalFunktionRecht method}}

\begin{fulllineitems}
\phantomsection\label{\detokenize{masterCodeDoc:aemter.models.HistoricalFunktionRecht.get_history_type_display}}\pysiglinewithargsret{\sphinxbfcode{\sphinxupquote{get\_history\_type\_display}}}{\emph{*}, \emph{field=\textless{}django.db.models.fields.CharField: history\_type\textgreater{}}}{}
\end{fulllineitems}

\index{get\_next\_by\_history\_date() (aemter.models.HistoricalFunktionRecht method)@\spxentry{get\_next\_by\_history\_date()}\spxextra{aemter.models.HistoricalFunktionRecht method}}

\begin{fulllineitems}
\phantomsection\label{\detokenize{masterCodeDoc:aemter.models.HistoricalFunktionRecht.get_next_by_history_date}}\pysiglinewithargsret{\sphinxbfcode{\sphinxupquote{get\_next\_by\_history\_date}}}{\emph{*}, \emph{field=\textless{}django.db.models.fields.DateTimeField: history\_date\textgreater{}}, \emph{is\_next=True}, \emph{**kwargs}}{}
\end{fulllineitems}

\index{get\_previous\_by\_history\_date() (aemter.models.HistoricalFunktionRecht method)@\spxentry{get\_previous\_by\_history\_date()}\spxextra{aemter.models.HistoricalFunktionRecht method}}

\begin{fulllineitems}
\phantomsection\label{\detokenize{masterCodeDoc:aemter.models.HistoricalFunktionRecht.get_previous_by_history_date}}\pysiglinewithargsret{\sphinxbfcode{\sphinxupquote{get\_previous\_by\_history\_date}}}{\emph{*}, \emph{field=\textless{}django.db.models.fields.DateTimeField: history\_date\textgreater{}}, \emph{is\_next=False}, \emph{**kwargs}}{}
\end{fulllineitems}

\index{history\_change\_reason (aemter.models.HistoricalFunktionRecht attribute)@\spxentry{history\_change\_reason}\spxextra{aemter.models.HistoricalFunktionRecht attribute}}

\begin{fulllineitems}
\phantomsection\label{\detokenize{masterCodeDoc:aemter.models.HistoricalFunktionRecht.history_change_reason}}\pysigline{\sphinxbfcode{\sphinxupquote{history\_change\_reason}}}
A wrapper for a deferred\sphinxhyphen{}loading field. When the value is read from this
object the first time, the query is executed.

\end{fulllineitems}

\index{history\_date (aemter.models.HistoricalFunktionRecht attribute)@\spxentry{history\_date}\spxextra{aemter.models.HistoricalFunktionRecht attribute}}

\begin{fulllineitems}
\phantomsection\label{\detokenize{masterCodeDoc:aemter.models.HistoricalFunktionRecht.history_date}}\pysigline{\sphinxbfcode{\sphinxupquote{history\_date}}}
A wrapper for a deferred\sphinxhyphen{}loading field. When the value is read from this
object the first time, the query is executed.

\end{fulllineitems}

\index{history\_id (aemter.models.HistoricalFunktionRecht attribute)@\spxentry{history\_id}\spxextra{aemter.models.HistoricalFunktionRecht attribute}}

\begin{fulllineitems}
\phantomsection\label{\detokenize{masterCodeDoc:aemter.models.HistoricalFunktionRecht.history_id}}\pysigline{\sphinxbfcode{\sphinxupquote{history\_id}}}
A wrapper for a deferred\sphinxhyphen{}loading field. When the value is read from this
object the first time, the query is executed.

\end{fulllineitems}

\index{history\_object (aemter.models.HistoricalFunktionRecht attribute)@\spxentry{history\_object}\spxextra{aemter.models.HistoricalFunktionRecht attribute}}

\begin{fulllineitems}
\phantomsection\label{\detokenize{masterCodeDoc:aemter.models.HistoricalFunktionRecht.history_object}}\pysigline{\sphinxbfcode{\sphinxupquote{history\_object}}}
\end{fulllineitems}

\index{history\_type (aemter.models.HistoricalFunktionRecht attribute)@\spxentry{history\_type}\spxextra{aemter.models.HistoricalFunktionRecht attribute}}

\begin{fulllineitems}
\phantomsection\label{\detokenize{masterCodeDoc:aemter.models.HistoricalFunktionRecht.history_type}}\pysigline{\sphinxbfcode{\sphinxupquote{history\_type}}}
A wrapper for a deferred\sphinxhyphen{}loading field. When the value is read from this
object the first time, the query is executed.

\end{fulllineitems}

\index{history\_user (aemter.models.HistoricalFunktionRecht attribute)@\spxentry{history\_user}\spxextra{aemter.models.HistoricalFunktionRecht attribute}}

\begin{fulllineitems}
\phantomsection\label{\detokenize{masterCodeDoc:aemter.models.HistoricalFunktionRecht.history_user}}\pysigline{\sphinxbfcode{\sphinxupquote{history\_user}}}
Accessor to the related object on the forward side of a many\sphinxhyphen{}to\sphinxhyphen{}one or
one\sphinxhyphen{}to\sphinxhyphen{}one (via ForwardOneToOneDescriptor subclass) relation.

In the example:

\begin{sphinxVerbatim}[commandchars=\\\{\}]
\PYG{k}{class} \PYG{n+nc}{Child}\PYG{p}{(}\PYG{n}{Model}\PYG{p}{)}\PYG{p}{:}
    \PYG{n}{parent} \PYG{o}{=} \PYG{n}{ForeignKey}\PYG{p}{(}\PYG{n}{Parent}\PYG{p}{,} \PYG{n}{related\PYGZus{}name}\PYG{o}{=}\PYG{l+s+s1}{\PYGZsq{}}\PYG{l+s+s1}{children}\PYG{l+s+s1}{\PYGZsq{}}\PYG{p}{)}
\end{sphinxVerbatim}

\sphinxcode{\sphinxupquote{Child.parent}} is a \sphinxcode{\sphinxupquote{ForwardManyToOneDescriptor}} instance.

\end{fulllineitems}

\index{history\_user\_id (aemter.models.HistoricalFunktionRecht attribute)@\spxentry{history\_user\_id}\spxextra{aemter.models.HistoricalFunktionRecht attribute}}

\begin{fulllineitems}
\phantomsection\label{\detokenize{masterCodeDoc:aemter.models.HistoricalFunktionRecht.history_user_id}}\pysigline{\sphinxbfcode{\sphinxupquote{history\_user\_id}}}
\end{fulllineitems}

\index{id (aemter.models.HistoricalFunktionRecht attribute)@\spxentry{id}\spxextra{aemter.models.HistoricalFunktionRecht attribute}}

\begin{fulllineitems}
\phantomsection\label{\detokenize{masterCodeDoc:aemter.models.HistoricalFunktionRecht.id}}\pysigline{\sphinxbfcode{\sphinxupquote{id}}}
A wrapper for a deferred\sphinxhyphen{}loading field. When the value is read from this
object the first time, the query is executed.

\end{fulllineitems}

\index{instance() (aemter.models.HistoricalFunktionRecht property)@\spxentry{instance()}\spxextra{aemter.models.HistoricalFunktionRecht property}}

\begin{fulllineitems}
\phantomsection\label{\detokenize{masterCodeDoc:aemter.models.HistoricalFunktionRecht.instance}}\pysigline{\sphinxbfcode{\sphinxupquote{property }}\sphinxbfcode{\sphinxupquote{instance}}}
\end{fulllineitems}

\index{instance\_type (aemter.models.HistoricalFunktionRecht attribute)@\spxentry{instance\_type}\spxextra{aemter.models.HistoricalFunktionRecht attribute}}

\begin{fulllineitems}
\phantomsection\label{\detokenize{masterCodeDoc:aemter.models.HistoricalFunktionRecht.instance_type}}\pysigline{\sphinxbfcode{\sphinxupquote{instance\_type}}}
alias of {\hyperref[\detokenize{masterCodeDoc:aemter.models.FunktionRecht}]{\sphinxcrossref{\sphinxcode{\sphinxupquote{FunktionRecht}}}}}

\end{fulllineitems}

\index{next\_record() (aemter.models.HistoricalFunktionRecht property)@\spxentry{next\_record()}\spxextra{aemter.models.HistoricalFunktionRecht property}}

\begin{fulllineitems}
\phantomsection\label{\detokenize{masterCodeDoc:aemter.models.HistoricalFunktionRecht.next_record}}\pysigline{\sphinxbfcode{\sphinxupquote{property }}\sphinxbfcode{\sphinxupquote{next\_record}}}
Get the next history record for the instance. \sphinxtitleref{None} if last.

\end{fulllineitems}

\index{objects (aemter.models.HistoricalFunktionRecht attribute)@\spxentry{objects}\spxextra{aemter.models.HistoricalFunktionRecht attribute}}

\begin{fulllineitems}
\phantomsection\label{\detokenize{masterCodeDoc:aemter.models.HistoricalFunktionRecht.objects}}\pysigline{\sphinxbfcode{\sphinxupquote{objects}}\sphinxbfcode{\sphinxupquote{ = \textless{}django.db.models.manager.Manager object\textgreater{}}}}
\end{fulllineitems}

\index{prev\_record() (aemter.models.HistoricalFunktionRecht property)@\spxentry{prev\_record()}\spxextra{aemter.models.HistoricalFunktionRecht property}}

\begin{fulllineitems}
\phantomsection\label{\detokenize{masterCodeDoc:aemter.models.HistoricalFunktionRecht.prev_record}}\pysigline{\sphinxbfcode{\sphinxupquote{property }}\sphinxbfcode{\sphinxupquote{prev\_record}}}
Get the previous history record for the instance. \sphinxtitleref{None} if first.

\end{fulllineitems}

\index{recht (aemter.models.HistoricalFunktionRecht attribute)@\spxentry{recht}\spxextra{aemter.models.HistoricalFunktionRecht attribute}}

\begin{fulllineitems}
\phantomsection\label{\detokenize{masterCodeDoc:aemter.models.HistoricalFunktionRecht.recht}}\pysigline{\sphinxbfcode{\sphinxupquote{recht}}}
Accessor to the related object on the forward side of a many\sphinxhyphen{}to\sphinxhyphen{}one or
one\sphinxhyphen{}to\sphinxhyphen{}one (via ForwardOneToOneDescriptor subclass) relation.

In the example:

\begin{sphinxVerbatim}[commandchars=\\\{\}]
\PYG{k}{class} \PYG{n+nc}{Child}\PYG{p}{(}\PYG{n}{Model}\PYG{p}{)}\PYG{p}{:}
    \PYG{n}{parent} \PYG{o}{=} \PYG{n}{ForeignKey}\PYG{p}{(}\PYG{n}{Parent}\PYG{p}{,} \PYG{n}{related\PYGZus{}name}\PYG{o}{=}\PYG{l+s+s1}{\PYGZsq{}}\PYG{l+s+s1}{children}\PYG{l+s+s1}{\PYGZsq{}}\PYG{p}{)}
\end{sphinxVerbatim}

\sphinxcode{\sphinxupquote{Child.parent}} is a \sphinxcode{\sphinxupquote{ForwardManyToOneDescriptor}} instance.

\end{fulllineitems}

\index{recht\_id (aemter.models.HistoricalFunktionRecht attribute)@\spxentry{recht\_id}\spxextra{aemter.models.HistoricalFunktionRecht attribute}}

\begin{fulllineitems}
\phantomsection\label{\detokenize{masterCodeDoc:aemter.models.HistoricalFunktionRecht.recht_id}}\pysigline{\sphinxbfcode{\sphinxupquote{recht\_id}}}
\end{fulllineitems}

\index{revert\_url() (aemter.models.HistoricalFunktionRecht method)@\spxentry{revert\_url()}\spxextra{aemter.models.HistoricalFunktionRecht method}}

\begin{fulllineitems}
\phantomsection\label{\detokenize{masterCodeDoc:aemter.models.HistoricalFunktionRecht.revert_url}}\pysiglinewithargsret{\sphinxbfcode{\sphinxupquote{revert\_url}}}{}{}
URL for this change in the default admin site.

\end{fulllineitems}


\end{fulllineitems}

\index{HistoricalOrganisationseinheit (class in aemter.models)@\spxentry{HistoricalOrganisationseinheit}\spxextra{class in aemter.models}}

\begin{fulllineitems}
\phantomsection\label{\detokenize{masterCodeDoc:aemter.models.HistoricalOrganisationseinheit}}\pysiglinewithargsret{\sphinxbfcode{\sphinxupquote{class }}\sphinxcode{\sphinxupquote{aemter.models.}}\sphinxbfcode{\sphinxupquote{HistoricalOrganisationseinheit}}}{\emph{\DUrole{n}{id}}, \emph{\DUrole{n}{bezeichnung}}, \emph{\DUrole{n}{funktionen\_ohne\_unterbereich\_count}}, \emph{\DUrole{n}{history\_id}}, \emph{\DUrole{n}{history\_date}}, \emph{\DUrole{n}{history\_change\_reason}}, \emph{\DUrole{n}{history\_type}}, \emph{\DUrole{n}{history\_user}}}{}~\index{HistoricalOrganisationseinheit.DoesNotExist@\spxentry{HistoricalOrganisationseinheit.DoesNotExist}}

\begin{fulllineitems}
\phantomsection\label{\detokenize{masterCodeDoc:aemter.models.HistoricalOrganisationseinheit.DoesNotExist}}\pysigline{\sphinxbfcode{\sphinxupquote{exception }}\sphinxbfcode{\sphinxupquote{DoesNotExist}}}
\end{fulllineitems}

\index{HistoricalOrganisationseinheit.MultipleObjectsReturned@\spxentry{HistoricalOrganisationseinheit.MultipleObjectsReturned}}

\begin{fulllineitems}
\phantomsection\label{\detokenize{masterCodeDoc:aemter.models.HistoricalOrganisationseinheit.MultipleObjectsReturned}}\pysigline{\sphinxbfcode{\sphinxupquote{exception }}\sphinxbfcode{\sphinxupquote{MultipleObjectsReturned}}}
\end{fulllineitems}

\index{bezeichnung (aemter.models.HistoricalOrganisationseinheit attribute)@\spxentry{bezeichnung}\spxextra{aemter.models.HistoricalOrganisationseinheit attribute}}

\begin{fulllineitems}
\phantomsection\label{\detokenize{masterCodeDoc:aemter.models.HistoricalOrganisationseinheit.bezeichnung}}\pysigline{\sphinxbfcode{\sphinxupquote{bezeichnung}}}
A wrapper for a deferred\sphinxhyphen{}loading field. When the value is read from this
object the first time, the query is executed.

\end{fulllineitems}

\index{funktionen\_ohne\_unterbereich\_count (aemter.models.HistoricalOrganisationseinheit attribute)@\spxentry{funktionen\_ohne\_unterbereich\_count}\spxextra{aemter.models.HistoricalOrganisationseinheit attribute}}

\begin{fulllineitems}
\phantomsection\label{\detokenize{masterCodeDoc:aemter.models.HistoricalOrganisationseinheit.funktionen_ohne_unterbereich_count}}\pysigline{\sphinxbfcode{\sphinxupquote{funktionen\_ohne\_unterbereich\_count}}}
A wrapper for a deferred\sphinxhyphen{}loading field. When the value is read from this
object the first time, the query is executed.

\end{fulllineitems}

\index{get\_default\_history\_user() (aemter.models.HistoricalOrganisationseinheit static method)@\spxentry{get\_default\_history\_user()}\spxextra{aemter.models.HistoricalOrganisationseinheit static method}}

\begin{fulllineitems}
\phantomsection\label{\detokenize{masterCodeDoc:aemter.models.HistoricalOrganisationseinheit.get_default_history_user}}\pysiglinewithargsret{\sphinxbfcode{\sphinxupquote{static }}\sphinxbfcode{\sphinxupquote{get\_default\_history\_user}}}{\emph{\DUrole{n}{instance}}}{}
Returns the user specified by \sphinxtitleref{get\_user} method for manually creating
historical objects

\end{fulllineitems}

\index{get\_history\_type\_display() (aemter.models.HistoricalOrganisationseinheit method)@\spxentry{get\_history\_type\_display()}\spxextra{aemter.models.HistoricalOrganisationseinheit method}}

\begin{fulllineitems}
\phantomsection\label{\detokenize{masterCodeDoc:aemter.models.HistoricalOrganisationseinheit.get_history_type_display}}\pysiglinewithargsret{\sphinxbfcode{\sphinxupquote{get\_history\_type\_display}}}{\emph{*}, \emph{field=\textless{}django.db.models.fields.CharField: history\_type\textgreater{}}}{}
\end{fulllineitems}

\index{get\_next\_by\_history\_date() (aemter.models.HistoricalOrganisationseinheit method)@\spxentry{get\_next\_by\_history\_date()}\spxextra{aemter.models.HistoricalOrganisationseinheit method}}

\begin{fulllineitems}
\phantomsection\label{\detokenize{masterCodeDoc:aemter.models.HistoricalOrganisationseinheit.get_next_by_history_date}}\pysiglinewithargsret{\sphinxbfcode{\sphinxupquote{get\_next\_by\_history\_date}}}{\emph{*}, \emph{field=\textless{}django.db.models.fields.DateTimeField: history\_date\textgreater{}}, \emph{is\_next=True}, \emph{**kwargs}}{}
\end{fulllineitems}

\index{get\_previous\_by\_history\_date() (aemter.models.HistoricalOrganisationseinheit method)@\spxentry{get\_previous\_by\_history\_date()}\spxextra{aemter.models.HistoricalOrganisationseinheit method}}

\begin{fulllineitems}
\phantomsection\label{\detokenize{masterCodeDoc:aemter.models.HistoricalOrganisationseinheit.get_previous_by_history_date}}\pysiglinewithargsret{\sphinxbfcode{\sphinxupquote{get\_previous\_by\_history\_date}}}{\emph{*}, \emph{field=\textless{}django.db.models.fields.DateTimeField: history\_date\textgreater{}}, \emph{is\_next=False}, \emph{**kwargs}}{}
\end{fulllineitems}

\index{history\_change\_reason (aemter.models.HistoricalOrganisationseinheit attribute)@\spxentry{history\_change\_reason}\spxextra{aemter.models.HistoricalOrganisationseinheit attribute}}

\begin{fulllineitems}
\phantomsection\label{\detokenize{masterCodeDoc:aemter.models.HistoricalOrganisationseinheit.history_change_reason}}\pysigline{\sphinxbfcode{\sphinxupquote{history\_change\_reason}}}
A wrapper for a deferred\sphinxhyphen{}loading field. When the value is read from this
object the first time, the query is executed.

\end{fulllineitems}

\index{history\_date (aemter.models.HistoricalOrganisationseinheit attribute)@\spxentry{history\_date}\spxextra{aemter.models.HistoricalOrganisationseinheit attribute}}

\begin{fulllineitems}
\phantomsection\label{\detokenize{masterCodeDoc:aemter.models.HistoricalOrganisationseinheit.history_date}}\pysigline{\sphinxbfcode{\sphinxupquote{history\_date}}}
A wrapper for a deferred\sphinxhyphen{}loading field. When the value is read from this
object the first time, the query is executed.

\end{fulllineitems}

\index{history\_id (aemter.models.HistoricalOrganisationseinheit attribute)@\spxentry{history\_id}\spxextra{aemter.models.HistoricalOrganisationseinheit attribute}}

\begin{fulllineitems}
\phantomsection\label{\detokenize{masterCodeDoc:aemter.models.HistoricalOrganisationseinheit.history_id}}\pysigline{\sphinxbfcode{\sphinxupquote{history\_id}}}
A wrapper for a deferred\sphinxhyphen{}loading field. When the value is read from this
object the first time, the query is executed.

\end{fulllineitems}

\index{history\_object (aemter.models.HistoricalOrganisationseinheit attribute)@\spxentry{history\_object}\spxextra{aemter.models.HistoricalOrganisationseinheit attribute}}

\begin{fulllineitems}
\phantomsection\label{\detokenize{masterCodeDoc:aemter.models.HistoricalOrganisationseinheit.history_object}}\pysigline{\sphinxbfcode{\sphinxupquote{history\_object}}}
\end{fulllineitems}

\index{history\_type (aemter.models.HistoricalOrganisationseinheit attribute)@\spxentry{history\_type}\spxextra{aemter.models.HistoricalOrganisationseinheit attribute}}

\begin{fulllineitems}
\phantomsection\label{\detokenize{masterCodeDoc:aemter.models.HistoricalOrganisationseinheit.history_type}}\pysigline{\sphinxbfcode{\sphinxupquote{history\_type}}}
A wrapper for a deferred\sphinxhyphen{}loading field. When the value is read from this
object the first time, the query is executed.

\end{fulllineitems}

\index{history\_user (aemter.models.HistoricalOrganisationseinheit attribute)@\spxentry{history\_user}\spxextra{aemter.models.HistoricalOrganisationseinheit attribute}}

\begin{fulllineitems}
\phantomsection\label{\detokenize{masterCodeDoc:aemter.models.HistoricalOrganisationseinheit.history_user}}\pysigline{\sphinxbfcode{\sphinxupquote{history\_user}}}
Accessor to the related object on the forward side of a many\sphinxhyphen{}to\sphinxhyphen{}one or
one\sphinxhyphen{}to\sphinxhyphen{}one (via ForwardOneToOneDescriptor subclass) relation.

In the example:

\begin{sphinxVerbatim}[commandchars=\\\{\}]
\PYG{k}{class} \PYG{n+nc}{Child}\PYG{p}{(}\PYG{n}{Model}\PYG{p}{)}\PYG{p}{:}
    \PYG{n}{parent} \PYG{o}{=} \PYG{n}{ForeignKey}\PYG{p}{(}\PYG{n}{Parent}\PYG{p}{,} \PYG{n}{related\PYGZus{}name}\PYG{o}{=}\PYG{l+s+s1}{\PYGZsq{}}\PYG{l+s+s1}{children}\PYG{l+s+s1}{\PYGZsq{}}\PYG{p}{)}
\end{sphinxVerbatim}

\sphinxcode{\sphinxupquote{Child.parent}} is a \sphinxcode{\sphinxupquote{ForwardManyToOneDescriptor}} instance.

\end{fulllineitems}

\index{history\_user\_id (aemter.models.HistoricalOrganisationseinheit attribute)@\spxentry{history\_user\_id}\spxextra{aemter.models.HistoricalOrganisationseinheit attribute}}

\begin{fulllineitems}
\phantomsection\label{\detokenize{masterCodeDoc:aemter.models.HistoricalOrganisationseinheit.history_user_id}}\pysigline{\sphinxbfcode{\sphinxupquote{history\_user\_id}}}
\end{fulllineitems}

\index{id (aemter.models.HistoricalOrganisationseinheit attribute)@\spxentry{id}\spxextra{aemter.models.HistoricalOrganisationseinheit attribute}}

\begin{fulllineitems}
\phantomsection\label{\detokenize{masterCodeDoc:aemter.models.HistoricalOrganisationseinheit.id}}\pysigline{\sphinxbfcode{\sphinxupquote{id}}}
A wrapper for a deferred\sphinxhyphen{}loading field. When the value is read from this
object the first time, the query is executed.

\end{fulllineitems}

\index{instance() (aemter.models.HistoricalOrganisationseinheit property)@\spxentry{instance()}\spxextra{aemter.models.HistoricalOrganisationseinheit property}}

\begin{fulllineitems}
\phantomsection\label{\detokenize{masterCodeDoc:aemter.models.HistoricalOrganisationseinheit.instance}}\pysigline{\sphinxbfcode{\sphinxupquote{property }}\sphinxbfcode{\sphinxupquote{instance}}}
\end{fulllineitems}

\index{instance\_type (aemter.models.HistoricalOrganisationseinheit attribute)@\spxentry{instance\_type}\spxextra{aemter.models.HistoricalOrganisationseinheit attribute}}

\begin{fulllineitems}
\phantomsection\label{\detokenize{masterCodeDoc:aemter.models.HistoricalOrganisationseinheit.instance_type}}\pysigline{\sphinxbfcode{\sphinxupquote{instance\_type}}}
alias of {\hyperref[\detokenize{masterCodeDoc:aemter.models.Organisationseinheit}]{\sphinxcrossref{\sphinxcode{\sphinxupquote{Organisationseinheit}}}}}

\end{fulllineitems}

\index{next\_record() (aemter.models.HistoricalOrganisationseinheit property)@\spxentry{next\_record()}\spxextra{aemter.models.HistoricalOrganisationseinheit property}}

\begin{fulllineitems}
\phantomsection\label{\detokenize{masterCodeDoc:aemter.models.HistoricalOrganisationseinheit.next_record}}\pysigline{\sphinxbfcode{\sphinxupquote{property }}\sphinxbfcode{\sphinxupquote{next\_record}}}
Get the next history record for the instance. \sphinxtitleref{None} if last.

\end{fulllineitems}

\index{objects (aemter.models.HistoricalOrganisationseinheit attribute)@\spxentry{objects}\spxextra{aemter.models.HistoricalOrganisationseinheit attribute}}

\begin{fulllineitems}
\phantomsection\label{\detokenize{masterCodeDoc:aemter.models.HistoricalOrganisationseinheit.objects}}\pysigline{\sphinxbfcode{\sphinxupquote{objects}}\sphinxbfcode{\sphinxupquote{ = \textless{}django.db.models.manager.Manager object\textgreater{}}}}
\end{fulllineitems}

\index{prev\_record() (aemter.models.HistoricalOrganisationseinheit property)@\spxentry{prev\_record()}\spxextra{aemter.models.HistoricalOrganisationseinheit property}}

\begin{fulllineitems}
\phantomsection\label{\detokenize{masterCodeDoc:aemter.models.HistoricalOrganisationseinheit.prev_record}}\pysigline{\sphinxbfcode{\sphinxupquote{property }}\sphinxbfcode{\sphinxupquote{prev\_record}}}
Get the previous history record for the instance. \sphinxtitleref{None} if first.

\end{fulllineitems}

\index{revert\_url() (aemter.models.HistoricalOrganisationseinheit method)@\spxentry{revert\_url()}\spxextra{aemter.models.HistoricalOrganisationseinheit method}}

\begin{fulllineitems}
\phantomsection\label{\detokenize{masterCodeDoc:aemter.models.HistoricalOrganisationseinheit.revert_url}}\pysiglinewithargsret{\sphinxbfcode{\sphinxupquote{revert\_url}}}{}{}
URL for this change in the default admin site.

\end{fulllineitems}


\end{fulllineitems}

\index{HistoricalRecht (class in aemter.models)@\spxentry{HistoricalRecht}\spxextra{class in aemter.models}}

\begin{fulllineitems}
\phantomsection\label{\detokenize{masterCodeDoc:aemter.models.HistoricalRecht}}\pysiglinewithargsret{\sphinxbfcode{\sphinxupquote{class }}\sphinxcode{\sphinxupquote{aemter.models.}}\sphinxbfcode{\sphinxupquote{HistoricalRecht}}}{\emph{\DUrole{n}{id}}, \emph{\DUrole{n}{bezeichnung}}, \emph{\DUrole{n}{history\_id}}, \emph{\DUrole{n}{history\_date}}, \emph{\DUrole{n}{history\_change\_reason}}, \emph{\DUrole{n}{history\_type}}, \emph{\DUrole{n}{history\_user}}}{}~\index{HistoricalRecht.DoesNotExist@\spxentry{HistoricalRecht.DoesNotExist}}

\begin{fulllineitems}
\phantomsection\label{\detokenize{masterCodeDoc:aemter.models.HistoricalRecht.DoesNotExist}}\pysigline{\sphinxbfcode{\sphinxupquote{exception }}\sphinxbfcode{\sphinxupquote{DoesNotExist}}}
\end{fulllineitems}

\index{HistoricalRecht.MultipleObjectsReturned@\spxentry{HistoricalRecht.MultipleObjectsReturned}}

\begin{fulllineitems}
\phantomsection\label{\detokenize{masterCodeDoc:aemter.models.HistoricalRecht.MultipleObjectsReturned}}\pysigline{\sphinxbfcode{\sphinxupquote{exception }}\sphinxbfcode{\sphinxupquote{MultipleObjectsReturned}}}
\end{fulllineitems}

\index{bezeichnung (aemter.models.HistoricalRecht attribute)@\spxentry{bezeichnung}\spxextra{aemter.models.HistoricalRecht attribute}}

\begin{fulllineitems}
\phantomsection\label{\detokenize{masterCodeDoc:aemter.models.HistoricalRecht.bezeichnung}}\pysigline{\sphinxbfcode{\sphinxupquote{bezeichnung}}}
A wrapper for a deferred\sphinxhyphen{}loading field. When the value is read from this
object the first time, the query is executed.

\end{fulllineitems}

\index{get\_default\_history\_user() (aemter.models.HistoricalRecht static method)@\spxentry{get\_default\_history\_user()}\spxextra{aemter.models.HistoricalRecht static method}}

\begin{fulllineitems}
\phantomsection\label{\detokenize{masterCodeDoc:aemter.models.HistoricalRecht.get_default_history_user}}\pysiglinewithargsret{\sphinxbfcode{\sphinxupquote{static }}\sphinxbfcode{\sphinxupquote{get\_default\_history\_user}}}{\emph{\DUrole{n}{instance}}}{}
Returns the user specified by \sphinxtitleref{get\_user} method for manually creating
historical objects

\end{fulllineitems}

\index{get\_history\_type\_display() (aemter.models.HistoricalRecht method)@\spxentry{get\_history\_type\_display()}\spxextra{aemter.models.HistoricalRecht method}}

\begin{fulllineitems}
\phantomsection\label{\detokenize{masterCodeDoc:aemter.models.HistoricalRecht.get_history_type_display}}\pysiglinewithargsret{\sphinxbfcode{\sphinxupquote{get\_history\_type\_display}}}{\emph{*}, \emph{field=\textless{}django.db.models.fields.CharField: history\_type\textgreater{}}}{}
\end{fulllineitems}

\index{get\_next\_by\_history\_date() (aemter.models.HistoricalRecht method)@\spxentry{get\_next\_by\_history\_date()}\spxextra{aemter.models.HistoricalRecht method}}

\begin{fulllineitems}
\phantomsection\label{\detokenize{masterCodeDoc:aemter.models.HistoricalRecht.get_next_by_history_date}}\pysiglinewithargsret{\sphinxbfcode{\sphinxupquote{get\_next\_by\_history\_date}}}{\emph{*}, \emph{field=\textless{}django.db.models.fields.DateTimeField: history\_date\textgreater{}}, \emph{is\_next=True}, \emph{**kwargs}}{}
\end{fulllineitems}

\index{get\_previous\_by\_history\_date() (aemter.models.HistoricalRecht method)@\spxentry{get\_previous\_by\_history\_date()}\spxextra{aemter.models.HistoricalRecht method}}

\begin{fulllineitems}
\phantomsection\label{\detokenize{masterCodeDoc:aemter.models.HistoricalRecht.get_previous_by_history_date}}\pysiglinewithargsret{\sphinxbfcode{\sphinxupquote{get\_previous\_by\_history\_date}}}{\emph{*}, \emph{field=\textless{}django.db.models.fields.DateTimeField: history\_date\textgreater{}}, \emph{is\_next=False}, \emph{**kwargs}}{}
\end{fulllineitems}

\index{history\_change\_reason (aemter.models.HistoricalRecht attribute)@\spxentry{history\_change\_reason}\spxextra{aemter.models.HistoricalRecht attribute}}

\begin{fulllineitems}
\phantomsection\label{\detokenize{masterCodeDoc:aemter.models.HistoricalRecht.history_change_reason}}\pysigline{\sphinxbfcode{\sphinxupquote{history\_change\_reason}}}
A wrapper for a deferred\sphinxhyphen{}loading field. When the value is read from this
object the first time, the query is executed.

\end{fulllineitems}

\index{history\_date (aemter.models.HistoricalRecht attribute)@\spxentry{history\_date}\spxextra{aemter.models.HistoricalRecht attribute}}

\begin{fulllineitems}
\phantomsection\label{\detokenize{masterCodeDoc:aemter.models.HistoricalRecht.history_date}}\pysigline{\sphinxbfcode{\sphinxupquote{history\_date}}}
A wrapper for a deferred\sphinxhyphen{}loading field. When the value is read from this
object the first time, the query is executed.

\end{fulllineitems}

\index{history\_id (aemter.models.HistoricalRecht attribute)@\spxentry{history\_id}\spxextra{aemter.models.HistoricalRecht attribute}}

\begin{fulllineitems}
\phantomsection\label{\detokenize{masterCodeDoc:aemter.models.HistoricalRecht.history_id}}\pysigline{\sphinxbfcode{\sphinxupquote{history\_id}}}
A wrapper for a deferred\sphinxhyphen{}loading field. When the value is read from this
object the first time, the query is executed.

\end{fulllineitems}

\index{history\_object (aemter.models.HistoricalRecht attribute)@\spxentry{history\_object}\spxextra{aemter.models.HistoricalRecht attribute}}

\begin{fulllineitems}
\phantomsection\label{\detokenize{masterCodeDoc:aemter.models.HistoricalRecht.history_object}}\pysigline{\sphinxbfcode{\sphinxupquote{history\_object}}}
\end{fulllineitems}

\index{history\_type (aemter.models.HistoricalRecht attribute)@\spxentry{history\_type}\spxextra{aemter.models.HistoricalRecht attribute}}

\begin{fulllineitems}
\phantomsection\label{\detokenize{masterCodeDoc:aemter.models.HistoricalRecht.history_type}}\pysigline{\sphinxbfcode{\sphinxupquote{history\_type}}}
A wrapper for a deferred\sphinxhyphen{}loading field. When the value is read from this
object the first time, the query is executed.

\end{fulllineitems}

\index{history\_user (aemter.models.HistoricalRecht attribute)@\spxentry{history\_user}\spxextra{aemter.models.HistoricalRecht attribute}}

\begin{fulllineitems}
\phantomsection\label{\detokenize{masterCodeDoc:aemter.models.HistoricalRecht.history_user}}\pysigline{\sphinxbfcode{\sphinxupquote{history\_user}}}
Accessor to the related object on the forward side of a many\sphinxhyphen{}to\sphinxhyphen{}one or
one\sphinxhyphen{}to\sphinxhyphen{}one (via ForwardOneToOneDescriptor subclass) relation.

In the example:

\begin{sphinxVerbatim}[commandchars=\\\{\}]
\PYG{k}{class} \PYG{n+nc}{Child}\PYG{p}{(}\PYG{n}{Model}\PYG{p}{)}\PYG{p}{:}
    \PYG{n}{parent} \PYG{o}{=} \PYG{n}{ForeignKey}\PYG{p}{(}\PYG{n}{Parent}\PYG{p}{,} \PYG{n}{related\PYGZus{}name}\PYG{o}{=}\PYG{l+s+s1}{\PYGZsq{}}\PYG{l+s+s1}{children}\PYG{l+s+s1}{\PYGZsq{}}\PYG{p}{)}
\end{sphinxVerbatim}

\sphinxcode{\sphinxupquote{Child.parent}} is a \sphinxcode{\sphinxupquote{ForwardManyToOneDescriptor}} instance.

\end{fulllineitems}

\index{history\_user\_id (aemter.models.HistoricalRecht attribute)@\spxentry{history\_user\_id}\spxextra{aemter.models.HistoricalRecht attribute}}

\begin{fulllineitems}
\phantomsection\label{\detokenize{masterCodeDoc:aemter.models.HistoricalRecht.history_user_id}}\pysigline{\sphinxbfcode{\sphinxupquote{history\_user\_id}}}
\end{fulllineitems}

\index{id (aemter.models.HistoricalRecht attribute)@\spxentry{id}\spxextra{aemter.models.HistoricalRecht attribute}}

\begin{fulllineitems}
\phantomsection\label{\detokenize{masterCodeDoc:aemter.models.HistoricalRecht.id}}\pysigline{\sphinxbfcode{\sphinxupquote{id}}}
A wrapper for a deferred\sphinxhyphen{}loading field. When the value is read from this
object the first time, the query is executed.

\end{fulllineitems}

\index{instance() (aemter.models.HistoricalRecht property)@\spxentry{instance()}\spxextra{aemter.models.HistoricalRecht property}}

\begin{fulllineitems}
\phantomsection\label{\detokenize{masterCodeDoc:aemter.models.HistoricalRecht.instance}}\pysigline{\sphinxbfcode{\sphinxupquote{property }}\sphinxbfcode{\sphinxupquote{instance}}}
\end{fulllineitems}

\index{instance\_type (aemter.models.HistoricalRecht attribute)@\spxentry{instance\_type}\spxextra{aemter.models.HistoricalRecht attribute}}

\begin{fulllineitems}
\phantomsection\label{\detokenize{masterCodeDoc:aemter.models.HistoricalRecht.instance_type}}\pysigline{\sphinxbfcode{\sphinxupquote{instance\_type}}}
alias of {\hyperref[\detokenize{masterCodeDoc:aemter.models.Recht}]{\sphinxcrossref{\sphinxcode{\sphinxupquote{Recht}}}}}

\end{fulllineitems}

\index{next\_record() (aemter.models.HistoricalRecht property)@\spxentry{next\_record()}\spxextra{aemter.models.HistoricalRecht property}}

\begin{fulllineitems}
\phantomsection\label{\detokenize{masterCodeDoc:aemter.models.HistoricalRecht.next_record}}\pysigline{\sphinxbfcode{\sphinxupquote{property }}\sphinxbfcode{\sphinxupquote{next\_record}}}
Get the next history record for the instance. \sphinxtitleref{None} if last.

\end{fulllineitems}

\index{objects (aemter.models.HistoricalRecht attribute)@\spxentry{objects}\spxextra{aemter.models.HistoricalRecht attribute}}

\begin{fulllineitems}
\phantomsection\label{\detokenize{masterCodeDoc:aemter.models.HistoricalRecht.objects}}\pysigline{\sphinxbfcode{\sphinxupquote{objects}}\sphinxbfcode{\sphinxupquote{ = \textless{}django.db.models.manager.Manager object\textgreater{}}}}
\end{fulllineitems}

\index{prev\_record() (aemter.models.HistoricalRecht property)@\spxentry{prev\_record()}\spxextra{aemter.models.HistoricalRecht property}}

\begin{fulllineitems}
\phantomsection\label{\detokenize{masterCodeDoc:aemter.models.HistoricalRecht.prev_record}}\pysigline{\sphinxbfcode{\sphinxupquote{property }}\sphinxbfcode{\sphinxupquote{prev\_record}}}
Get the previous history record for the instance. \sphinxtitleref{None} if first.

\end{fulllineitems}

\index{revert\_url() (aemter.models.HistoricalRecht method)@\spxentry{revert\_url()}\spxextra{aemter.models.HistoricalRecht method}}

\begin{fulllineitems}
\phantomsection\label{\detokenize{masterCodeDoc:aemter.models.HistoricalRecht.revert_url}}\pysiglinewithargsret{\sphinxbfcode{\sphinxupquote{revert\_url}}}{}{}
URL for this change in the default admin site.

\end{fulllineitems}


\end{fulllineitems}

\index{HistoricalUnterbereich (class in aemter.models)@\spxentry{HistoricalUnterbereich}\spxextra{class in aemter.models}}

\begin{fulllineitems}
\phantomsection\label{\detokenize{masterCodeDoc:aemter.models.HistoricalUnterbereich}}\pysiglinewithargsret{\sphinxbfcode{\sphinxupquote{class }}\sphinxcode{\sphinxupquote{aemter.models.}}\sphinxbfcode{\sphinxupquote{HistoricalUnterbereich}}}{\emph{\DUrole{n}{id}}, \emph{\DUrole{n}{bezeichnung}}, \emph{\DUrole{n}{organisationseinheit}}, \emph{\DUrole{n}{history\_id}}, \emph{\DUrole{n}{history\_date}}, \emph{\DUrole{n}{history\_change\_reason}}, \emph{\DUrole{n}{history\_type}}, \emph{\DUrole{n}{history\_user}}}{}~\index{HistoricalUnterbereich.DoesNotExist@\spxentry{HistoricalUnterbereich.DoesNotExist}}

\begin{fulllineitems}
\phantomsection\label{\detokenize{masterCodeDoc:aemter.models.HistoricalUnterbereich.DoesNotExist}}\pysigline{\sphinxbfcode{\sphinxupquote{exception }}\sphinxbfcode{\sphinxupquote{DoesNotExist}}}
\end{fulllineitems}

\index{HistoricalUnterbereich.MultipleObjectsReturned@\spxentry{HistoricalUnterbereich.MultipleObjectsReturned}}

\begin{fulllineitems}
\phantomsection\label{\detokenize{masterCodeDoc:aemter.models.HistoricalUnterbereich.MultipleObjectsReturned}}\pysigline{\sphinxbfcode{\sphinxupquote{exception }}\sphinxbfcode{\sphinxupquote{MultipleObjectsReturned}}}
\end{fulllineitems}

\index{bezeichnung (aemter.models.HistoricalUnterbereich attribute)@\spxentry{bezeichnung}\spxextra{aemter.models.HistoricalUnterbereich attribute}}

\begin{fulllineitems}
\phantomsection\label{\detokenize{masterCodeDoc:aemter.models.HistoricalUnterbereich.bezeichnung}}\pysigline{\sphinxbfcode{\sphinxupquote{bezeichnung}}}
A wrapper for a deferred\sphinxhyphen{}loading field. When the value is read from this
object the first time, the query is executed.

\end{fulllineitems}

\index{get\_default\_history\_user() (aemter.models.HistoricalUnterbereich static method)@\spxentry{get\_default\_history\_user()}\spxextra{aemter.models.HistoricalUnterbereich static method}}

\begin{fulllineitems}
\phantomsection\label{\detokenize{masterCodeDoc:aemter.models.HistoricalUnterbereich.get_default_history_user}}\pysiglinewithargsret{\sphinxbfcode{\sphinxupquote{static }}\sphinxbfcode{\sphinxupquote{get\_default\_history\_user}}}{\emph{\DUrole{n}{instance}}}{}
Returns the user specified by \sphinxtitleref{get\_user} method for manually creating
historical objects

\end{fulllineitems}

\index{get\_history\_type\_display() (aemter.models.HistoricalUnterbereich method)@\spxentry{get\_history\_type\_display()}\spxextra{aemter.models.HistoricalUnterbereich method}}

\begin{fulllineitems}
\phantomsection\label{\detokenize{masterCodeDoc:aemter.models.HistoricalUnterbereich.get_history_type_display}}\pysiglinewithargsret{\sphinxbfcode{\sphinxupquote{get\_history\_type\_display}}}{\emph{*}, \emph{field=\textless{}django.db.models.fields.CharField: history\_type\textgreater{}}}{}
\end{fulllineitems}

\index{get\_next\_by\_history\_date() (aemter.models.HistoricalUnterbereich method)@\spxentry{get\_next\_by\_history\_date()}\spxextra{aemter.models.HistoricalUnterbereich method}}

\begin{fulllineitems}
\phantomsection\label{\detokenize{masterCodeDoc:aemter.models.HistoricalUnterbereich.get_next_by_history_date}}\pysiglinewithargsret{\sphinxbfcode{\sphinxupquote{get\_next\_by\_history\_date}}}{\emph{*}, \emph{field=\textless{}django.db.models.fields.DateTimeField: history\_date\textgreater{}}, \emph{is\_next=True}, \emph{**kwargs}}{}
\end{fulllineitems}

\index{get\_previous\_by\_history\_date() (aemter.models.HistoricalUnterbereich method)@\spxentry{get\_previous\_by\_history\_date()}\spxextra{aemter.models.HistoricalUnterbereich method}}

\begin{fulllineitems}
\phantomsection\label{\detokenize{masterCodeDoc:aemter.models.HistoricalUnterbereich.get_previous_by_history_date}}\pysiglinewithargsret{\sphinxbfcode{\sphinxupquote{get\_previous\_by\_history\_date}}}{\emph{*}, \emph{field=\textless{}django.db.models.fields.DateTimeField: history\_date\textgreater{}}, \emph{is\_next=False}, \emph{**kwargs}}{}
\end{fulllineitems}

\index{history\_change\_reason (aemter.models.HistoricalUnterbereich attribute)@\spxentry{history\_change\_reason}\spxextra{aemter.models.HistoricalUnterbereich attribute}}

\begin{fulllineitems}
\phantomsection\label{\detokenize{masterCodeDoc:aemter.models.HistoricalUnterbereich.history_change_reason}}\pysigline{\sphinxbfcode{\sphinxupquote{history\_change\_reason}}}
A wrapper for a deferred\sphinxhyphen{}loading field. When the value is read from this
object the first time, the query is executed.

\end{fulllineitems}

\index{history\_date (aemter.models.HistoricalUnterbereich attribute)@\spxentry{history\_date}\spxextra{aemter.models.HistoricalUnterbereich attribute}}

\begin{fulllineitems}
\phantomsection\label{\detokenize{masterCodeDoc:aemter.models.HistoricalUnterbereich.history_date}}\pysigline{\sphinxbfcode{\sphinxupquote{history\_date}}}
A wrapper for a deferred\sphinxhyphen{}loading field. When the value is read from this
object the first time, the query is executed.

\end{fulllineitems}

\index{history\_id (aemter.models.HistoricalUnterbereich attribute)@\spxentry{history\_id}\spxextra{aemter.models.HistoricalUnterbereich attribute}}

\begin{fulllineitems}
\phantomsection\label{\detokenize{masterCodeDoc:aemter.models.HistoricalUnterbereich.history_id}}\pysigline{\sphinxbfcode{\sphinxupquote{history\_id}}}
A wrapper for a deferred\sphinxhyphen{}loading field. When the value is read from this
object the first time, the query is executed.

\end{fulllineitems}

\index{history\_object (aemter.models.HistoricalUnterbereich attribute)@\spxentry{history\_object}\spxextra{aemter.models.HistoricalUnterbereich attribute}}

\begin{fulllineitems}
\phantomsection\label{\detokenize{masterCodeDoc:aemter.models.HistoricalUnterbereich.history_object}}\pysigline{\sphinxbfcode{\sphinxupquote{history\_object}}}
\end{fulllineitems}

\index{history\_type (aemter.models.HistoricalUnterbereich attribute)@\spxentry{history\_type}\spxextra{aemter.models.HistoricalUnterbereich attribute}}

\begin{fulllineitems}
\phantomsection\label{\detokenize{masterCodeDoc:aemter.models.HistoricalUnterbereich.history_type}}\pysigline{\sphinxbfcode{\sphinxupquote{history\_type}}}
A wrapper for a deferred\sphinxhyphen{}loading field. When the value is read from this
object the first time, the query is executed.

\end{fulllineitems}

\index{history\_user (aemter.models.HistoricalUnterbereich attribute)@\spxentry{history\_user}\spxextra{aemter.models.HistoricalUnterbereich attribute}}

\begin{fulllineitems}
\phantomsection\label{\detokenize{masterCodeDoc:aemter.models.HistoricalUnterbereich.history_user}}\pysigline{\sphinxbfcode{\sphinxupquote{history\_user}}}
Accessor to the related object on the forward side of a many\sphinxhyphen{}to\sphinxhyphen{}one or
one\sphinxhyphen{}to\sphinxhyphen{}one (via ForwardOneToOneDescriptor subclass) relation.

In the example:

\begin{sphinxVerbatim}[commandchars=\\\{\}]
\PYG{k}{class} \PYG{n+nc}{Child}\PYG{p}{(}\PYG{n}{Model}\PYG{p}{)}\PYG{p}{:}
    \PYG{n}{parent} \PYG{o}{=} \PYG{n}{ForeignKey}\PYG{p}{(}\PYG{n}{Parent}\PYG{p}{,} \PYG{n}{related\PYGZus{}name}\PYG{o}{=}\PYG{l+s+s1}{\PYGZsq{}}\PYG{l+s+s1}{children}\PYG{l+s+s1}{\PYGZsq{}}\PYG{p}{)}
\end{sphinxVerbatim}

\sphinxcode{\sphinxupquote{Child.parent}} is a \sphinxcode{\sphinxupquote{ForwardManyToOneDescriptor}} instance.

\end{fulllineitems}

\index{history\_user\_id (aemter.models.HistoricalUnterbereich attribute)@\spxentry{history\_user\_id}\spxextra{aemter.models.HistoricalUnterbereich attribute}}

\begin{fulllineitems}
\phantomsection\label{\detokenize{masterCodeDoc:aemter.models.HistoricalUnterbereich.history_user_id}}\pysigline{\sphinxbfcode{\sphinxupquote{history\_user\_id}}}
\end{fulllineitems}

\index{id (aemter.models.HistoricalUnterbereich attribute)@\spxentry{id}\spxextra{aemter.models.HistoricalUnterbereich attribute}}

\begin{fulllineitems}
\phantomsection\label{\detokenize{masterCodeDoc:aemter.models.HistoricalUnterbereich.id}}\pysigline{\sphinxbfcode{\sphinxupquote{id}}}
A wrapper for a deferred\sphinxhyphen{}loading field. When the value is read from this
object the first time, the query is executed.

\end{fulllineitems}

\index{instance() (aemter.models.HistoricalUnterbereich property)@\spxentry{instance()}\spxextra{aemter.models.HistoricalUnterbereich property}}

\begin{fulllineitems}
\phantomsection\label{\detokenize{masterCodeDoc:aemter.models.HistoricalUnterbereich.instance}}\pysigline{\sphinxbfcode{\sphinxupquote{property }}\sphinxbfcode{\sphinxupquote{instance}}}
\end{fulllineitems}

\index{instance\_type (aemter.models.HistoricalUnterbereich attribute)@\spxentry{instance\_type}\spxextra{aemter.models.HistoricalUnterbereich attribute}}

\begin{fulllineitems}
\phantomsection\label{\detokenize{masterCodeDoc:aemter.models.HistoricalUnterbereich.instance_type}}\pysigline{\sphinxbfcode{\sphinxupquote{instance\_type}}}
alias of {\hyperref[\detokenize{masterCodeDoc:aemter.models.Unterbereich}]{\sphinxcrossref{\sphinxcode{\sphinxupquote{Unterbereich}}}}}

\end{fulllineitems}

\index{next\_record() (aemter.models.HistoricalUnterbereich property)@\spxentry{next\_record()}\spxextra{aemter.models.HistoricalUnterbereich property}}

\begin{fulllineitems}
\phantomsection\label{\detokenize{masterCodeDoc:aemter.models.HistoricalUnterbereich.next_record}}\pysigline{\sphinxbfcode{\sphinxupquote{property }}\sphinxbfcode{\sphinxupquote{next\_record}}}
Get the next history record for the instance. \sphinxtitleref{None} if last.

\end{fulllineitems}

\index{objects (aemter.models.HistoricalUnterbereich attribute)@\spxentry{objects}\spxextra{aemter.models.HistoricalUnterbereich attribute}}

\begin{fulllineitems}
\phantomsection\label{\detokenize{masterCodeDoc:aemter.models.HistoricalUnterbereich.objects}}\pysigline{\sphinxbfcode{\sphinxupquote{objects}}\sphinxbfcode{\sphinxupquote{ = \textless{}django.db.models.manager.Manager object\textgreater{}}}}
\end{fulllineitems}

\index{organisationseinheit (aemter.models.HistoricalUnterbereich attribute)@\spxentry{organisationseinheit}\spxextra{aemter.models.HistoricalUnterbereich attribute}}

\begin{fulllineitems}
\phantomsection\label{\detokenize{masterCodeDoc:aemter.models.HistoricalUnterbereich.organisationseinheit}}\pysigline{\sphinxbfcode{\sphinxupquote{organisationseinheit}}}
Accessor to the related object on the forward side of a many\sphinxhyphen{}to\sphinxhyphen{}one or
one\sphinxhyphen{}to\sphinxhyphen{}one (via ForwardOneToOneDescriptor subclass) relation.

In the example:

\begin{sphinxVerbatim}[commandchars=\\\{\}]
\PYG{k}{class} \PYG{n+nc}{Child}\PYG{p}{(}\PYG{n}{Model}\PYG{p}{)}\PYG{p}{:}
    \PYG{n}{parent} \PYG{o}{=} \PYG{n}{ForeignKey}\PYG{p}{(}\PYG{n}{Parent}\PYG{p}{,} \PYG{n}{related\PYGZus{}name}\PYG{o}{=}\PYG{l+s+s1}{\PYGZsq{}}\PYG{l+s+s1}{children}\PYG{l+s+s1}{\PYGZsq{}}\PYG{p}{)}
\end{sphinxVerbatim}

\sphinxcode{\sphinxupquote{Child.parent}} is a \sphinxcode{\sphinxupquote{ForwardManyToOneDescriptor}} instance.

\end{fulllineitems}

\index{organisationseinheit\_id (aemter.models.HistoricalUnterbereich attribute)@\spxentry{organisationseinheit\_id}\spxextra{aemter.models.HistoricalUnterbereich attribute}}

\begin{fulllineitems}
\phantomsection\label{\detokenize{masterCodeDoc:aemter.models.HistoricalUnterbereich.organisationseinheit_id}}\pysigline{\sphinxbfcode{\sphinxupquote{organisationseinheit\_id}}}
\end{fulllineitems}

\index{prev\_record() (aemter.models.HistoricalUnterbereich property)@\spxentry{prev\_record()}\spxextra{aemter.models.HistoricalUnterbereich property}}

\begin{fulllineitems}
\phantomsection\label{\detokenize{masterCodeDoc:aemter.models.HistoricalUnterbereich.prev_record}}\pysigline{\sphinxbfcode{\sphinxupquote{property }}\sphinxbfcode{\sphinxupquote{prev\_record}}}
Get the previous history record for the instance. \sphinxtitleref{None} if first.

\end{fulllineitems}

\index{revert\_url() (aemter.models.HistoricalUnterbereich method)@\spxentry{revert\_url()}\spxextra{aemter.models.HistoricalUnterbereich method}}

\begin{fulllineitems}
\phantomsection\label{\detokenize{masterCodeDoc:aemter.models.HistoricalUnterbereich.revert_url}}\pysiglinewithargsret{\sphinxbfcode{\sphinxupquote{revert\_url}}}{}{}
URL for this change in the default admin site.

\end{fulllineitems}


\end{fulllineitems}

\index{Organisationseinheit (class in aemter.models)@\spxentry{Organisationseinheit}\spxextra{class in aemter.models}}

\begin{fulllineitems}
\phantomsection\label{\detokenize{masterCodeDoc:aemter.models.Organisationseinheit}}\pysiglinewithargsret{\sphinxbfcode{\sphinxupquote{class }}\sphinxcode{\sphinxupquote{aemter.models.}}\sphinxbfcode{\sphinxupquote{Organisationseinheit}}}{\emph{\DUrole{o}{*}\DUrole{n}{args}}, \emph{\DUrole{o}{**}\DUrole{n}{kwargs}}}{}
Datenbankmodel Organisationseinheit

Felder:
\begin{itemize}
\item {} 
bezeichnung

\item {} 
history

\item {} 
funktionen\_ohne\_unterbereich\_count

\end{itemize}
\index{Organisationseinheit.DoesNotExist@\spxentry{Organisationseinheit.DoesNotExist}}

\begin{fulllineitems}
\phantomsection\label{\detokenize{masterCodeDoc:aemter.models.Organisationseinheit.DoesNotExist}}\pysigline{\sphinxbfcode{\sphinxupquote{exception }}\sphinxbfcode{\sphinxupquote{DoesNotExist}}}
\end{fulllineitems}

\index{Organisationseinheit.MultipleObjectsReturned@\spxentry{Organisationseinheit.MultipleObjectsReturned}}

\begin{fulllineitems}
\phantomsection\label{\detokenize{masterCodeDoc:aemter.models.Organisationseinheit.MultipleObjectsReturned}}\pysigline{\sphinxbfcode{\sphinxupquote{exception }}\sphinxbfcode{\sphinxupquote{MultipleObjectsReturned}}}
\end{fulllineitems}

\index{bezeichnung (aemter.models.Organisationseinheit attribute)@\spxentry{bezeichnung}\spxextra{aemter.models.Organisationseinheit attribute}}

\begin{fulllineitems}
\phantomsection\label{\detokenize{masterCodeDoc:aemter.models.Organisationseinheit.bezeichnung}}\pysigline{\sphinxbfcode{\sphinxupquote{bezeichnung}}}
A wrapper for a deferred\sphinxhyphen{}loading field. When the value is read from this
object the first time, the query is executed.

\end{fulllineitems}

\index{funktion\_set (aemter.models.Organisationseinheit attribute)@\spxentry{funktion\_set}\spxextra{aemter.models.Organisationseinheit attribute}}

\begin{fulllineitems}
\phantomsection\label{\detokenize{masterCodeDoc:aemter.models.Organisationseinheit.funktion_set}}\pysigline{\sphinxbfcode{\sphinxupquote{funktion\_set}}}
Accessor to the related objects manager on the reverse side of a
many\sphinxhyphen{}to\sphinxhyphen{}one relation.

In the example:

\begin{sphinxVerbatim}[commandchars=\\\{\}]
\PYG{k}{class} \PYG{n+nc}{Child}\PYG{p}{(}\PYG{n}{Model}\PYG{p}{)}\PYG{p}{:}
    \PYG{n}{parent} \PYG{o}{=} \PYG{n}{ForeignKey}\PYG{p}{(}\PYG{n}{Parent}\PYG{p}{,} \PYG{n}{related\PYGZus{}name}\PYG{o}{=}\PYG{l+s+s1}{\PYGZsq{}}\PYG{l+s+s1}{children}\PYG{l+s+s1}{\PYGZsq{}}\PYG{p}{)}
\end{sphinxVerbatim}

\sphinxcode{\sphinxupquote{Parent.children}} is a \sphinxcode{\sphinxupquote{ReverseManyToOneDescriptor}} instance.

Most of the implementation is delegated to a dynamically defined manager
class built by \sphinxcode{\sphinxupquote{create\_forward\_many\_to\_many\_manager()}} defined below.

\end{fulllineitems}

\index{funktionen\_ohne\_unterbereich\_count (aemter.models.Organisationseinheit attribute)@\spxentry{funktionen\_ohne\_unterbereich\_count}\spxextra{aemter.models.Organisationseinheit attribute}}

\begin{fulllineitems}
\phantomsection\label{\detokenize{masterCodeDoc:aemter.models.Organisationseinheit.funktionen_ohne_unterbereich_count}}\pysigline{\sphinxbfcode{\sphinxupquote{funktionen\_ohne\_unterbereich\_count}}}
A wrapper for a deferred\sphinxhyphen{}loading field. When the value is read from this
object the first time, the query is executed.

\end{fulllineitems}

\index{history (aemter.models.Organisationseinheit attribute)@\spxentry{history}\spxextra{aemter.models.Organisationseinheit attribute}}

\begin{fulllineitems}
\phantomsection\label{\detokenize{masterCodeDoc:aemter.models.Organisationseinheit.history}}\pysigline{\sphinxbfcode{\sphinxupquote{history}}\sphinxbfcode{\sphinxupquote{ = \textless{}simple\_history.manager.HistoryManager object\textgreater{}}}}
\end{fulllineitems}

\index{id (aemter.models.Organisationseinheit attribute)@\spxentry{id}\spxextra{aemter.models.Organisationseinheit attribute}}

\begin{fulllineitems}
\phantomsection\label{\detokenize{masterCodeDoc:aemter.models.Organisationseinheit.id}}\pysigline{\sphinxbfcode{\sphinxupquote{id}}}
A wrapper for a deferred\sphinxhyphen{}loading field. When the value is read from this
object the first time, the query is executed.

\end{fulllineitems}

\index{objects (aemter.models.Organisationseinheit attribute)@\spxentry{objects}\spxextra{aemter.models.Organisationseinheit attribute}}

\begin{fulllineitems}
\phantomsection\label{\detokenize{masterCodeDoc:aemter.models.Organisationseinheit.objects}}\pysigline{\sphinxbfcode{\sphinxupquote{objects}}\sphinxbfcode{\sphinxupquote{ = \textless{}django.db.models.manager.Manager object\textgreater{}}}}
\end{fulllineitems}

\index{save\_without\_historical\_record() (aemter.models.Organisationseinheit method)@\spxentry{save\_without\_historical\_record()}\spxextra{aemter.models.Organisationseinheit method}}

\begin{fulllineitems}
\phantomsection\label{\detokenize{masterCodeDoc:aemter.models.Organisationseinheit.save_without_historical_record}}\pysiglinewithargsret{\sphinxbfcode{\sphinxupquote{save\_without\_historical\_record}}}{\emph{\DUrole{o}{*}\DUrole{n}{args}}, \emph{\DUrole{o}{**}\DUrole{n}{kwargs}}}{}
Save model without saving a historical record

Make sure you know what you’re doing before you use this method.

\end{fulllineitems}

\index{unterbereich\_set (aemter.models.Organisationseinheit attribute)@\spxentry{unterbereich\_set}\spxextra{aemter.models.Organisationseinheit attribute}}

\begin{fulllineitems}
\phantomsection\label{\detokenize{masterCodeDoc:aemter.models.Organisationseinheit.unterbereich_set}}\pysigline{\sphinxbfcode{\sphinxupquote{unterbereich\_set}}}
Accessor to the related objects manager on the reverse side of a
many\sphinxhyphen{}to\sphinxhyphen{}one relation.

In the example:

\begin{sphinxVerbatim}[commandchars=\\\{\}]
\PYG{k}{class} \PYG{n+nc}{Child}\PYG{p}{(}\PYG{n}{Model}\PYG{p}{)}\PYG{p}{:}
    \PYG{n}{parent} \PYG{o}{=} \PYG{n}{ForeignKey}\PYG{p}{(}\PYG{n}{Parent}\PYG{p}{,} \PYG{n}{related\PYGZus{}name}\PYG{o}{=}\PYG{l+s+s1}{\PYGZsq{}}\PYG{l+s+s1}{children}\PYG{l+s+s1}{\PYGZsq{}}\PYG{p}{)}
\end{sphinxVerbatim}

\sphinxcode{\sphinxupquote{Parent.children}} is a \sphinxcode{\sphinxupquote{ReverseManyToOneDescriptor}} instance.

Most of the implementation is delegated to a dynamically defined manager
class built by \sphinxcode{\sphinxupquote{create\_forward\_many\_to\_many\_manager()}} defined below.

\end{fulllineitems}


\end{fulllineitems}

\index{Recht (class in aemter.models)@\spxentry{Recht}\spxextra{class in aemter.models}}

\begin{fulllineitems}
\phantomsection\label{\detokenize{masterCodeDoc:aemter.models.Recht}}\pysiglinewithargsret{\sphinxbfcode{\sphinxupquote{class }}\sphinxcode{\sphinxupquote{aemter.models.}}\sphinxbfcode{\sphinxupquote{Recht}}}{\emph{\DUrole{o}{*}\DUrole{n}{args}}, \emph{\DUrole{o}{**}\DUrole{n}{kwargs}}}{}
Datenbankmodell Recht

Felder:
\begin{itemize}
\item {} 
bezeichnung

\item {} 
history

\end{itemize}
\index{Recht.DoesNotExist@\spxentry{Recht.DoesNotExist}}

\begin{fulllineitems}
\phantomsection\label{\detokenize{masterCodeDoc:aemter.models.Recht.DoesNotExist}}\pysigline{\sphinxbfcode{\sphinxupquote{exception }}\sphinxbfcode{\sphinxupquote{DoesNotExist}}}
\end{fulllineitems}

\index{Recht.MultipleObjectsReturned@\spxentry{Recht.MultipleObjectsReturned}}

\begin{fulllineitems}
\phantomsection\label{\detokenize{masterCodeDoc:aemter.models.Recht.MultipleObjectsReturned}}\pysigline{\sphinxbfcode{\sphinxupquote{exception }}\sphinxbfcode{\sphinxupquote{MultipleObjectsReturned}}}
\end{fulllineitems}

\index{bezeichnung (aemter.models.Recht attribute)@\spxentry{bezeichnung}\spxextra{aemter.models.Recht attribute}}

\begin{fulllineitems}
\phantomsection\label{\detokenize{masterCodeDoc:aemter.models.Recht.bezeichnung}}\pysigline{\sphinxbfcode{\sphinxupquote{bezeichnung}}}
A wrapper for a deferred\sphinxhyphen{}loading field. When the value is read from this
object the first time, the query is executed.

\end{fulllineitems}

\index{checklisterecht\_set (aemter.models.Recht attribute)@\spxentry{checklisterecht\_set}\spxextra{aemter.models.Recht attribute}}

\begin{fulllineitems}
\phantomsection\label{\detokenize{masterCodeDoc:aemter.models.Recht.checklisterecht_set}}\pysigline{\sphinxbfcode{\sphinxupquote{checklisterecht\_set}}}
Accessor to the related objects manager on the reverse side of a
many\sphinxhyphen{}to\sphinxhyphen{}one relation.

In the example:

\begin{sphinxVerbatim}[commandchars=\\\{\}]
\PYG{k}{class} \PYG{n+nc}{Child}\PYG{p}{(}\PYG{n}{Model}\PYG{p}{)}\PYG{p}{:}
    \PYG{n}{parent} \PYG{o}{=} \PYG{n}{ForeignKey}\PYG{p}{(}\PYG{n}{Parent}\PYG{p}{,} \PYG{n}{related\PYGZus{}name}\PYG{o}{=}\PYG{l+s+s1}{\PYGZsq{}}\PYG{l+s+s1}{children}\PYG{l+s+s1}{\PYGZsq{}}\PYG{p}{)}
\end{sphinxVerbatim}

\sphinxcode{\sphinxupquote{Parent.children}} is a \sphinxcode{\sphinxupquote{ReverseManyToOneDescriptor}} instance.

Most of the implementation is delegated to a dynamically defined manager
class built by \sphinxcode{\sphinxupquote{create\_forward\_many\_to\_many\_manager()}} defined below.

\end{fulllineitems}

\index{funktionrecht\_set (aemter.models.Recht attribute)@\spxentry{funktionrecht\_set}\spxextra{aemter.models.Recht attribute}}

\begin{fulllineitems}
\phantomsection\label{\detokenize{masterCodeDoc:aemter.models.Recht.funktionrecht_set}}\pysigline{\sphinxbfcode{\sphinxupquote{funktionrecht\_set}}}
Accessor to the related objects manager on the reverse side of a
many\sphinxhyphen{}to\sphinxhyphen{}one relation.

In the example:

\begin{sphinxVerbatim}[commandchars=\\\{\}]
\PYG{k}{class} \PYG{n+nc}{Child}\PYG{p}{(}\PYG{n}{Model}\PYG{p}{)}\PYG{p}{:}
    \PYG{n}{parent} \PYG{o}{=} \PYG{n}{ForeignKey}\PYG{p}{(}\PYG{n}{Parent}\PYG{p}{,} \PYG{n}{related\PYGZus{}name}\PYG{o}{=}\PYG{l+s+s1}{\PYGZsq{}}\PYG{l+s+s1}{children}\PYG{l+s+s1}{\PYGZsq{}}\PYG{p}{)}
\end{sphinxVerbatim}

\sphinxcode{\sphinxupquote{Parent.children}} is a \sphinxcode{\sphinxupquote{ReverseManyToOneDescriptor}} instance.

Most of the implementation is delegated to a dynamically defined manager
class built by \sphinxcode{\sphinxupquote{create\_forward\_many\_to\_many\_manager()}} defined below.

\end{fulllineitems}

\index{history (aemter.models.Recht attribute)@\spxentry{history}\spxextra{aemter.models.Recht attribute}}

\begin{fulllineitems}
\phantomsection\label{\detokenize{masterCodeDoc:aemter.models.Recht.history}}\pysigline{\sphinxbfcode{\sphinxupquote{history}}\sphinxbfcode{\sphinxupquote{ = \textless{}simple\_history.manager.HistoryManager object\textgreater{}}}}
\end{fulllineitems}

\index{id (aemter.models.Recht attribute)@\spxentry{id}\spxextra{aemter.models.Recht attribute}}

\begin{fulllineitems}
\phantomsection\label{\detokenize{masterCodeDoc:aemter.models.Recht.id}}\pysigline{\sphinxbfcode{\sphinxupquote{id}}}
A wrapper for a deferred\sphinxhyphen{}loading field. When the value is read from this
object the first time, the query is executed.

\end{fulllineitems}

\index{objects (aemter.models.Recht attribute)@\spxentry{objects}\spxextra{aemter.models.Recht attribute}}

\begin{fulllineitems}
\phantomsection\label{\detokenize{masterCodeDoc:aemter.models.Recht.objects}}\pysigline{\sphinxbfcode{\sphinxupquote{objects}}\sphinxbfcode{\sphinxupquote{ = \textless{}django.db.models.manager.Manager object\textgreater{}}}}
\end{fulllineitems}

\index{save\_without\_historical\_record() (aemter.models.Recht method)@\spxentry{save\_without\_historical\_record()}\spxextra{aemter.models.Recht method}}

\begin{fulllineitems}
\phantomsection\label{\detokenize{masterCodeDoc:aemter.models.Recht.save_without_historical_record}}\pysiglinewithargsret{\sphinxbfcode{\sphinxupquote{save\_without\_historical\_record}}}{\emph{\DUrole{o}{*}\DUrole{n}{args}}, \emph{\DUrole{o}{**}\DUrole{n}{kwargs}}}{}
Save model without saving a historical record

Make sure you know what you’re doing before you use this method.

\end{fulllineitems}


\end{fulllineitems}

\index{Unterbereich (class in aemter.models)@\spxentry{Unterbereich}\spxextra{class in aemter.models}}

\begin{fulllineitems}
\phantomsection\label{\detokenize{masterCodeDoc:aemter.models.Unterbereich}}\pysiglinewithargsret{\sphinxbfcode{\sphinxupquote{class }}\sphinxcode{\sphinxupquote{aemter.models.}}\sphinxbfcode{\sphinxupquote{Unterbereich}}}{\emph{\DUrole{o}{*}\DUrole{n}{args}}, \emph{\DUrole{o}{**}\DUrole{n}{kwargs}}}{}
Datenbankmodel Unterbereich

Felder:
\begin{itemize}
\item {} 
bezeichnung

\item {} 
organisationseinheit (Referenziert zugehörige Organisationseinheit)

\item {} 
history

\end{itemize}
\index{Unterbereich.DoesNotExist@\spxentry{Unterbereich.DoesNotExist}}

\begin{fulllineitems}
\phantomsection\label{\detokenize{masterCodeDoc:aemter.models.Unterbereich.DoesNotExist}}\pysigline{\sphinxbfcode{\sphinxupquote{exception }}\sphinxbfcode{\sphinxupquote{DoesNotExist}}}
\end{fulllineitems}

\index{Unterbereich.MultipleObjectsReturned@\spxentry{Unterbereich.MultipleObjectsReturned}}

\begin{fulllineitems}
\phantomsection\label{\detokenize{masterCodeDoc:aemter.models.Unterbereich.MultipleObjectsReturned}}\pysigline{\sphinxbfcode{\sphinxupquote{exception }}\sphinxbfcode{\sphinxupquote{MultipleObjectsReturned}}}
\end{fulllineitems}

\index{bezeichnung (aemter.models.Unterbereich attribute)@\spxentry{bezeichnung}\spxextra{aemter.models.Unterbereich attribute}}

\begin{fulllineitems}
\phantomsection\label{\detokenize{masterCodeDoc:aemter.models.Unterbereich.bezeichnung}}\pysigline{\sphinxbfcode{\sphinxupquote{bezeichnung}}}
A wrapper for a deferred\sphinxhyphen{}loading field. When the value is read from this
object the first time, the query is executed.

\end{fulllineitems}

\index{funktion\_set (aemter.models.Unterbereich attribute)@\spxentry{funktion\_set}\spxextra{aemter.models.Unterbereich attribute}}

\begin{fulllineitems}
\phantomsection\label{\detokenize{masterCodeDoc:aemter.models.Unterbereich.funktion_set}}\pysigline{\sphinxbfcode{\sphinxupquote{funktion\_set}}}
Accessor to the related objects manager on the reverse side of a
many\sphinxhyphen{}to\sphinxhyphen{}one relation.

In the example:

\begin{sphinxVerbatim}[commandchars=\\\{\}]
\PYG{k}{class} \PYG{n+nc}{Child}\PYG{p}{(}\PYG{n}{Model}\PYG{p}{)}\PYG{p}{:}
    \PYG{n}{parent} \PYG{o}{=} \PYG{n}{ForeignKey}\PYG{p}{(}\PYG{n}{Parent}\PYG{p}{,} \PYG{n}{related\PYGZus{}name}\PYG{o}{=}\PYG{l+s+s1}{\PYGZsq{}}\PYG{l+s+s1}{children}\PYG{l+s+s1}{\PYGZsq{}}\PYG{p}{)}
\end{sphinxVerbatim}

\sphinxcode{\sphinxupquote{Parent.children}} is a \sphinxcode{\sphinxupquote{ReverseManyToOneDescriptor}} instance.

Most of the implementation is delegated to a dynamically defined manager
class built by \sphinxcode{\sphinxupquote{create\_forward\_many\_to\_many\_manager()}} defined below.

\end{fulllineitems}

\index{history (aemter.models.Unterbereich attribute)@\spxentry{history}\spxextra{aemter.models.Unterbereich attribute}}

\begin{fulllineitems}
\phantomsection\label{\detokenize{masterCodeDoc:aemter.models.Unterbereich.history}}\pysigline{\sphinxbfcode{\sphinxupquote{history}}\sphinxbfcode{\sphinxupquote{ = \textless{}simple\_history.manager.HistoryManager object\textgreater{}}}}
\end{fulllineitems}

\index{id (aemter.models.Unterbereich attribute)@\spxentry{id}\spxextra{aemter.models.Unterbereich attribute}}

\begin{fulllineitems}
\phantomsection\label{\detokenize{masterCodeDoc:aemter.models.Unterbereich.id}}\pysigline{\sphinxbfcode{\sphinxupquote{id}}}
A wrapper for a deferred\sphinxhyphen{}loading field. When the value is read from this
object the first time, the query is executed.

\end{fulllineitems}

\index{objects (aemter.models.Unterbereich attribute)@\spxentry{objects}\spxextra{aemter.models.Unterbereich attribute}}

\begin{fulllineitems}
\phantomsection\label{\detokenize{masterCodeDoc:aemter.models.Unterbereich.objects}}\pysigline{\sphinxbfcode{\sphinxupquote{objects}}\sphinxbfcode{\sphinxupquote{ = \textless{}django.db.models.manager.Manager object\textgreater{}}}}
\end{fulllineitems}

\index{organisationseinheit (aemter.models.Unterbereich attribute)@\spxentry{organisationseinheit}\spxextra{aemter.models.Unterbereich attribute}}

\begin{fulllineitems}
\phantomsection\label{\detokenize{masterCodeDoc:aemter.models.Unterbereich.organisationseinheit}}\pysigline{\sphinxbfcode{\sphinxupquote{organisationseinheit}}}
Accessor to the related object on the forward side of a many\sphinxhyphen{}to\sphinxhyphen{}one or
one\sphinxhyphen{}to\sphinxhyphen{}one (via ForwardOneToOneDescriptor subclass) relation.

In the example:

\begin{sphinxVerbatim}[commandchars=\\\{\}]
\PYG{k}{class} \PYG{n+nc}{Child}\PYG{p}{(}\PYG{n}{Model}\PYG{p}{)}\PYG{p}{:}
    \PYG{n}{parent} \PYG{o}{=} \PYG{n}{ForeignKey}\PYG{p}{(}\PYG{n}{Parent}\PYG{p}{,} \PYG{n}{related\PYGZus{}name}\PYG{o}{=}\PYG{l+s+s1}{\PYGZsq{}}\PYG{l+s+s1}{children}\PYG{l+s+s1}{\PYGZsq{}}\PYG{p}{)}
\end{sphinxVerbatim}

\sphinxcode{\sphinxupquote{Child.parent}} is a \sphinxcode{\sphinxupquote{ForwardManyToOneDescriptor}} instance.

\end{fulllineitems}

\index{organisationseinheit\_id (aemter.models.Unterbereich attribute)@\spxentry{organisationseinheit\_id}\spxextra{aemter.models.Unterbereich attribute}}

\begin{fulllineitems}
\phantomsection\label{\detokenize{masterCodeDoc:aemter.models.Unterbereich.organisationseinheit_id}}\pysigline{\sphinxbfcode{\sphinxupquote{organisationseinheit\_id}}}
\end{fulllineitems}

\index{save\_without\_historical\_record() (aemter.models.Unterbereich method)@\spxentry{save\_without\_historical\_record()}\spxextra{aemter.models.Unterbereich method}}

\begin{fulllineitems}
\phantomsection\label{\detokenize{masterCodeDoc:aemter.models.Unterbereich.save_without_historical_record}}\pysiglinewithargsret{\sphinxbfcode{\sphinxupquote{save\_without\_historical\_record}}}{\emph{\DUrole{o}{*}\DUrole{n}{args}}, \emph{\DUrole{o}{**}\DUrole{n}{kwargs}}}{}
Save model without saving a historical record

Make sure you know what you’re doing before you use this method.

\end{fulllineitems}


\end{fulllineitems}



\subsection{Views}
\label{\detokenize{masterCodeDoc:id1}}\phantomsection\label{\detokenize{masterCodeDoc:module-aemter.views}}\index{module@\spxentry{module}!aemter.views@\spxentry{aemter.views}}\index{aemter.views@\spxentry{aemter.views}!module@\spxentry{module}}\index{main\_screen() (in module aemter.views)@\spxentry{main\_screen()}\spxextra{in module aemter.views}}

\begin{fulllineitems}
\phantomsection\label{\detokenize{masterCodeDoc:aemter.views.main_screen}}\pysiglinewithargsret{\sphinxcode{\sphinxupquote{aemter.views.}}\sphinxbfcode{\sphinxupquote{main\_screen}}}{\emph{\DUrole{n}{request}}}{}
Displays the Funktionen\sphinxhyphen{}screen

\end{fulllineitems}



\subsection{Templates}
\label{\detokenize{masterCodeDoc:id2}}
Alle Templates sind unter \sphinxtitleref{aemter/templates/aemter} zu finden.


\subsubsection{main\_screen.html}
\label{\detokenize{masterCodeDoc:main-screen-html}}

\subsubsection{department\_row.html}
\label{\detokenize{masterCodeDoc:department-row-html}}

\section{Historie}
\label{\detokenize{masterCodeDoc:historie}}
Die Historie zeichnet sämtliche Änderungen (hinzufügen, bearbeiten, löschen) an den folgenden Models des Systems auf:
\begin{itemize}
\item {} 
django.contrib.auth.models.User (Systemnutzer)

\item {} 
aemter.* (alle Models der App aemter)

\item {} 
mitglieder.* (alle Models der App mitglieder)

\end{itemize}

Diese Einträge können von Administratoren des Systems auf der entsprechenden Seite unter ./admin/historie eingesehen werden.


\subsection{Abhängigkeiten}
\label{\detokenize{masterCodeDoc:abhangigkeiten}}

\subsubsection{django\sphinxhyphen{}simple\sphinxhyphen{}history}
\label{\detokenize{masterCodeDoc:django-simple-history}}\begin{itemize}
\item {} 
Installation: \sphinxcode{\sphinxupquote{pip install django\sphinxhyphen{}simple\sphinxhyphen{}history}}

\item {} 
Dokumentation: \sphinxurl{https://django-simple-history.readthedocs.io/en/latest/}

\end{itemize}

django\sphinxhyphen{}simple\sphinxhyphen{}history ermöglicht das automatische Aufzeichnen des Zustands eines Models beim Ausführen einer Änderungsoperation (hinzufügen, bearbeiten, löschen).


\subsection{Views}
\label{\detokenize{masterCodeDoc:id3}}\phantomsection\label{\detokenize{masterCodeDoc:module-historie.views}}\index{module@\spxentry{module}!historie.views@\spxentry{historie.views}}\index{historie.views@\spxentry{historie.views}!module@\spxentry{module}}\index{fetch\_entries() (in module historie.views)@\spxentry{fetch\_entries()}\spxextra{in module historie.views}}

\begin{fulllineitems}
\phantomsection\label{\detokenize{masterCodeDoc:historie.views.fetch_entries}}\pysiglinewithargsret{\sphinxcode{\sphinxupquote{historie.views.}}\sphinxbfcode{\sphinxupquote{fetch\_entries}}}{\emph{\DUrole{n}{request}}}{}
Mit \sphinxtitleref{fetch\_entries} kann eine Liste von Historien\sphinxhyphen{}Einträgen mitsamt passender Pagination angefordert werden, welche die Einträge enthält, die…
\begin{itemize}
\item {} 
…zum angegeben Tab bzw. Model gehören.

\item {} 
…in denen die angegebenen Suchbegriffe vorkommen.

\item {} 
…zur angeforderten Seite gehören.

\end{itemize}

Folgende Aufgaben werden durch diese übernommen:
\begin{itemize}
\item {} 
Zugriffsbeschränkung: Zugriff wird nur gewährt, wenn der Nutzer angemeldet UND Administrator ist.

\item {} 
Bereitstellung von Daten: Die View stellt die gewünschte Seite der Historien\sphinxhyphen{}Einträge bereit, welche für das gewünschte Model zu den angegebenen Suchbegriffen gefunden wurden.

\item {} 
Rendern der Liste mit den passenden Historien\sphinxhyphen{}Einträgen und der zugehörigen Pagination.

\end{itemize}

Je nachdem, ob in der \sphinxtitleref{request} Suchbegriffe mitgegeben wurden, werden entweder alle Einträge oder die nach den Suchbegriffen
gefilterten Einträge bereitgestellt. Die Filterung funktioniert dabei folgendermaßen:
\begin{itemize}
\item {} 
Das gewünschte Model wird dahingehend untersucht, ob die wichtigsten Felder eines Eintrags (bei Mitgliedern z.B. ID, Vorname und Name) die Suchbegriffe enthalten.

\item {} 
Hierfür werden die in Django integrierten \sphinxtitleref{Q Objects} verwendet.

\item {} 
Alle gefundenen Einträge werden in einem QuerySet zusammengefasst, welches anschließend an \sphinxcode{\sphinxupquote{render}} übergeben wird.

\end{itemize}
\begin{quote}\begin{description}
\item[{Parameters}] \leavevmode
\sphinxstyleliteralstrong{\sphinxupquote{request}} \textendash{} Die HTML\sphinxhyphen{}Request, welche den Aufruf der View ausgelöst hat. 
Enthält stets die gewünschte Seitenzahl, den Namen des Tabs und damit Models, zu dem das Ergebnis geliefert werden soll und optional Suchbegriffe, nach denen das Model durchsucht werden soll.

\item[{Returns}] \leavevmode
Die gerenderte Liste mit den entsprechenden Historien\sphinxhyphen{}Einträgen und der zugehörigen Pagination.

\end{description}\end{quote}

\end{fulllineitems}

\index{list() (in module historie.views)@\spxentry{list()}\spxextra{in module historie.views}}

\begin{fulllineitems}
\phantomsection\label{\detokenize{masterCodeDoc:historie.views.list}}\pysiglinewithargsret{\sphinxcode{\sphinxupquote{historie.views.}}\sphinxbfcode{\sphinxupquote{list}}}{\emph{\DUrole{n}{request}}}{}
Die \sphinxtitleref{list}\sphinxhyphen{}View wird aufgerufen, wenn der Nutzer über einen Link erstmalig die Historie aufruft (z.B. aus dem Menü heraus).

Folgende Aufgaben werden von dieser übernommen:
\begin{itemize}
\item {} 
Bereitstellung von Daten: Es werden alle Historien\sphinxhyphen{}Einträge für alle Tabs geholt, anschließend in Seiten à 15 Elemente aufgeteilt und jeweils die erste Seite an die View übergeben.

\item {} 
Zugriffsbeschränkung: Zugriff wird nur gewährt, wenn der Nutzer angemeldet UND Administrator ist.

\item {} 
Rendern des Templates der gesamten Seite.

\end{itemize}
\begin{quote}\begin{description}
\item[{Parameters}] \leavevmode
\sphinxstyleliteralstrong{\sphinxupquote{request}} \textendash{} Die HTML\sphinxhyphen{}Request, welche den Aufruf der View ausgelöst hat.

\item[{Returns}] \leavevmode
Die gerenderte View.

\end{description}\end{quote}

\end{fulllineitems}



\subsection{Templates}
\label{\detokenize{masterCodeDoc:id4}}
Alle Templates sind unter \sphinxtitleref{historie/templates/historie} zu finden.


\subsubsection{list.html}
\label{\detokenize{masterCodeDoc:list-html}}
Enthält den Grundaufbau der Historie. Die Historie wird hier in die 3 Tabs “Mitglieder”, “Ämter” und “Nutzer” unterteilt.


\subsubsection{tabs/*}
\label{\detokenize{masterCodeDoc:tabs}}
Unterteilt die drei Tabs “Mitglieder”, “Ämter” und “Nutzer” ggf. in weitere Tabs, z.B. bei Mitglieder in “Stammdaten”, “E\sphinxhyphen{}Mail\sphinxhyphen{}Adressen” und “Ämter”.
Für jedes Model wird für jeden Historien\sphinxhyphen{}Eintrag im entsprechenden (Unter\sphinxhyphen{})Tab eine neue Listenzeile samt Modal generiert.


\subsubsection{tabs/\_pagination.html}
\label{\detokenize{masterCodeDoc:tabs-pagination-html}}
Enthält das Template für die Pagination (die Unterteilung der Historien\sphinxhyphen{}Einträge in Seiten).


\subsubsection{row.html}
\label{\detokenize{masterCodeDoc:row-html}}
\begin{DUlineblock}{0em}
\item[] Je nachdem, zu welchem Model der Eintrag gehört, werden hier die zum Eintrag gehörenden zusätzlichen Daten inkludiert.
\end{DUlineblock}

\begin{DUlineblock}{0em}
\item[] \sphinxtitleref{Beispiel:} Im Model MitgliedMails werden nur die IDs der Mitglieder gespeichert. Damit die Historie aber lesbarer und einfacher verständlich ist,
werden z.B. Vor\sphinxhyphen{} und Nachname des zugehörigen Mitglieds ermittelt, einmal zum aktuellen und einmal zum Zeitpunkt der Erstellung des Historien\sphinxhyphen{}Eintrags.
\end{DUlineblock}

\begin{DUlineblock}{0em}
\item[] Falls der Historien\sphinxhyphen{}Eintrag zur Änderung eines Datensatzes gehört, werden außerdem die zusätzlichen Daten zum Vorher\sphinxhyphen{}Datensatz inkludiert.
\end{DUlineblock}


\subsubsection{rowContent.html}
\label{\detokenize{masterCodeDoc:rowcontent-html}}
Der eigentliche Inhalt eines Datensatzes. Hier wird der Grundaufbau der Zeile in der angezeigten Liste von Einträgen sowie des zugehörigen Modals beschrieben.


\subsubsection{\_titleBuilder.html}
\label{\detokenize{masterCodeDoc:titlebuilder-html}}
Beschreibt, wie der Titel, welcher in der Liste von Einträgen sowie auf dem Modal angezeigt wird, zusammengebaut wird.


\subsubsection{\_noResultsRow.html}
\label{\detokenize{masterCodeDoc:noresultsrow-html}}
Wird inkludiert, falls zu einer Suchanfrage bzw. zu einem Model keine Historien\sphinxhyphen{}Einträge vorhanden sind.


\subsubsection{\_modalDataIncludes.html}
\label{\detokenize{masterCodeDoc:modaldataincludes-html}}
Je nachdem, zu welchem Model der Eintrag gehört, wird hier der entsprechende Inhalt des zugehörigen Modals inkludiert.


\subsubsection{modalData/*}
\label{\detokenize{masterCodeDoc:modaldata}}
Für jedes Model wird hier beschrieben, welche Daten wie im zugehörigen Modal präsentiert werden sollen. Falls der Historien\sphinxhyphen{}Eintrag zur Änderung eines bestehenden Datensatzes
gehört, wird ebenfalls der Datensatz vor der Änderung mitsamt zusätzlichen Daten angezeigt.


\subsection{Template Tags}
\label{\detokenize{masterCodeDoc:module-historie.templatetags.t_historie.to_class_name}}\label{\detokenize{masterCodeDoc:template-tags}}\index{module@\spxentry{module}!historie.templatetags.t\_historie.to\_class\_name@\spxentry{historie.templatetags.t\_historie.to\_class\_name}}\index{historie.templatetags.t\_historie.to\_class\_name@\spxentry{historie.templatetags.t\_historie.to\_class\_name}!module@\spxentry{module}}\index{to\_class\_name() (in module historie.templatetags.t\_historie.to\_class\_name)@\spxentry{to\_class\_name()}\spxextra{in module historie.templatetags.t\_historie.to\_class\_name}}

\begin{fulllineitems}
\phantomsection\label{\detokenize{masterCodeDoc:historie.templatetags.t_historie.to_class_name.to_class_name}}\pysiglinewithargsret{\sphinxcode{\sphinxupquote{historie.templatetags.t\_historie.to\_class\_name.}}\sphinxbfcode{\sphinxupquote{to\_class\_name}}}{\emph{\DUrole{n}{value}}}{}
Gibt den Namen der Klasse des übergebenen Objekts zurück.
\begin{quote}\begin{description}
\item[{Parameters}] \leavevmode
\sphinxstyleliteralstrong{\sphinxupquote{value}} (\sphinxstyleliteralemphasis{\sphinxupquote{any}}) \textendash{} Das Objekt, von dem die Klasse ermittelt werden soll.

\item[{Returns}] \leavevmode
Den Namen der Klasse des Objekts.

\item[{Return type}] \leavevmode
str

\end{description}\end{quote}

\end{fulllineitems}

\phantomsection\label{\detokenize{masterCodeDoc:module-historie.templatetags.t_historie.get_associated_data}}\index{module@\spxentry{module}!historie.templatetags.t\_historie.get\_associated\_data@\spxentry{historie.templatetags.t\_historie.get\_associated\_data}}\index{historie.templatetags.t\_historie.get\_associated\_data@\spxentry{historie.templatetags.t\_historie.get\_associated\_data}!module@\spxentry{module}}\index{get\_associated\_data() (in module historie.templatetags.t\_historie.get\_associated\_data)@\spxentry{get\_associated\_data()}\spxextra{in module historie.templatetags.t\_historie.get\_associated\_data}}

\begin{fulllineitems}
\phantomsection\label{\detokenize{masterCodeDoc:historie.templatetags.t_historie.get_associated_data.get_associated_data}}\pysiglinewithargsret{\sphinxcode{\sphinxupquote{historie.templatetags.t\_historie.get\_associated\_data.}}\sphinxbfcode{\sphinxupquote{get\_associated\_data}}}{\emph{\DUrole{n}{desiredInfo}}, \emph{\DUrole{n}{queryType}}, \emph{\DUrole{n}{primaryKey}}, \emph{\DUrole{n}{timestamp}}}{}
Ermittelt zu einem gegebenen Historien\sphinxhyphen{}Eintrag gehörige, zusätzliche Daten.

Beispiel: Ein Historien\sphinxhyphen{}Eintrag zum Model mitglied.MitgliedMail enhält nur die ID des entsprechenden Mitglieds.
Mit \sphinxcode{\sphinxupquote{get\_associated\_data}} können sämtliche Daten des zur ID gehörenden Mitglieds (z.B. Name oder Anschrift) ermittelt werden,
sowohl zum jetzigen Zeitpunkt als auch zu dem Zeitpunkt, zu dem der Historien\sphinxhyphen{}Eintrag angelegt wurde.
\begin{quote}\begin{description}
\item[{Parameters}] \leavevmode\begin{itemize}
\item {} 
\sphinxstyleliteralstrong{\sphinxupquote{desiredInfo}} (\sphinxstyleliteralemphasis{\sphinxupquote{str}}) \textendash{} Der Name des Models, aus welchem die zusätzlichen Daten ermittelt werden sollen.
Zulässige Werte: “Mitglied”, “Funktion”, “Unterbereich”, “Organisationseinheit”, “Recht”

\item {} 
\sphinxstyleliteralstrong{\sphinxupquote{queryType}} (\sphinxstyleliteralemphasis{\sphinxupquote{str}}) \textendash{} Gibt an, ob die aktuellen Daten oder die Daten zum Zeitpunkt des Eintrags in die Historie ermittelt werden sollen.
Zulässige Werte: “latest”, “historical”

\item {} 
\sphinxstyleliteralstrong{\sphinxupquote{primaryKey}} (\sphinxstyleliteralemphasis{\sphinxupquote{int}}) \textendash{} Die ID des betroffenen Datensatzes, für welchen die Daten ermittelt werden sollen; z.B. die ID des Mitglieds.

\item {} 
\sphinxstyleliteralstrong{\sphinxupquote{timestamp}} (\sphinxstyleliteralemphasis{\sphinxupquote{datetime}}\sphinxstyleliteralemphasis{\sphinxupquote{, }}\sphinxstyleliteralemphasis{\sphinxupquote{"semi\sphinxhyphen{}optional"}}) \textendash{} Der Zeitstempel, zu welchem der Historien\sphinxhyphen{}Eintrag angelegt wurde. Wird nur benötigt, falls die Daten zum Zeitpunkt
des Erstellens des Historien\sphinxhyphen{}Eintrags ermittelt werden sollen.

\end{itemize}

\item[{Returns}] \leavevmode
Eine Instanz des mittels \sphinxtitleref{desiredInfo} angegebenen Models, welche sämtliche Daten des Objekts mit der ID \sphinxtitleref{primaryKey} zum mittels
\sphinxtitleref{queryType} und ggf. \sphinxtitleref{timestamp} angegebenen Zeitpunkt enthält.

\end{description}\end{quote}

\end{fulllineitems}



\section{Mitglieder}
\label{\detokenize{masterCodeDoc:mitglieder}}
Die Django\sphinxhyphen{}Applikation “mitglieder” dient zur Verwaltung (Anzeigen, Erstellen, Löschen, Bearbeiten) von Mitgliedern des StuRas. Nutzer, bei denen es sich nicht um Admins handelt, sind dabei nur zum Einsehen der Mitgliederdaten befugt.

Zu jedem Mitglied werden folgende Attribute gespeichert:
\begin{itemize}
\item {} 
Vorname

\item {} 
Nachname

\item {} 
Spitzname (optional)

\item {} 
Ämter (optional)

\item {} 
E\sphinxhyphen{}Mail\sphinxhyphen{}Adresse(n) (optional)

\item {} 
Anschrift (optional)

\item {} 
Telefonnummer(n) (optional)

\end{itemize}


\subsection{Abhängigkeiten}
\label{\detokenize{masterCodeDoc:id5}}

\subsubsection{simplejson}
\label{\detokenize{masterCodeDoc:simplejson}}\begin{itemize}
\item {} 
Installation: \sphinxcode{\sphinxupquote{pip install simplejson}}

\item {} 
Dokumentation: \sphinxurl{https://simplejson.readthedocs.io/en/latest/}

\end{itemize}

simplejson ist ein En\sphinxhyphen{} und Decoder für JSON.


\subsection{Views}
\label{\detokenize{masterCodeDoc:id6}}\phantomsection\label{\detokenize{masterCodeDoc:module-mitglieder.views}}\index{module@\spxentry{module}!mitglieder.views@\spxentry{mitglieder.views}}\index{mitglieder.views@\spxentry{mitglieder.views}!module@\spxentry{module}}\index{bereiche\_laden() (in module mitglieder.views)@\spxentry{bereiche\_laden()}\spxextra{in module mitglieder.views}}

\begin{fulllineitems}
\phantomsection\label{\detokenize{masterCodeDoc:mitglieder.views.bereiche_laden}}\pysiglinewithargsret{\sphinxcode{\sphinxupquote{mitglieder.views.}}\sphinxbfcode{\sphinxupquote{bereiche\_laden}}}{\emph{\DUrole{n}{request}}}{}
Rendert ein Dropdown mit allen Bereichen eines bestimmten Referats beim dazugehörigen Amt, nachdem ein Referat bei der Mitgliedererstellung oder \sphinxhyphen{}bearbeitung ausgewählt wurde.

Aufgaben:
\begin{itemize}
\item {} 
Bereitstellung der Daten: Alle Bereiche eines Referats werden aus der Datenbank entnommen.

\item {} 
Rendern des Templates

\item {} 
Rechteeinschränkung: Nur angemeldete Nutzer können den Vorgang auslösen

\end{itemize}
\begin{quote}\begin{description}
\item[{Parameters}] \leavevmode
\sphinxstyleliteralstrong{\sphinxupquote{request}} \textendash{} Die Ajax\sphinxhyphen{}Request, welche den Aufruf der Funktion ausgelöst hat. Enthält den Namen des ausgewählten Referats sowie die Nummer des Amts eines Mitglieds.

\item[{Returns}] \leavevmode
Das gerenderte Dropdown.

\end{description}\end{quote}

\end{fulllineitems}

\index{email\_html\_laden() (in module mitglieder.views)@\spxentry{email\_html\_laden()}\spxextra{in module mitglieder.views}}

\begin{fulllineitems}
\phantomsection\label{\detokenize{masterCodeDoc:mitglieder.views.email_html_laden}}\pysiglinewithargsret{\sphinxcode{\sphinxupquote{mitglieder.views.}}\sphinxbfcode{\sphinxupquote{email\_html\_laden}}}{\emph{\DUrole{n}{request}}}{}
Rendert ein Formular für eine weitere E\sphinxhyphen{}Mail, nachdem diese angefordert wurde und inkrementiert die Anzahl der Formulare für eine E\sphinxhyphen{}Mail in der View.

Aufgaben:
\begin{itemize}
\item {} 
Rendern des Formulars

\item {} 
Erfassen der Anzahl der E\sphinxhyphen{}Mails eines Mitglieds

\item {} 
Rechteeinschränkung: Nur angemeldete Nutzer können den Vorgang auslösen

\end{itemize}
\begin{quote}\begin{description}
\item[{Parameters}] \leavevmode
\sphinxstyleliteralstrong{\sphinxupquote{request}} \textendash{} Die Ajax\sphinxhyphen{}Request, welche den Aufruf der Funktion ausgelöst hat.

\item[{Returns}] \leavevmode
HTTP Response

\end{description}\end{quote}

\end{fulllineitems}

\index{email\_loeschen() (in module mitglieder.views)@\spxentry{email\_loeschen()}\spxextra{in module mitglieder.views}}

\begin{fulllineitems}
\phantomsection\label{\detokenize{masterCodeDoc:mitglieder.views.email_loeschen}}\pysiglinewithargsret{\sphinxcode{\sphinxupquote{mitglieder.views.}}\sphinxbfcode{\sphinxupquote{email\_loeschen}}}{\emph{\DUrole{n}{request}}}{}
Dekrementiert die Anzahl der Formulare für eine E\sphinxhyphen{}Mail in der mitgliedBearbeitenView oder mitgliedErstellenView nach Löschen eines Formulars.

Aufgaben:
\begin{itemize}
\item {} 
Erfassen der Anzahl der E\sphinxhyphen{}Mails

\item {} 
Rechteeinschränkung: Nur angemeldete Nutzer können den Vorgang auslösen

\end{itemize}
\begin{quote}\begin{description}
\item[{Parameters}] \leavevmode
\sphinxstyleliteralstrong{\sphinxupquote{request}} \textendash{} Die Ajax\sphinxhyphen{}Request, welche den Aufruf der Funktion ausgelöst hat.

\item[{Returns}] \leavevmode
HTTP Response

\end{description}\end{quote}

\end{fulllineitems}

\index{erstellen() (in module mitglieder.views)@\spxentry{erstellen()}\spxextra{in module mitglieder.views}}

\begin{fulllineitems}
\phantomsection\label{\detokenize{masterCodeDoc:mitglieder.views.erstellen}}\pysiglinewithargsret{\sphinxcode{\sphinxupquote{mitglieder.views.}}\sphinxbfcode{\sphinxupquote{erstellen}}}{\emph{\DUrole{n}{request}}}{}
Speichert ein neues Mitglied in der Datenbank.

Aufgaben:
\begin{itemize}
\item {} 
Speichern der Daten: Die Daten werden aus request gelesen und in die Datenbank eingefügt.

\item {} 
Weiterleitung zur Mitgliederanischt.

\item {} 
Rechteeinschränkung: Nur Admins können die Funktion auslösen.

\end{itemize}
\begin{quote}\begin{description}
\item[{Parameters}] \leavevmode
\sphinxstyleliteralstrong{\sphinxupquote{request}} \textendash{} Die POST\sphinxhyphen{}Request, welche den Aufruf der Funktion ausgelöst hat. Enthält alle Daten zu einem Mitglied.

\item[{Returns}] \leavevmode
Weiterleitung zur Mitgliederansicht.

\end{description}\end{quote}

\end{fulllineitems}

\index{funktion\_loeschen() (in module mitglieder.views)@\spxentry{funktion\_loeschen()}\spxextra{in module mitglieder.views}}

\begin{fulllineitems}
\phantomsection\label{\detokenize{masterCodeDoc:mitglieder.views.funktion_loeschen}}\pysiglinewithargsret{\sphinxcode{\sphinxupquote{mitglieder.views.}}\sphinxbfcode{\sphinxupquote{funktion\_loeschen}}}{\emph{\DUrole{n}{request}}}{}
Dekrementiert die Anzahl der Formulare für ein Amt in der mitgliedBearbeitenView oder mitgliedErstellenView nach Löschen eines Formulars.

Aufgaben:
\begin{itemize}
\item {} 
Erfassen der Anzahl der Ämter

\item {} 
Rechteeinschränkung: Nur angemeldete Nutzer können den Vorgang auslösen

\end{itemize}
\begin{quote}\begin{description}
\item[{Parameters}] \leavevmode
\sphinxstyleliteralstrong{\sphinxupquote{request}} \textendash{} Die Ajax\sphinxhyphen{}Request, welche den Aufruf der Funktion ausgelöst hat.

\item[{Returns}] \leavevmode
HTTP Response

\end{description}\end{quote}

\end{fulllineitems}

\index{funktionen\_html\_laden() (in module mitglieder.views)@\spxentry{funktionen\_html\_laden()}\spxextra{in module mitglieder.views}}

\begin{fulllineitems}
\phantomsection\label{\detokenize{masterCodeDoc:mitglieder.views.funktionen_html_laden}}\pysiglinewithargsret{\sphinxcode{\sphinxupquote{mitglieder.views.}}\sphinxbfcode{\sphinxupquote{funktionen\_html\_laden}}}{\emph{\DUrole{n}{request}}}{}
Rendert ein Formular für ein weiteres Amt, nachdem dieses angefordert wurde und inkrementiert die Anzahl der Formulare für ein Amt in der View.

Aufgaben:
\begin{itemize}
\item {} 
Bereitstellung der Daten: Alle Referate werden aus der Datenbank entnommen.

\item {} 
Rendern des Templates

\item {} 
Rechteeinschränkung: Nur angemeldete Nutzer können den Vorgang auslösen

\end{itemize}
\begin{quote}\begin{description}
\item[{Parameters}] \leavevmode
\sphinxstyleliteralstrong{\sphinxupquote{request}} \textendash{} Die Ajax\sphinxhyphen{}Request, welche den Aufruf der Funktion ausgelöst hat.

\item[{Returns}] \leavevmode
Das gerenderte Formular.

\end{description}\end{quote}

\end{fulllineitems}

\index{funktionen\_laden() (in module mitglieder.views)@\spxentry{funktionen\_laden()}\spxextra{in module mitglieder.views}}

\begin{fulllineitems}
\phantomsection\label{\detokenize{masterCodeDoc:mitglieder.views.funktionen_laden}}\pysiglinewithargsret{\sphinxcode{\sphinxupquote{mitglieder.views.}}\sphinxbfcode{\sphinxupquote{funktionen\_laden}}}{\emph{\DUrole{n}{request}}}{}
Rendert ein Dropdown mit allen Ämtern eines bestimmten Bereich beim dazugehörigen Amt, nachdem ein Bereich bei der Mitgliedererstellung oder \sphinxhyphen{}bearbeitung ausgewählt wurde.

Aufgaben:
\begin{itemize}
\item {} 
Bereitstellung der Daten: Alle Ämter eines Bereichs werden aus der Datenbank entnommen.

\item {} 
Rendern des Templates

\item {} 
Rechteeinschränkung: Nur angemeldete Nutzer können den Vorgang auslösen

\end{itemize}
\begin{quote}\begin{description}
\item[{Parameters}] \leavevmode
\sphinxstyleliteralstrong{\sphinxupquote{request}} \textendash{} Die Ajax\sphinxhyphen{}Request, welche den Aufruf der Funktion ausgelöst hat. Enthält den Namen des ausgewählten Bereichs sowie die dazugehörige Nummer des Amts eines Mitglieds.

\item[{Returns}] \leavevmode
Das gerenderte Dropdown.

\end{description}\end{quote}

\end{fulllineitems}

\index{funktionen\_max\_member\_ueberpruefen() (in module mitglieder.views)@\spxentry{funktionen\_max\_member\_ueberpruefen()}\spxextra{in module mitglieder.views}}

\begin{fulllineitems}
\phantomsection\label{\detokenize{masterCodeDoc:mitglieder.views.funktionen_max_member_ueberpruefen}}\pysiglinewithargsret{\sphinxcode{\sphinxupquote{mitglieder.views.}}\sphinxbfcode{\sphinxupquote{funktionen\_max\_member\_ueberpruefen}}}{\emph{\DUrole{n}{request}}, \emph{\DUrole{n}{mitglied\_id}}}{}
\end{fulllineitems}

\index{main\_screen() (in module mitglieder.views)@\spxentry{main\_screen()}\spxextra{in module mitglieder.views}}

\begin{fulllineitems}
\phantomsection\label{\detokenize{masterCodeDoc:mitglieder.views.main_screen}}\pysiglinewithargsret{\sphinxcode{\sphinxupquote{mitglieder.views.}}\sphinxbfcode{\sphinxupquote{main\_screen}}}{\emph{\DUrole{n}{request}}}{}
Zeigt eine Tabelle mit Migliedern an und ermöglicht die Suche nach Mitgliedern mit bestimmten Namen.
Admins wird zusätzlich das Löschen von einem oder mehreren Mitgliedern sowie das Wechseln zur View zum Erstellen oder zum Bearbeiten ermöglicht.

Aufgaben:
\begin{itemize}
\item {} 
Bereitstellung der Daten: Die View holt sämtliche Mitglieder\sphinxhyphen{}Einträge aus der Datenbank und stellt diese als Kontext bereit.

\item {} 
Rendern des Templates

\item {} 
Rechteeinschränkung: Nur Admins können Mitglieder erstellen, bearbeiten und löschen.

\end{itemize}
\begin{quote}\begin{description}
\item[{Parameters}] \leavevmode
\sphinxstyleliteralstrong{\sphinxupquote{request}} \textendash{} Die HTML\sphinxhyphen{}Request, welche den Aufruf der View ausgelöst hat.

\item[{Returns}] \leavevmode
Die gerenderte View.

\end{description}\end{quote}

\end{fulllineitems}

\index{mitgliedBearbeitenView() (in module mitglieder.views)@\spxentry{mitgliedBearbeitenView()}\spxextra{in module mitglieder.views}}

\begin{fulllineitems}
\phantomsection\label{\detokenize{masterCodeDoc:mitglieder.views.mitgliedBearbeitenView}}\pysiglinewithargsret{\sphinxcode{\sphinxupquote{mitglieder.views.}}\sphinxbfcode{\sphinxupquote{mitgliedBearbeitenView}}}{\emph{\DUrole{n}{request}}, \emph{\DUrole{n}{mitglied\_id}}}{}
View zum Bearbeiten eines Mitglieds.

Stellt Textfelder, Dropwdowns und Buttons zum Bearbeiten der Attribute bereit, welche mit derzeitigen Attributen des Mitglieds befüllt sind. Über Buttons können weitere Ämter und E\sphinxhyphen{}Mail\sphinxhyphen{}Adressen hinzugefügt oder bereits bestehende entfernt werden.

Mit Betätigung des Speichern\sphinxhyphen{}Buttons wird überprüft, ob Name, Vorname, Ämter und E\sphinxhyphen{}Mail\sphinxhyphen{}Adressen ausgefüllt wurden und ob alle E\sphinxhyphen{}Mail\sphinxhyphen{}Adressen gültig sind. Bei erfolgreicher Prüfung wird das Mitglied gespeichert und der
Nutzer zu main\_screen umgeleitet, ansonsten werden Felder mit fehlenden oder fehlerhaften Eingaben rot markiert.

Aufgaben:
\begin{itemize}
\item {} 
Zugriffsbeschränkung: Zugriff wird nur gewährt, wenn der Nutzer angemeldet UND Administrator ist.

\item {} 
Bereitstellung der Daten: Die View holt Attribute eines Mitglieds aus der Datenbank und zeigt diese an.

\item {} 
Rendern des Templates

\item {} 
Speichern des Mitglieds in der Datenbank

\end{itemize}
\begin{quote}\begin{description}
\item[{Parameters}] \leavevmode\begin{itemize}
\item {} 
\sphinxstyleliteralstrong{\sphinxupquote{request}} \textendash{} Die HTML\sphinxhyphen{}Request, welche den Aufruf der View ausgelöst hat.

\item {} 
\sphinxstyleliteralstrong{\sphinxupquote{mitglied\_id}} \textendash{} Id des Mitglieds, das bearbeitet werden soll

\end{itemize}

\item[{Returns}] \leavevmode
Die gerenderte View.

\end{description}\end{quote}

\end{fulllineitems}

\index{mitgliedErstellenView() (in module mitglieder.views)@\spxentry{mitgliedErstellenView()}\spxextra{in module mitglieder.views}}

\begin{fulllineitems}
\phantomsection\label{\detokenize{masterCodeDoc:mitglieder.views.mitgliedErstellenView}}\pysiglinewithargsret{\sphinxcode{\sphinxupquote{mitglieder.views.}}\sphinxbfcode{\sphinxupquote{mitgliedErstellenView}}}{\emph{\DUrole{n}{request}}}{}
View zum Erstellen eines Mitglieds.

Stellt Textfelder, Dropwdowns und Buttons zum Hinzufügen der Attribute bereit. Anfangs steht jeweils genau ein Eingabebereich für ein Amt und eine E\sphinxhyphen{}Mail\sphinxhyphen{}Adresse zur Verfügung. Über Buttons können weitere dieser hinzugefügt oder bereits bestehende entfernt werden.

Mit Betätigung des Speichern\sphinxhyphen{}Buttons wird überprüft, ob Name, Vorname, Ämter und E\sphinxhyphen{}Mail\sphinxhyphen{}Adressen ausgefüllt wurden und ob alle E\sphinxhyphen{}Mail\sphinxhyphen{}Adressen gültig sind. Bei erfolgreicher Prüfung wird das Mitglied gespeichert und der
Nutzer zu main\_screen umgeleitet, ansonsten werden Felder mit fehlenden oder fehlerhaften Eingaben rot markiert.

Aufgaben:
\begin{itemize}
\item {} 
Zugriffsbeschränkung: Zugriff wird nur gewährt, wenn der Nutzer angemeldet UND Administrator ist.

\item {} 
Rendern des Templates

\item {} 
Speichern des Mitglieds in der Datenbank

\end{itemize}
\begin{quote}\begin{description}
\item[{Parameters}] \leavevmode
\sphinxstyleliteralstrong{\sphinxupquote{request}} \textendash{} Die HTML\sphinxhyphen{}Request, welche den Aufruf der View ausgelöst hat.

\item[{Returns}] \leavevmode
Die gerenderte View.

\end{description}\end{quote}

\end{fulllineitems}

\index{mitglied\_laden() (in module mitglieder.views)@\spxentry{mitglied\_laden()}\spxextra{in module mitglieder.views}}

\begin{fulllineitems}
\phantomsection\label{\detokenize{masterCodeDoc:mitglieder.views.mitglied_laden}}\pysiglinewithargsret{\sphinxcode{\sphinxupquote{mitglieder.views.}}\sphinxbfcode{\sphinxupquote{mitglied\_laden}}}{\emph{\DUrole{n}{request}}}{}
Rendert ein Modal mit allen Daten eines aus der Tabelle gewählten Mitlieds.

Aufgaben:
\begin{itemize}
\item {} 
Bereitstellung der Daten: Die Mitglied\sphinxhyphen{}Id wird aus request gelesen und extrahieren aller Daten zum Mitglied mit dieser Id

\item {} 
Rendern des Templates

\item {} 
Rechteeinschränkung: Nur angemeldete Nutzer können das gerenderte Template anfordern.

\end{itemize}
\begin{quote}\begin{description}
\item[{Parameters}] \leavevmode
\sphinxstyleliteralstrong{\sphinxupquote{request}} \textendash{} Die Ajax\sphinxhyphen{}Request, welche den Aufruf der Funktion ausgelöst hat. Enthält die Id des Mitglieds, dessen Daten angezeigt werden sollen.

\item[{Returns}] \leavevmode
Das gerenderte Modal, das mit Daten des angeforderten Mitglieds ausgefüllt wurde

\end{description}\end{quote}

\end{fulllineitems}

\index{mitglieder\_loeschen() (in module mitglieder.views)@\spxentry{mitglieder\_loeschen()}\spxextra{in module mitglieder.views}}

\begin{fulllineitems}
\phantomsection\label{\detokenize{masterCodeDoc:mitglieder.views.mitglieder_loeschen}}\pysiglinewithargsret{\sphinxcode{\sphinxupquote{mitglieder.views.}}\sphinxbfcode{\sphinxupquote{mitglieder\_loeschen}}}{\emph{\DUrole{n}{request}}}{}
Löscht ausgewählte Mitglieder aus der Datenbank.

Aufgaben:
\begin{itemize}
\item {} 
Entfernen der Daten: Alle Daten der Mitglieder werden aus der Datenbank entfernt.

\item {} 
Rendern des Templates

\item {} 
Rechteeinschränkung: Nur angemeldete Nutzer können Löschvorgänge auslösen

\end{itemize}
\begin{quote}\begin{description}
\item[{Parameters}] \leavevmode
\sphinxstyleliteralstrong{\sphinxupquote{request}} \textendash{} Die Ajax\sphinxhyphen{}Request, welche den Aufruf der Funktion ausgelöst hat. Enthält die Ids der Mitglieder, die entfernt werden sollen

\item[{Returns}] \leavevmode
HTTP Response

\end{description}\end{quote}

\end{fulllineitems}

\index{speichern() (in module mitglieder.views)@\spxentry{speichern()}\spxextra{in module mitglieder.views}}

\begin{fulllineitems}
\phantomsection\label{\detokenize{masterCodeDoc:mitglieder.views.speichern}}\pysiglinewithargsret{\sphinxcode{\sphinxupquote{mitglieder.views.}}\sphinxbfcode{\sphinxupquote{speichern}}}{\emph{\DUrole{n}{request}}, \emph{\DUrole{n}{mitglied\_id}}}{}
Speichert ein bearbeitetes Mitglied in der Datenbank.

Aufgaben:
\begin{itemize}
\item {} 
Speichern der Daten: Die Daten werden aus request gelesen und in der Datenbank gespeichert. Ämter und E\sphinxhyphen{}Mails werden gespeichert, indem zunächst alle bereits vorhandenen Instanzen gelöscht werden
und anschließend alle Ämter und E\sphinxhyphen{}Mails aus request gespeichert werden.

\item {} 
Weiterleitung zur Mitgliederanischt.

\item {} 
Rechteeinschränkung: Nur Admins können die Funktion auslösen.

\end{itemize}
\begin{quote}\begin{description}
\item[{Parameters}] \leavevmode\begin{itemize}
\item {} 
\sphinxstyleliteralstrong{\sphinxupquote{request}} \textendash{} Die POST\sphinxhyphen{}Request, welche den Aufruf der Funktion ausgelöst hat. Enthält alle Daten zu einem Mitglied.

\item {} 
\sphinxstyleliteralstrong{\sphinxupquote{mitglied\_id}} \textendash{} Die Id des Mitglieds, das bearbeitet wurde.

\end{itemize}

\item[{Returns}] \leavevmode
Weiterleitung zur Mitgliederansicht.

\end{description}\end{quote}

\end{fulllineitems}

\index{suchen() (in module mitglieder.views)@\spxentry{suchen()}\spxextra{in module mitglieder.views}}

\begin{fulllineitems}
\phantomsection\label{\detokenize{masterCodeDoc:mitglieder.views.suchen}}\pysiglinewithargsret{\sphinxcode{\sphinxupquote{mitglieder.views.}}\sphinxbfcode{\sphinxupquote{suchen}}}{\emph{\DUrole{n}{request}}}{}
Anzeige von Mitgliedern, deren Namen auf die Sucheingabe passen.

Aufgaben:
\begin{itemize}
\item {} 
Bereitstellung der Daten: Die Sucheingabe wird in mehrere Suchbegriffe unterteilt. Bei allen Mitgliedern der Datenbank wird überprüft, ob sie mindestens einen der Suchbegriffe
im Vor\sphinxhyphen{} oder Nachnamen als Substring enthalten. Diese Mitglieder werden angezeigt und nach der Anzahl der Suchbegriffe, die auf den Vor\sphinxhyphen{} oder Nachnamen passen, sortiert.

\item {} 
Rendern des Templates

\item {} 
Rechteeinschränkung: Nur angemeldete Nutzer können die Funktion auslösen.

\end{itemize}
\begin{quote}\begin{description}
\item[{Parameters}] \leavevmode
\sphinxstyleliteralstrong{\sphinxupquote{request}} \textendash{} Die Ajax\sphinxhyphen{}Request, welche den Aufruf der Funktion ausgelöst hat. Enthält die Sucheingabe.

\item[{Returns}] \leavevmode
Das gerenderte Templates mit den gefunden Mitgliedern.

\end{description}\end{quote}

\end{fulllineitems}



\subsection{Templates}
\label{\detokenize{masterCodeDoc:id7}}
Alle Templates sind unter \sphinxtitleref{mitglieder/templates/mitglieder} zu finden.


\subsubsection{aemter.html}
\label{\detokenize{masterCodeDoc:aemter-html}}
Enthält die Felder für den Kandidaturzeitraum und Dropdowns für Attribute eines Amts. Initial wird neben den Textfeldern nur das Dropwdown für ein Referat angezeigt. Die restlichen Dropdowns werden über die Auswahl eines Referats bzw. Bereichs angezeigt.


\subsubsection{amt\_dropdown\_list\_options.html}
\label{\detokenize{masterCodeDoc:amt-dropdown-list-options-html}}
Dropdown zur Auswahl eines Amts. Die angezeigten Auswahlmöglichkeiten hängen vom gewählten Bereich ab.


\subsubsection{bereich\_dropdown\_list\_options.html}
\label{\detokenize{masterCodeDoc:bereich-dropdown-list-options-html}}
Dropdown zur Auswahl eines Bereichs. Die angezeigten Auswahlmöglichkeiten hängen vom gewählten Referat ab.


\subsubsection{email.html}
\label{\detokenize{masterCodeDoc:email-html}}
Enthält ein Textfeld zur Eingabe einer E\sphinxhyphen{}Mail\sphinxhyphen{}Adresse sowie einen Löschbutton zum Entfernen der jeweiligen E\sphinxhyphen{}Mail\sphinxhyphen{}Adresse.


\subsubsection{email\_input.html}
\label{\detokenize{masterCodeDoc:email-input-html}}
Enthält 0 bis n Kopien von \sphinxtitleref{email.html} sowie einen Button, um genau eine weitere Kopie hinzuzufügen.


\subsubsection{mitglieder.html}
\label{\detokenize{masterCodeDoc:mitglieder-html}}
Ansicht mit allen Mitgliedern, welche über eine Tabelle angezeigt werden. Über das Anklicken einer Zeile wird ein Modal mit allen Daten eines Mitglieds angezeigt.
Auch die Suche nach Mitgliedsnamen, das Löschen von Mitgliedern und das Wechseln zur \sphinxtitleref{mitgliedErstellenView} kann hier durchgeführt werden.


\subsubsection{mitglied\_erstellen\_bearbeiten.html}
\label{\detokenize{masterCodeDoc:mitglied-erstellen-bearbeiten-html}}
Ermöglicht die Eingabe aller Attribute eines bereits existierenden oder neuen Mitglieds.


\subsubsection{modal.html}
\label{\detokenize{masterCodeDoc:modal-html}}
Ein Modal, das alle Attribute eines Mitglieds mit einer bestimmten Id anzeigt.


\subsubsection{row.html}
\label{\detokenize{masterCodeDoc:id8}}
Eine Tabellenzeile, um Name, Vorname, Ämter, E\sphinxhyphen{}Mail\sphinxhyphen{}Adressen und die Telefonnummer eines Mitglieds anzuzeigen. Admins wird hier zusätzlich eine Checkbox und ein Button zum Bearbeiten des Mitglieds angezeigt.
Wird von \sphinxtitleref{mitglieder.html} inkludiert.


\section{Checklisten}
\label{\detokenize{masterCodeDoc:checklisten}}
Checklisten allow administrators to keep track of which tasks need to be completed in order to welcome a new member (Mitglied) to StuRa or a new Funktion inside of the StuRa.
A Checkliste may consist of a set of general tasks (Aufgaben) and/or, if a Funktion was selected, all permissions (Rechte) that have to be granted to the member (Mitglied).


\subsection{Models}
\label{\detokenize{masterCodeDoc:id9}}\phantomsection\label{\detokenize{masterCodeDoc:module-checklisten.models}}\index{module@\spxentry{module}!checklisten.models@\spxentry{checklisten.models}}\index{checklisten.models@\spxentry{checklisten.models}!module@\spxentry{module}}\index{Aufgabe (class in checklisten.models)@\spxentry{Aufgabe}\spxextra{class in checklisten.models}}

\begin{fulllineitems}
\phantomsection\label{\detokenize{masterCodeDoc:checklisten.models.Aufgabe}}\pysiglinewithargsret{\sphinxbfcode{\sphinxupquote{class }}\sphinxcode{\sphinxupquote{checklisten.models.}}\sphinxbfcode{\sphinxupquote{Aufgabe}}}{\emph{\DUrole{o}{*}\DUrole{n}{args}}, \emph{\DUrole{o}{**}\DUrole{n}{kwargs}}}{}
Defines a general task that can be added to a checklist.
\begin{itemize}
\item {} 
bezeichnung: The task’s title that will be shown in the UI. May contain up to 50 characters and must not be null.

\item {} 
history: Contains a log of all changes made to the model. Provided by django\sphinxhyphen{}simple\sphinxhyphen{}history.

\end{itemize}
\index{Aufgabe.DoesNotExist@\spxentry{Aufgabe.DoesNotExist}}

\begin{fulllineitems}
\phantomsection\label{\detokenize{masterCodeDoc:checklisten.models.Aufgabe.DoesNotExist}}\pysigline{\sphinxbfcode{\sphinxupquote{exception }}\sphinxbfcode{\sphinxupquote{DoesNotExist}}}
\end{fulllineitems}

\index{Aufgabe.MultipleObjectsReturned@\spxentry{Aufgabe.MultipleObjectsReturned}}

\begin{fulllineitems}
\phantomsection\label{\detokenize{masterCodeDoc:checklisten.models.Aufgabe.MultipleObjectsReturned}}\pysigline{\sphinxbfcode{\sphinxupquote{exception }}\sphinxbfcode{\sphinxupquote{MultipleObjectsReturned}}}
\end{fulllineitems}

\index{bezeichnung (checklisten.models.Aufgabe attribute)@\spxentry{bezeichnung}\spxextra{checklisten.models.Aufgabe attribute}}

\begin{fulllineitems}
\phantomsection\label{\detokenize{masterCodeDoc:checklisten.models.Aufgabe.bezeichnung}}\pysigline{\sphinxbfcode{\sphinxupquote{bezeichnung}}}
A wrapper for a deferred\sphinxhyphen{}loading field. When the value is read from this
object the first time, the query is executed.

\end{fulllineitems}

\index{checklisteaufgabe\_set (checklisten.models.Aufgabe attribute)@\spxentry{checklisteaufgabe\_set}\spxextra{checklisten.models.Aufgabe attribute}}

\begin{fulllineitems}
\phantomsection\label{\detokenize{masterCodeDoc:checklisten.models.Aufgabe.checklisteaufgabe_set}}\pysigline{\sphinxbfcode{\sphinxupquote{checklisteaufgabe\_set}}}
Accessor to the related objects manager on the reverse side of a
many\sphinxhyphen{}to\sphinxhyphen{}one relation.

In the example:

\begin{sphinxVerbatim}[commandchars=\\\{\}]
\PYG{k}{class} \PYG{n+nc}{Child}\PYG{p}{(}\PYG{n}{Model}\PYG{p}{)}\PYG{p}{:}
    \PYG{n}{parent} \PYG{o}{=} \PYG{n}{ForeignKey}\PYG{p}{(}\PYG{n}{Parent}\PYG{p}{,} \PYG{n}{related\PYGZus{}name}\PYG{o}{=}\PYG{l+s+s1}{\PYGZsq{}}\PYG{l+s+s1}{children}\PYG{l+s+s1}{\PYGZsq{}}\PYG{p}{)}
\end{sphinxVerbatim}

\sphinxcode{\sphinxupquote{Parent.children}} is a \sphinxcode{\sphinxupquote{ReverseManyToOneDescriptor}} instance.

Most of the implementation is delegated to a dynamically defined manager
class built by \sphinxcode{\sphinxupquote{create\_forward\_many\_to\_many\_manager()}} defined below.

\end{fulllineitems}

\index{history (checklisten.models.Aufgabe attribute)@\spxentry{history}\spxextra{checklisten.models.Aufgabe attribute}}

\begin{fulllineitems}
\phantomsection\label{\detokenize{masterCodeDoc:checklisten.models.Aufgabe.history}}\pysigline{\sphinxbfcode{\sphinxupquote{history}}\sphinxbfcode{\sphinxupquote{ = \textless{}simple\_history.manager.HistoryManager object\textgreater{}}}}
\end{fulllineitems}

\index{id (checklisten.models.Aufgabe attribute)@\spxentry{id}\spxextra{checklisten.models.Aufgabe attribute}}

\begin{fulllineitems}
\phantomsection\label{\detokenize{masterCodeDoc:checklisten.models.Aufgabe.id}}\pysigline{\sphinxbfcode{\sphinxupquote{id}}}
A wrapper for a deferred\sphinxhyphen{}loading field. When the value is read from this
object the first time, the query is executed.

\end{fulllineitems}

\index{objects (checklisten.models.Aufgabe attribute)@\spxentry{objects}\spxextra{checklisten.models.Aufgabe attribute}}

\begin{fulllineitems}
\phantomsection\label{\detokenize{masterCodeDoc:checklisten.models.Aufgabe.objects}}\pysigline{\sphinxbfcode{\sphinxupquote{objects}}\sphinxbfcode{\sphinxupquote{ = \textless{}django.db.models.manager.Manager object\textgreater{}}}}
\end{fulllineitems}

\index{save\_without\_historical\_record() (checklisten.models.Aufgabe method)@\spxentry{save\_without\_historical\_record()}\spxextra{checklisten.models.Aufgabe method}}

\begin{fulllineitems}
\phantomsection\label{\detokenize{masterCodeDoc:checklisten.models.Aufgabe.save_without_historical_record}}\pysiglinewithargsret{\sphinxbfcode{\sphinxupquote{save\_without\_historical\_record}}}{\emph{\DUrole{o}{*}\DUrole{n}{args}}, \emph{\DUrole{o}{**}\DUrole{n}{kwargs}}}{}
Save model without saving a historical record

Make sure you know what you’re doing before you use this method.

\end{fulllineitems}


\end{fulllineitems}

\index{Checkliste (class in checklisten.models)@\spxentry{Checkliste}\spxextra{class in checklisten.models}}

\begin{fulllineitems}
\phantomsection\label{\detokenize{masterCodeDoc:checklisten.models.Checkliste}}\pysiglinewithargsret{\sphinxbfcode{\sphinxupquote{class }}\sphinxcode{\sphinxupquote{checklisten.models.}}\sphinxbfcode{\sphinxupquote{Checkliste}}}{\emph{\DUrole{o}{*}\DUrole{n}{args}}, \emph{\DUrole{o}{**}\DUrole{n}{kwargs}}}{}
Represents a checklist.
\begin{itemize}
\item {} 
mitglied: The Mitglied the checklist was created for. Must not be null.

\item {} 
amt: The Funktion the checklist was created for. Can be null.

\item {} 
history: Contains a log of all changes made to the model. Provided by django\sphinxhyphen{}simple\sphinxhyphen{}history.

\end{itemize}

Note that the checklist will be deleted if the associated Mitglied or Funktion is deleted (cascade).
\index{Checkliste.DoesNotExist@\spxentry{Checkliste.DoesNotExist}}

\begin{fulllineitems}
\phantomsection\label{\detokenize{masterCodeDoc:checklisten.models.Checkliste.DoesNotExist}}\pysigline{\sphinxbfcode{\sphinxupquote{exception }}\sphinxbfcode{\sphinxupquote{DoesNotExist}}}
\end{fulllineitems}

\index{Checkliste.MultipleObjectsReturned@\spxentry{Checkliste.MultipleObjectsReturned}}

\begin{fulllineitems}
\phantomsection\label{\detokenize{masterCodeDoc:checklisten.models.Checkliste.MultipleObjectsReturned}}\pysigline{\sphinxbfcode{\sphinxupquote{exception }}\sphinxbfcode{\sphinxupquote{MultipleObjectsReturned}}}
\end{fulllineitems}

\index{amt (checklisten.models.Checkliste attribute)@\spxentry{amt}\spxextra{checklisten.models.Checkliste attribute}}

\begin{fulllineitems}
\phantomsection\label{\detokenize{masterCodeDoc:checklisten.models.Checkliste.amt}}\pysigline{\sphinxbfcode{\sphinxupquote{amt}}}
Accessor to the related object on the forward side of a many\sphinxhyphen{}to\sphinxhyphen{}one or
one\sphinxhyphen{}to\sphinxhyphen{}one (via ForwardOneToOneDescriptor subclass) relation.

In the example:

\begin{sphinxVerbatim}[commandchars=\\\{\}]
\PYG{k}{class} \PYG{n+nc}{Child}\PYG{p}{(}\PYG{n}{Model}\PYG{p}{)}\PYG{p}{:}
    \PYG{n}{parent} \PYG{o}{=} \PYG{n}{ForeignKey}\PYG{p}{(}\PYG{n}{Parent}\PYG{p}{,} \PYG{n}{related\PYGZus{}name}\PYG{o}{=}\PYG{l+s+s1}{\PYGZsq{}}\PYG{l+s+s1}{children}\PYG{l+s+s1}{\PYGZsq{}}\PYG{p}{)}
\end{sphinxVerbatim}

\sphinxcode{\sphinxupquote{Child.parent}} is a \sphinxcode{\sphinxupquote{ForwardManyToOneDescriptor}} instance.

\end{fulllineitems}

\index{amt\_id (checklisten.models.Checkliste attribute)@\spxentry{amt\_id}\spxextra{checklisten.models.Checkliste attribute}}

\begin{fulllineitems}
\phantomsection\label{\detokenize{masterCodeDoc:checklisten.models.Checkliste.amt_id}}\pysigline{\sphinxbfcode{\sphinxupquote{amt\_id}}}
\end{fulllineitems}

\index{checklisteaufgabe\_set (checklisten.models.Checkliste attribute)@\spxentry{checklisteaufgabe\_set}\spxextra{checklisten.models.Checkliste attribute}}

\begin{fulllineitems}
\phantomsection\label{\detokenize{masterCodeDoc:checklisten.models.Checkliste.checklisteaufgabe_set}}\pysigline{\sphinxbfcode{\sphinxupquote{checklisteaufgabe\_set}}}
Accessor to the related objects manager on the reverse side of a
many\sphinxhyphen{}to\sphinxhyphen{}one relation.

In the example:

\begin{sphinxVerbatim}[commandchars=\\\{\}]
\PYG{k}{class} \PYG{n+nc}{Child}\PYG{p}{(}\PYG{n}{Model}\PYG{p}{)}\PYG{p}{:}
    \PYG{n}{parent} \PYG{o}{=} \PYG{n}{ForeignKey}\PYG{p}{(}\PYG{n}{Parent}\PYG{p}{,} \PYG{n}{related\PYGZus{}name}\PYG{o}{=}\PYG{l+s+s1}{\PYGZsq{}}\PYG{l+s+s1}{children}\PYG{l+s+s1}{\PYGZsq{}}\PYG{p}{)}
\end{sphinxVerbatim}

\sphinxcode{\sphinxupquote{Parent.children}} is a \sphinxcode{\sphinxupquote{ReverseManyToOneDescriptor}} instance.

Most of the implementation is delegated to a dynamically defined manager
class built by \sphinxcode{\sphinxupquote{create\_forward\_many\_to\_many\_manager()}} defined below.

\end{fulllineitems}

\index{checklisterecht\_set (checklisten.models.Checkliste attribute)@\spxentry{checklisterecht\_set}\spxextra{checklisten.models.Checkliste attribute}}

\begin{fulllineitems}
\phantomsection\label{\detokenize{masterCodeDoc:checklisten.models.Checkliste.checklisterecht_set}}\pysigline{\sphinxbfcode{\sphinxupquote{checklisterecht\_set}}}
Accessor to the related objects manager on the reverse side of a
many\sphinxhyphen{}to\sphinxhyphen{}one relation.

In the example:

\begin{sphinxVerbatim}[commandchars=\\\{\}]
\PYG{k}{class} \PYG{n+nc}{Child}\PYG{p}{(}\PYG{n}{Model}\PYG{p}{)}\PYG{p}{:}
    \PYG{n}{parent} \PYG{o}{=} \PYG{n}{ForeignKey}\PYG{p}{(}\PYG{n}{Parent}\PYG{p}{,} \PYG{n}{related\PYGZus{}name}\PYG{o}{=}\PYG{l+s+s1}{\PYGZsq{}}\PYG{l+s+s1}{children}\PYG{l+s+s1}{\PYGZsq{}}\PYG{p}{)}
\end{sphinxVerbatim}

\sphinxcode{\sphinxupquote{Parent.children}} is a \sphinxcode{\sphinxupquote{ReverseManyToOneDescriptor}} instance.

Most of the implementation is delegated to a dynamically defined manager
class built by \sphinxcode{\sphinxupquote{create\_forward\_many\_to\_many\_manager()}} defined below.

\end{fulllineitems}

\index{history (checklisten.models.Checkliste attribute)@\spxentry{history}\spxextra{checklisten.models.Checkliste attribute}}

\begin{fulllineitems}
\phantomsection\label{\detokenize{masterCodeDoc:checklisten.models.Checkliste.history}}\pysigline{\sphinxbfcode{\sphinxupquote{history}}\sphinxbfcode{\sphinxupquote{ = \textless{}simple\_history.manager.HistoryManager object\textgreater{}}}}
\end{fulllineitems}

\index{id (checklisten.models.Checkliste attribute)@\spxentry{id}\spxextra{checklisten.models.Checkliste attribute}}

\begin{fulllineitems}
\phantomsection\label{\detokenize{masterCodeDoc:checklisten.models.Checkliste.id}}\pysigline{\sphinxbfcode{\sphinxupquote{id}}}
A wrapper for a deferred\sphinxhyphen{}loading field. When the value is read from this
object the first time, the query is executed.

\end{fulllineitems}

\index{mitglied (checklisten.models.Checkliste attribute)@\spxentry{mitglied}\spxextra{checklisten.models.Checkliste attribute}}

\begin{fulllineitems}
\phantomsection\label{\detokenize{masterCodeDoc:checklisten.models.Checkliste.mitglied}}\pysigline{\sphinxbfcode{\sphinxupquote{mitglied}}}
Accessor to the related object on the forward side of a many\sphinxhyphen{}to\sphinxhyphen{}one or
one\sphinxhyphen{}to\sphinxhyphen{}one (via ForwardOneToOneDescriptor subclass) relation.

In the example:

\begin{sphinxVerbatim}[commandchars=\\\{\}]
\PYG{k}{class} \PYG{n+nc}{Child}\PYG{p}{(}\PYG{n}{Model}\PYG{p}{)}\PYG{p}{:}
    \PYG{n}{parent} \PYG{o}{=} \PYG{n}{ForeignKey}\PYG{p}{(}\PYG{n}{Parent}\PYG{p}{,} \PYG{n}{related\PYGZus{}name}\PYG{o}{=}\PYG{l+s+s1}{\PYGZsq{}}\PYG{l+s+s1}{children}\PYG{l+s+s1}{\PYGZsq{}}\PYG{p}{)}
\end{sphinxVerbatim}

\sphinxcode{\sphinxupquote{Child.parent}} is a \sphinxcode{\sphinxupquote{ForwardManyToOneDescriptor}} instance.

\end{fulllineitems}

\index{mitglied\_id (checklisten.models.Checkliste attribute)@\spxentry{mitglied\_id}\spxextra{checklisten.models.Checkliste attribute}}

\begin{fulllineitems}
\phantomsection\label{\detokenize{masterCodeDoc:checklisten.models.Checkliste.mitglied_id}}\pysigline{\sphinxbfcode{\sphinxupquote{mitglied\_id}}}
\end{fulllineitems}

\index{objects (checklisten.models.Checkliste attribute)@\spxentry{objects}\spxextra{checklisten.models.Checkliste attribute}}

\begin{fulllineitems}
\phantomsection\label{\detokenize{masterCodeDoc:checklisten.models.Checkliste.objects}}\pysigline{\sphinxbfcode{\sphinxupquote{objects}}\sphinxbfcode{\sphinxupquote{ = \textless{}django.db.models.manager.Manager object\textgreater{}}}}
\end{fulllineitems}

\index{save\_without\_historical\_record() (checklisten.models.Checkliste method)@\spxentry{save\_without\_historical\_record()}\spxextra{checklisten.models.Checkliste method}}

\begin{fulllineitems}
\phantomsection\label{\detokenize{masterCodeDoc:checklisten.models.Checkliste.save_without_historical_record}}\pysiglinewithargsret{\sphinxbfcode{\sphinxupquote{save\_without\_historical\_record}}}{\emph{\DUrole{o}{*}\DUrole{n}{args}}, \emph{\DUrole{o}{**}\DUrole{n}{kwargs}}}{}
Save model without saving a historical record

Make sure you know what you’re doing before you use this method.

\end{fulllineitems}


\end{fulllineitems}

\index{ChecklisteAufgabe (class in checklisten.models)@\spxentry{ChecklisteAufgabe}\spxextra{class in checklisten.models}}

\begin{fulllineitems}
\phantomsection\label{\detokenize{masterCodeDoc:checklisten.models.ChecklisteAufgabe}}\pysiglinewithargsret{\sphinxbfcode{\sphinxupquote{class }}\sphinxcode{\sphinxupquote{checklisten.models.}}\sphinxbfcode{\sphinxupquote{ChecklisteAufgabe}}}{\emph{\DUrole{o}{*}\DUrole{n}{args}}, \emph{\DUrole{o}{**}\DUrole{n}{kwargs}}}{}
Represents a general task inside of a specific checklist.
\begin{itemize}
\item {} 
checkliste: The checklist that this task belongs to. Must not be null.

\item {} 
aufgabe: The task that was assigned to the checklist. Must not be null.

\item {} 
abgehakt: Whether or not the task has been completed for this specific checklist. Must not be null and is false by default.

\item {} 
history: Contains a log of all changes made to the model. Provided by django\sphinxhyphen{}simple\sphinxhyphen{}history.

\end{itemize}

Please note that the entry is deleted if the associated Checkliste or Aufgabe is deleted (cascade).
\index{ChecklisteAufgabe.DoesNotExist@\spxentry{ChecklisteAufgabe.DoesNotExist}}

\begin{fulllineitems}
\phantomsection\label{\detokenize{masterCodeDoc:checklisten.models.ChecklisteAufgabe.DoesNotExist}}\pysigline{\sphinxbfcode{\sphinxupquote{exception }}\sphinxbfcode{\sphinxupquote{DoesNotExist}}}
\end{fulllineitems}

\index{ChecklisteAufgabe.MultipleObjectsReturned@\spxentry{ChecklisteAufgabe.MultipleObjectsReturned}}

\begin{fulllineitems}
\phantomsection\label{\detokenize{masterCodeDoc:checklisten.models.ChecklisteAufgabe.MultipleObjectsReturned}}\pysigline{\sphinxbfcode{\sphinxupquote{exception }}\sphinxbfcode{\sphinxupquote{MultipleObjectsReturned}}}
\end{fulllineitems}

\index{abgehakt (checklisten.models.ChecklisteAufgabe attribute)@\spxentry{abgehakt}\spxextra{checklisten.models.ChecklisteAufgabe attribute}}

\begin{fulllineitems}
\phantomsection\label{\detokenize{masterCodeDoc:checklisten.models.ChecklisteAufgabe.abgehakt}}\pysigline{\sphinxbfcode{\sphinxupquote{abgehakt}}}
A wrapper for a deferred\sphinxhyphen{}loading field. When the value is read from this
object the first time, the query is executed.

\end{fulllineitems}

\index{aufgabe (checklisten.models.ChecklisteAufgabe attribute)@\spxentry{aufgabe}\spxextra{checklisten.models.ChecklisteAufgabe attribute}}

\begin{fulllineitems}
\phantomsection\label{\detokenize{masterCodeDoc:checklisten.models.ChecklisteAufgabe.aufgabe}}\pysigline{\sphinxbfcode{\sphinxupquote{aufgabe}}}
Accessor to the related object on the forward side of a many\sphinxhyphen{}to\sphinxhyphen{}one or
one\sphinxhyphen{}to\sphinxhyphen{}one (via ForwardOneToOneDescriptor subclass) relation.

In the example:

\begin{sphinxVerbatim}[commandchars=\\\{\}]
\PYG{k}{class} \PYG{n+nc}{Child}\PYG{p}{(}\PYG{n}{Model}\PYG{p}{)}\PYG{p}{:}
    \PYG{n}{parent} \PYG{o}{=} \PYG{n}{ForeignKey}\PYG{p}{(}\PYG{n}{Parent}\PYG{p}{,} \PYG{n}{related\PYGZus{}name}\PYG{o}{=}\PYG{l+s+s1}{\PYGZsq{}}\PYG{l+s+s1}{children}\PYG{l+s+s1}{\PYGZsq{}}\PYG{p}{)}
\end{sphinxVerbatim}

\sphinxcode{\sphinxupquote{Child.parent}} is a \sphinxcode{\sphinxupquote{ForwardManyToOneDescriptor}} instance.

\end{fulllineitems}

\index{aufgabe\_id (checklisten.models.ChecklisteAufgabe attribute)@\spxentry{aufgabe\_id}\spxextra{checklisten.models.ChecklisteAufgabe attribute}}

\begin{fulllineitems}
\phantomsection\label{\detokenize{masterCodeDoc:checklisten.models.ChecklisteAufgabe.aufgabe_id}}\pysigline{\sphinxbfcode{\sphinxupquote{aufgabe\_id}}}
\end{fulllineitems}

\index{checkliste (checklisten.models.ChecklisteAufgabe attribute)@\spxentry{checkliste}\spxextra{checklisten.models.ChecklisteAufgabe attribute}}

\begin{fulllineitems}
\phantomsection\label{\detokenize{masterCodeDoc:checklisten.models.ChecklisteAufgabe.checkliste}}\pysigline{\sphinxbfcode{\sphinxupquote{checkliste}}}
Accessor to the related object on the forward side of a many\sphinxhyphen{}to\sphinxhyphen{}one or
one\sphinxhyphen{}to\sphinxhyphen{}one (via ForwardOneToOneDescriptor subclass) relation.

In the example:

\begin{sphinxVerbatim}[commandchars=\\\{\}]
\PYG{k}{class} \PYG{n+nc}{Child}\PYG{p}{(}\PYG{n}{Model}\PYG{p}{)}\PYG{p}{:}
    \PYG{n}{parent} \PYG{o}{=} \PYG{n}{ForeignKey}\PYG{p}{(}\PYG{n}{Parent}\PYG{p}{,} \PYG{n}{related\PYGZus{}name}\PYG{o}{=}\PYG{l+s+s1}{\PYGZsq{}}\PYG{l+s+s1}{children}\PYG{l+s+s1}{\PYGZsq{}}\PYG{p}{)}
\end{sphinxVerbatim}

\sphinxcode{\sphinxupquote{Child.parent}} is a \sphinxcode{\sphinxupquote{ForwardManyToOneDescriptor}} instance.

\end{fulllineitems}

\index{checkliste\_id (checklisten.models.ChecklisteAufgabe attribute)@\spxentry{checkliste\_id}\spxextra{checklisten.models.ChecklisteAufgabe attribute}}

\begin{fulllineitems}
\phantomsection\label{\detokenize{masterCodeDoc:checklisten.models.ChecklisteAufgabe.checkliste_id}}\pysigline{\sphinxbfcode{\sphinxupquote{checkliste\_id}}}
\end{fulllineitems}

\index{history (checklisten.models.ChecklisteAufgabe attribute)@\spxentry{history}\spxextra{checklisten.models.ChecklisteAufgabe attribute}}

\begin{fulllineitems}
\phantomsection\label{\detokenize{masterCodeDoc:checklisten.models.ChecklisteAufgabe.history}}\pysigline{\sphinxbfcode{\sphinxupquote{history}}\sphinxbfcode{\sphinxupquote{ = \textless{}simple\_history.manager.HistoryManager object\textgreater{}}}}
\end{fulllineitems}

\index{id (checklisten.models.ChecklisteAufgabe attribute)@\spxentry{id}\spxextra{checklisten.models.ChecklisteAufgabe attribute}}

\begin{fulllineitems}
\phantomsection\label{\detokenize{masterCodeDoc:checklisten.models.ChecklisteAufgabe.id}}\pysigline{\sphinxbfcode{\sphinxupquote{id}}}
A wrapper for a deferred\sphinxhyphen{}loading field. When the value is read from this
object the first time, the query is executed.

\end{fulllineitems}

\index{objects (checklisten.models.ChecklisteAufgabe attribute)@\spxentry{objects}\spxextra{checklisten.models.ChecklisteAufgabe attribute}}

\begin{fulllineitems}
\phantomsection\label{\detokenize{masterCodeDoc:checklisten.models.ChecklisteAufgabe.objects}}\pysigline{\sphinxbfcode{\sphinxupquote{objects}}\sphinxbfcode{\sphinxupquote{ = \textless{}django.db.models.manager.Manager object\textgreater{}}}}
\end{fulllineitems}

\index{save\_without\_historical\_record() (checklisten.models.ChecklisteAufgabe method)@\spxentry{save\_without\_historical\_record()}\spxextra{checklisten.models.ChecklisteAufgabe method}}

\begin{fulllineitems}
\phantomsection\label{\detokenize{masterCodeDoc:checklisten.models.ChecklisteAufgabe.save_without_historical_record}}\pysiglinewithargsret{\sphinxbfcode{\sphinxupquote{save\_without\_historical\_record}}}{\emph{\DUrole{o}{*}\DUrole{n}{args}}, \emph{\DUrole{o}{**}\DUrole{n}{kwargs}}}{}
Save model without saving a historical record

Make sure you know what you’re doing before you use this method.

\end{fulllineitems}


\end{fulllineitems}

\index{ChecklisteRecht (class in checklisten.models)@\spxentry{ChecklisteRecht}\spxextra{class in checklisten.models}}

\begin{fulllineitems}
\phantomsection\label{\detokenize{masterCodeDoc:checklisten.models.ChecklisteRecht}}\pysiglinewithargsret{\sphinxbfcode{\sphinxupquote{class }}\sphinxcode{\sphinxupquote{checklisten.models.}}\sphinxbfcode{\sphinxupquote{ChecklisteRecht}}}{\emph{\DUrole{o}{*}\DUrole{n}{args}}, \emph{\DUrole{o}{**}\DUrole{n}{kwargs}}}{}
Represents a Recht to be given inside of a specific checklist.
\begin{itemize}
\item {} 
checkliste: The checklist that this Recht was assigned to. Must not be null.

\item {} 
recht: The Recht that was assigned to the checklist. Must not be null.

\item {} 
abgehakt: Whether or not the Recht has been given for this specific checklist. Must not be null and is false by default.

\item {} 
history: Contains a log of all changes made to the model. Provided by django\sphinxhyphen{}simple\sphinxhyphen{}history.

\end{itemize}

Please note that the entry is deleted if the associated Checkliste or Recht is deleted (cascade).
\index{ChecklisteRecht.DoesNotExist@\spxentry{ChecklisteRecht.DoesNotExist}}

\begin{fulllineitems}
\phantomsection\label{\detokenize{masterCodeDoc:checklisten.models.ChecklisteRecht.DoesNotExist}}\pysigline{\sphinxbfcode{\sphinxupquote{exception }}\sphinxbfcode{\sphinxupquote{DoesNotExist}}}
\end{fulllineitems}

\index{ChecklisteRecht.MultipleObjectsReturned@\spxentry{ChecklisteRecht.MultipleObjectsReturned}}

\begin{fulllineitems}
\phantomsection\label{\detokenize{masterCodeDoc:checklisten.models.ChecklisteRecht.MultipleObjectsReturned}}\pysigline{\sphinxbfcode{\sphinxupquote{exception }}\sphinxbfcode{\sphinxupquote{MultipleObjectsReturned}}}
\end{fulllineitems}

\index{abgehakt (checklisten.models.ChecklisteRecht attribute)@\spxentry{abgehakt}\spxextra{checklisten.models.ChecklisteRecht attribute}}

\begin{fulllineitems}
\phantomsection\label{\detokenize{masterCodeDoc:checklisten.models.ChecklisteRecht.abgehakt}}\pysigline{\sphinxbfcode{\sphinxupquote{abgehakt}}}
A wrapper for a deferred\sphinxhyphen{}loading field. When the value is read from this
object the first time, the query is executed.

\end{fulllineitems}

\index{checkliste (checklisten.models.ChecklisteRecht attribute)@\spxentry{checkliste}\spxextra{checklisten.models.ChecklisteRecht attribute}}

\begin{fulllineitems}
\phantomsection\label{\detokenize{masterCodeDoc:checklisten.models.ChecklisteRecht.checkliste}}\pysigline{\sphinxbfcode{\sphinxupquote{checkliste}}}
Accessor to the related object on the forward side of a many\sphinxhyphen{}to\sphinxhyphen{}one or
one\sphinxhyphen{}to\sphinxhyphen{}one (via ForwardOneToOneDescriptor subclass) relation.

In the example:

\begin{sphinxVerbatim}[commandchars=\\\{\}]
\PYG{k}{class} \PYG{n+nc}{Child}\PYG{p}{(}\PYG{n}{Model}\PYG{p}{)}\PYG{p}{:}
    \PYG{n}{parent} \PYG{o}{=} \PYG{n}{ForeignKey}\PYG{p}{(}\PYG{n}{Parent}\PYG{p}{,} \PYG{n}{related\PYGZus{}name}\PYG{o}{=}\PYG{l+s+s1}{\PYGZsq{}}\PYG{l+s+s1}{children}\PYG{l+s+s1}{\PYGZsq{}}\PYG{p}{)}
\end{sphinxVerbatim}

\sphinxcode{\sphinxupquote{Child.parent}} is a \sphinxcode{\sphinxupquote{ForwardManyToOneDescriptor}} instance.

\end{fulllineitems}

\index{checkliste\_id (checklisten.models.ChecklisteRecht attribute)@\spxentry{checkliste\_id}\spxextra{checklisten.models.ChecklisteRecht attribute}}

\begin{fulllineitems}
\phantomsection\label{\detokenize{masterCodeDoc:checklisten.models.ChecklisteRecht.checkliste_id}}\pysigline{\sphinxbfcode{\sphinxupquote{checkliste\_id}}}
\end{fulllineitems}

\index{history (checklisten.models.ChecklisteRecht attribute)@\spxentry{history}\spxextra{checklisten.models.ChecklisteRecht attribute}}

\begin{fulllineitems}
\phantomsection\label{\detokenize{masterCodeDoc:checklisten.models.ChecklisteRecht.history}}\pysigline{\sphinxbfcode{\sphinxupquote{history}}\sphinxbfcode{\sphinxupquote{ = \textless{}simple\_history.manager.HistoryManager object\textgreater{}}}}
\end{fulllineitems}

\index{id (checklisten.models.ChecklisteRecht attribute)@\spxentry{id}\spxextra{checklisten.models.ChecklisteRecht attribute}}

\begin{fulllineitems}
\phantomsection\label{\detokenize{masterCodeDoc:checklisten.models.ChecklisteRecht.id}}\pysigline{\sphinxbfcode{\sphinxupquote{id}}}
A wrapper for a deferred\sphinxhyphen{}loading field. When the value is read from this
object the first time, the query is executed.

\end{fulllineitems}

\index{objects (checklisten.models.ChecklisteRecht attribute)@\spxentry{objects}\spxextra{checklisten.models.ChecklisteRecht attribute}}

\begin{fulllineitems}
\phantomsection\label{\detokenize{masterCodeDoc:checklisten.models.ChecklisteRecht.objects}}\pysigline{\sphinxbfcode{\sphinxupquote{objects}}\sphinxbfcode{\sphinxupquote{ = \textless{}django.db.models.manager.Manager object\textgreater{}}}}
\end{fulllineitems}

\index{recht (checklisten.models.ChecklisteRecht attribute)@\spxentry{recht}\spxextra{checklisten.models.ChecklisteRecht attribute}}

\begin{fulllineitems}
\phantomsection\label{\detokenize{masterCodeDoc:checklisten.models.ChecklisteRecht.recht}}\pysigline{\sphinxbfcode{\sphinxupquote{recht}}}
Accessor to the related object on the forward side of a many\sphinxhyphen{}to\sphinxhyphen{}one or
one\sphinxhyphen{}to\sphinxhyphen{}one (via ForwardOneToOneDescriptor subclass) relation.

In the example:

\begin{sphinxVerbatim}[commandchars=\\\{\}]
\PYG{k}{class} \PYG{n+nc}{Child}\PYG{p}{(}\PYG{n}{Model}\PYG{p}{)}\PYG{p}{:}
    \PYG{n}{parent} \PYG{o}{=} \PYG{n}{ForeignKey}\PYG{p}{(}\PYG{n}{Parent}\PYG{p}{,} \PYG{n}{related\PYGZus{}name}\PYG{o}{=}\PYG{l+s+s1}{\PYGZsq{}}\PYG{l+s+s1}{children}\PYG{l+s+s1}{\PYGZsq{}}\PYG{p}{)}
\end{sphinxVerbatim}

\sphinxcode{\sphinxupquote{Child.parent}} is a \sphinxcode{\sphinxupquote{ForwardManyToOneDescriptor}} instance.

\end{fulllineitems}

\index{recht\_id (checklisten.models.ChecklisteRecht attribute)@\spxentry{recht\_id}\spxextra{checklisten.models.ChecklisteRecht attribute}}

\begin{fulllineitems}
\phantomsection\label{\detokenize{masterCodeDoc:checklisten.models.ChecklisteRecht.recht_id}}\pysigline{\sphinxbfcode{\sphinxupquote{recht\_id}}}
\end{fulllineitems}

\index{save\_without\_historical\_record() (checklisten.models.ChecklisteRecht method)@\spxentry{save\_without\_historical\_record()}\spxextra{checklisten.models.ChecklisteRecht method}}

\begin{fulllineitems}
\phantomsection\label{\detokenize{masterCodeDoc:checklisten.models.ChecklisteRecht.save_without_historical_record}}\pysiglinewithargsret{\sphinxbfcode{\sphinxupquote{save\_without\_historical\_record}}}{\emph{\DUrole{o}{*}\DUrole{n}{args}}, \emph{\DUrole{o}{**}\DUrole{n}{kwargs}}}{}
Save model without saving a historical record

Make sure you know what you’re doing before you use this method.

\end{fulllineitems}


\end{fulllineitems}

\index{HistoricalAufgabe (class in checklisten.models)@\spxentry{HistoricalAufgabe}\spxextra{class in checklisten.models}}

\begin{fulllineitems}
\phantomsection\label{\detokenize{masterCodeDoc:checklisten.models.HistoricalAufgabe}}\pysiglinewithargsret{\sphinxbfcode{\sphinxupquote{class }}\sphinxcode{\sphinxupquote{checklisten.models.}}\sphinxbfcode{\sphinxupquote{HistoricalAufgabe}}}{\emph{\DUrole{n}{id}}, \emph{\DUrole{n}{bezeichnung}}, \emph{\DUrole{n}{history\_id}}, \emph{\DUrole{n}{history\_date}}, \emph{\DUrole{n}{history\_change\_reason}}, \emph{\DUrole{n}{history\_type}}, \emph{\DUrole{n}{history\_user}}}{}~\index{HistoricalAufgabe.DoesNotExist@\spxentry{HistoricalAufgabe.DoesNotExist}}

\begin{fulllineitems}
\phantomsection\label{\detokenize{masterCodeDoc:checklisten.models.HistoricalAufgabe.DoesNotExist}}\pysigline{\sphinxbfcode{\sphinxupquote{exception }}\sphinxbfcode{\sphinxupquote{DoesNotExist}}}
\end{fulllineitems}

\index{HistoricalAufgabe.MultipleObjectsReturned@\spxentry{HistoricalAufgabe.MultipleObjectsReturned}}

\begin{fulllineitems}
\phantomsection\label{\detokenize{masterCodeDoc:checklisten.models.HistoricalAufgabe.MultipleObjectsReturned}}\pysigline{\sphinxbfcode{\sphinxupquote{exception }}\sphinxbfcode{\sphinxupquote{MultipleObjectsReturned}}}
\end{fulllineitems}

\index{bezeichnung (checklisten.models.HistoricalAufgabe attribute)@\spxentry{bezeichnung}\spxextra{checklisten.models.HistoricalAufgabe attribute}}

\begin{fulllineitems}
\phantomsection\label{\detokenize{masterCodeDoc:checklisten.models.HistoricalAufgabe.bezeichnung}}\pysigline{\sphinxbfcode{\sphinxupquote{bezeichnung}}}
A wrapper for a deferred\sphinxhyphen{}loading field. When the value is read from this
object the first time, the query is executed.

\end{fulllineitems}

\index{get\_default\_history\_user() (checklisten.models.HistoricalAufgabe static method)@\spxentry{get\_default\_history\_user()}\spxextra{checklisten.models.HistoricalAufgabe static method}}

\begin{fulllineitems}
\phantomsection\label{\detokenize{masterCodeDoc:checklisten.models.HistoricalAufgabe.get_default_history_user}}\pysiglinewithargsret{\sphinxbfcode{\sphinxupquote{static }}\sphinxbfcode{\sphinxupquote{get\_default\_history\_user}}}{\emph{\DUrole{n}{instance}}}{}
Returns the user specified by \sphinxtitleref{get\_user} method for manually creating
historical objects

\end{fulllineitems}

\index{get\_history\_type\_display() (checklisten.models.HistoricalAufgabe method)@\spxentry{get\_history\_type\_display()}\spxextra{checklisten.models.HistoricalAufgabe method}}

\begin{fulllineitems}
\phantomsection\label{\detokenize{masterCodeDoc:checklisten.models.HistoricalAufgabe.get_history_type_display}}\pysiglinewithargsret{\sphinxbfcode{\sphinxupquote{get\_history\_type\_display}}}{\emph{*}, \emph{field=\textless{}django.db.models.fields.CharField: history\_type\textgreater{}}}{}
\end{fulllineitems}

\index{get\_next\_by\_history\_date() (checklisten.models.HistoricalAufgabe method)@\spxentry{get\_next\_by\_history\_date()}\spxextra{checklisten.models.HistoricalAufgabe method}}

\begin{fulllineitems}
\phantomsection\label{\detokenize{masterCodeDoc:checklisten.models.HistoricalAufgabe.get_next_by_history_date}}\pysiglinewithargsret{\sphinxbfcode{\sphinxupquote{get\_next\_by\_history\_date}}}{\emph{*}, \emph{field=\textless{}django.db.models.fields.DateTimeField: history\_date\textgreater{}}, \emph{is\_next=True}, \emph{**kwargs}}{}
\end{fulllineitems}

\index{get\_previous\_by\_history\_date() (checklisten.models.HistoricalAufgabe method)@\spxentry{get\_previous\_by\_history\_date()}\spxextra{checklisten.models.HistoricalAufgabe method}}

\begin{fulllineitems}
\phantomsection\label{\detokenize{masterCodeDoc:checklisten.models.HistoricalAufgabe.get_previous_by_history_date}}\pysiglinewithargsret{\sphinxbfcode{\sphinxupquote{get\_previous\_by\_history\_date}}}{\emph{*}, \emph{field=\textless{}django.db.models.fields.DateTimeField: history\_date\textgreater{}}, \emph{is\_next=False}, \emph{**kwargs}}{}
\end{fulllineitems}

\index{history\_change\_reason (checklisten.models.HistoricalAufgabe attribute)@\spxentry{history\_change\_reason}\spxextra{checklisten.models.HistoricalAufgabe attribute}}

\begin{fulllineitems}
\phantomsection\label{\detokenize{masterCodeDoc:checklisten.models.HistoricalAufgabe.history_change_reason}}\pysigline{\sphinxbfcode{\sphinxupquote{history\_change\_reason}}}
A wrapper for a deferred\sphinxhyphen{}loading field. When the value is read from this
object the first time, the query is executed.

\end{fulllineitems}

\index{history\_date (checklisten.models.HistoricalAufgabe attribute)@\spxentry{history\_date}\spxextra{checklisten.models.HistoricalAufgabe attribute}}

\begin{fulllineitems}
\phantomsection\label{\detokenize{masterCodeDoc:checklisten.models.HistoricalAufgabe.history_date}}\pysigline{\sphinxbfcode{\sphinxupquote{history\_date}}}
A wrapper for a deferred\sphinxhyphen{}loading field. When the value is read from this
object the first time, the query is executed.

\end{fulllineitems}

\index{history\_id (checklisten.models.HistoricalAufgabe attribute)@\spxentry{history\_id}\spxextra{checklisten.models.HistoricalAufgabe attribute}}

\begin{fulllineitems}
\phantomsection\label{\detokenize{masterCodeDoc:checklisten.models.HistoricalAufgabe.history_id}}\pysigline{\sphinxbfcode{\sphinxupquote{history\_id}}}
A wrapper for a deferred\sphinxhyphen{}loading field. When the value is read from this
object the first time, the query is executed.

\end{fulllineitems}

\index{history\_object (checklisten.models.HistoricalAufgabe attribute)@\spxentry{history\_object}\spxextra{checklisten.models.HistoricalAufgabe attribute}}

\begin{fulllineitems}
\phantomsection\label{\detokenize{masterCodeDoc:checklisten.models.HistoricalAufgabe.history_object}}\pysigline{\sphinxbfcode{\sphinxupquote{history\_object}}}
\end{fulllineitems}

\index{history\_type (checklisten.models.HistoricalAufgabe attribute)@\spxentry{history\_type}\spxextra{checklisten.models.HistoricalAufgabe attribute}}

\begin{fulllineitems}
\phantomsection\label{\detokenize{masterCodeDoc:checklisten.models.HistoricalAufgabe.history_type}}\pysigline{\sphinxbfcode{\sphinxupquote{history\_type}}}
A wrapper for a deferred\sphinxhyphen{}loading field. When the value is read from this
object the first time, the query is executed.

\end{fulllineitems}

\index{history\_user (checklisten.models.HistoricalAufgabe attribute)@\spxentry{history\_user}\spxextra{checklisten.models.HistoricalAufgabe attribute}}

\begin{fulllineitems}
\phantomsection\label{\detokenize{masterCodeDoc:checklisten.models.HistoricalAufgabe.history_user}}\pysigline{\sphinxbfcode{\sphinxupquote{history\_user}}}
Accessor to the related object on the forward side of a many\sphinxhyphen{}to\sphinxhyphen{}one or
one\sphinxhyphen{}to\sphinxhyphen{}one (via ForwardOneToOneDescriptor subclass) relation.

In the example:

\begin{sphinxVerbatim}[commandchars=\\\{\}]
\PYG{k}{class} \PYG{n+nc}{Child}\PYG{p}{(}\PYG{n}{Model}\PYG{p}{)}\PYG{p}{:}
    \PYG{n}{parent} \PYG{o}{=} \PYG{n}{ForeignKey}\PYG{p}{(}\PYG{n}{Parent}\PYG{p}{,} \PYG{n}{related\PYGZus{}name}\PYG{o}{=}\PYG{l+s+s1}{\PYGZsq{}}\PYG{l+s+s1}{children}\PYG{l+s+s1}{\PYGZsq{}}\PYG{p}{)}
\end{sphinxVerbatim}

\sphinxcode{\sphinxupquote{Child.parent}} is a \sphinxcode{\sphinxupquote{ForwardManyToOneDescriptor}} instance.

\end{fulllineitems}

\index{history\_user\_id (checklisten.models.HistoricalAufgabe attribute)@\spxentry{history\_user\_id}\spxextra{checklisten.models.HistoricalAufgabe attribute}}

\begin{fulllineitems}
\phantomsection\label{\detokenize{masterCodeDoc:checklisten.models.HistoricalAufgabe.history_user_id}}\pysigline{\sphinxbfcode{\sphinxupquote{history\_user\_id}}}
\end{fulllineitems}

\index{id (checklisten.models.HistoricalAufgabe attribute)@\spxentry{id}\spxextra{checklisten.models.HistoricalAufgabe attribute}}

\begin{fulllineitems}
\phantomsection\label{\detokenize{masterCodeDoc:checklisten.models.HistoricalAufgabe.id}}\pysigline{\sphinxbfcode{\sphinxupquote{id}}}
A wrapper for a deferred\sphinxhyphen{}loading field. When the value is read from this
object the first time, the query is executed.

\end{fulllineitems}

\index{instance() (checklisten.models.HistoricalAufgabe property)@\spxentry{instance()}\spxextra{checklisten.models.HistoricalAufgabe property}}

\begin{fulllineitems}
\phantomsection\label{\detokenize{masterCodeDoc:checklisten.models.HistoricalAufgabe.instance}}\pysigline{\sphinxbfcode{\sphinxupquote{property }}\sphinxbfcode{\sphinxupquote{instance}}}
\end{fulllineitems}

\index{instance\_type (checklisten.models.HistoricalAufgabe attribute)@\spxentry{instance\_type}\spxextra{checklisten.models.HistoricalAufgabe attribute}}

\begin{fulllineitems}
\phantomsection\label{\detokenize{masterCodeDoc:checklisten.models.HistoricalAufgabe.instance_type}}\pysigline{\sphinxbfcode{\sphinxupquote{instance\_type}}}
alias of {\hyperref[\detokenize{masterCodeDoc:checklisten.models.Aufgabe}]{\sphinxcrossref{\sphinxcode{\sphinxupquote{Aufgabe}}}}}

\end{fulllineitems}

\index{next\_record() (checklisten.models.HistoricalAufgabe property)@\spxentry{next\_record()}\spxextra{checklisten.models.HistoricalAufgabe property}}

\begin{fulllineitems}
\phantomsection\label{\detokenize{masterCodeDoc:checklisten.models.HistoricalAufgabe.next_record}}\pysigline{\sphinxbfcode{\sphinxupquote{property }}\sphinxbfcode{\sphinxupquote{next\_record}}}
Get the next history record for the instance. \sphinxtitleref{None} if last.

\end{fulllineitems}

\index{objects (checklisten.models.HistoricalAufgabe attribute)@\spxentry{objects}\spxextra{checklisten.models.HistoricalAufgabe attribute}}

\begin{fulllineitems}
\phantomsection\label{\detokenize{masterCodeDoc:checklisten.models.HistoricalAufgabe.objects}}\pysigline{\sphinxbfcode{\sphinxupquote{objects}}\sphinxbfcode{\sphinxupquote{ = \textless{}django.db.models.manager.Manager object\textgreater{}}}}
\end{fulllineitems}

\index{prev\_record() (checklisten.models.HistoricalAufgabe property)@\spxentry{prev\_record()}\spxextra{checklisten.models.HistoricalAufgabe property}}

\begin{fulllineitems}
\phantomsection\label{\detokenize{masterCodeDoc:checklisten.models.HistoricalAufgabe.prev_record}}\pysigline{\sphinxbfcode{\sphinxupquote{property }}\sphinxbfcode{\sphinxupquote{prev\_record}}}
Get the previous history record for the instance. \sphinxtitleref{None} if first.

\end{fulllineitems}

\index{revert\_url() (checklisten.models.HistoricalAufgabe method)@\spxentry{revert\_url()}\spxextra{checklisten.models.HistoricalAufgabe method}}

\begin{fulllineitems}
\phantomsection\label{\detokenize{masterCodeDoc:checklisten.models.HistoricalAufgabe.revert_url}}\pysiglinewithargsret{\sphinxbfcode{\sphinxupquote{revert\_url}}}{}{}
URL for this change in the default admin site.

\end{fulllineitems}


\end{fulllineitems}

\index{HistoricalCheckliste (class in checklisten.models)@\spxentry{HistoricalCheckliste}\spxextra{class in checklisten.models}}

\begin{fulllineitems}
\phantomsection\label{\detokenize{masterCodeDoc:checklisten.models.HistoricalCheckliste}}\pysiglinewithargsret{\sphinxbfcode{\sphinxupquote{class }}\sphinxcode{\sphinxupquote{checklisten.models.}}\sphinxbfcode{\sphinxupquote{HistoricalCheckliste}}}{\emph{\DUrole{n}{id}}, \emph{\DUrole{n}{mitglied}}, \emph{\DUrole{n}{amt}}, \emph{\DUrole{n}{history\_id}}, \emph{\DUrole{n}{history\_date}}, \emph{\DUrole{n}{history\_change\_reason}}, \emph{\DUrole{n}{history\_type}}, \emph{\DUrole{n}{history\_user}}}{}~\index{HistoricalCheckliste.DoesNotExist@\spxentry{HistoricalCheckliste.DoesNotExist}}

\begin{fulllineitems}
\phantomsection\label{\detokenize{masterCodeDoc:checklisten.models.HistoricalCheckliste.DoesNotExist}}\pysigline{\sphinxbfcode{\sphinxupquote{exception }}\sphinxbfcode{\sphinxupquote{DoesNotExist}}}
\end{fulllineitems}

\index{HistoricalCheckliste.MultipleObjectsReturned@\spxentry{HistoricalCheckliste.MultipleObjectsReturned}}

\begin{fulllineitems}
\phantomsection\label{\detokenize{masterCodeDoc:checklisten.models.HistoricalCheckliste.MultipleObjectsReturned}}\pysigline{\sphinxbfcode{\sphinxupquote{exception }}\sphinxbfcode{\sphinxupquote{MultipleObjectsReturned}}}
\end{fulllineitems}

\index{amt (checklisten.models.HistoricalCheckliste attribute)@\spxentry{amt}\spxextra{checklisten.models.HistoricalCheckliste attribute}}

\begin{fulllineitems}
\phantomsection\label{\detokenize{masterCodeDoc:checklisten.models.HistoricalCheckliste.amt}}\pysigline{\sphinxbfcode{\sphinxupquote{amt}}}
Accessor to the related object on the forward side of a many\sphinxhyphen{}to\sphinxhyphen{}one or
one\sphinxhyphen{}to\sphinxhyphen{}one (via ForwardOneToOneDescriptor subclass) relation.

In the example:

\begin{sphinxVerbatim}[commandchars=\\\{\}]
\PYG{k}{class} \PYG{n+nc}{Child}\PYG{p}{(}\PYG{n}{Model}\PYG{p}{)}\PYG{p}{:}
    \PYG{n}{parent} \PYG{o}{=} \PYG{n}{ForeignKey}\PYG{p}{(}\PYG{n}{Parent}\PYG{p}{,} \PYG{n}{related\PYGZus{}name}\PYG{o}{=}\PYG{l+s+s1}{\PYGZsq{}}\PYG{l+s+s1}{children}\PYG{l+s+s1}{\PYGZsq{}}\PYG{p}{)}
\end{sphinxVerbatim}

\sphinxcode{\sphinxupquote{Child.parent}} is a \sphinxcode{\sphinxupquote{ForwardManyToOneDescriptor}} instance.

\end{fulllineitems}

\index{amt\_id (checklisten.models.HistoricalCheckliste attribute)@\spxentry{amt\_id}\spxextra{checklisten.models.HistoricalCheckliste attribute}}

\begin{fulllineitems}
\phantomsection\label{\detokenize{masterCodeDoc:checklisten.models.HistoricalCheckliste.amt_id}}\pysigline{\sphinxbfcode{\sphinxupquote{amt\_id}}}
\end{fulllineitems}

\index{get\_default\_history\_user() (checklisten.models.HistoricalCheckliste static method)@\spxentry{get\_default\_history\_user()}\spxextra{checklisten.models.HistoricalCheckliste static method}}

\begin{fulllineitems}
\phantomsection\label{\detokenize{masterCodeDoc:checklisten.models.HistoricalCheckliste.get_default_history_user}}\pysiglinewithargsret{\sphinxbfcode{\sphinxupquote{static }}\sphinxbfcode{\sphinxupquote{get\_default\_history\_user}}}{\emph{\DUrole{n}{instance}}}{}
Returns the user specified by \sphinxtitleref{get\_user} method for manually creating
historical objects

\end{fulllineitems}

\index{get\_history\_type\_display() (checklisten.models.HistoricalCheckliste method)@\spxentry{get\_history\_type\_display()}\spxextra{checklisten.models.HistoricalCheckliste method}}

\begin{fulllineitems}
\phantomsection\label{\detokenize{masterCodeDoc:checklisten.models.HistoricalCheckliste.get_history_type_display}}\pysiglinewithargsret{\sphinxbfcode{\sphinxupquote{get\_history\_type\_display}}}{\emph{*}, \emph{field=\textless{}django.db.models.fields.CharField: history\_type\textgreater{}}}{}
\end{fulllineitems}

\index{get\_next\_by\_history\_date() (checklisten.models.HistoricalCheckliste method)@\spxentry{get\_next\_by\_history\_date()}\spxextra{checklisten.models.HistoricalCheckliste method}}

\begin{fulllineitems}
\phantomsection\label{\detokenize{masterCodeDoc:checklisten.models.HistoricalCheckliste.get_next_by_history_date}}\pysiglinewithargsret{\sphinxbfcode{\sphinxupquote{get\_next\_by\_history\_date}}}{\emph{*}, \emph{field=\textless{}django.db.models.fields.DateTimeField: history\_date\textgreater{}}, \emph{is\_next=True}, \emph{**kwargs}}{}
\end{fulllineitems}

\index{get\_previous\_by\_history\_date() (checklisten.models.HistoricalCheckliste method)@\spxentry{get\_previous\_by\_history\_date()}\spxextra{checklisten.models.HistoricalCheckliste method}}

\begin{fulllineitems}
\phantomsection\label{\detokenize{masterCodeDoc:checklisten.models.HistoricalCheckliste.get_previous_by_history_date}}\pysiglinewithargsret{\sphinxbfcode{\sphinxupquote{get\_previous\_by\_history\_date}}}{\emph{*}, \emph{field=\textless{}django.db.models.fields.DateTimeField: history\_date\textgreater{}}, \emph{is\_next=False}, \emph{**kwargs}}{}
\end{fulllineitems}

\index{history\_change\_reason (checklisten.models.HistoricalCheckliste attribute)@\spxentry{history\_change\_reason}\spxextra{checklisten.models.HistoricalCheckliste attribute}}

\begin{fulllineitems}
\phantomsection\label{\detokenize{masterCodeDoc:checklisten.models.HistoricalCheckliste.history_change_reason}}\pysigline{\sphinxbfcode{\sphinxupquote{history\_change\_reason}}}
A wrapper for a deferred\sphinxhyphen{}loading field. When the value is read from this
object the first time, the query is executed.

\end{fulllineitems}

\index{history\_date (checklisten.models.HistoricalCheckliste attribute)@\spxentry{history\_date}\spxextra{checklisten.models.HistoricalCheckliste attribute}}

\begin{fulllineitems}
\phantomsection\label{\detokenize{masterCodeDoc:checklisten.models.HistoricalCheckliste.history_date}}\pysigline{\sphinxbfcode{\sphinxupquote{history\_date}}}
A wrapper for a deferred\sphinxhyphen{}loading field. When the value is read from this
object the first time, the query is executed.

\end{fulllineitems}

\index{history\_id (checklisten.models.HistoricalCheckliste attribute)@\spxentry{history\_id}\spxextra{checklisten.models.HistoricalCheckliste attribute}}

\begin{fulllineitems}
\phantomsection\label{\detokenize{masterCodeDoc:checklisten.models.HistoricalCheckliste.history_id}}\pysigline{\sphinxbfcode{\sphinxupquote{history\_id}}}
A wrapper for a deferred\sphinxhyphen{}loading field. When the value is read from this
object the first time, the query is executed.

\end{fulllineitems}

\index{history\_object (checklisten.models.HistoricalCheckliste attribute)@\spxentry{history\_object}\spxextra{checklisten.models.HistoricalCheckliste attribute}}

\begin{fulllineitems}
\phantomsection\label{\detokenize{masterCodeDoc:checklisten.models.HistoricalCheckliste.history_object}}\pysigline{\sphinxbfcode{\sphinxupquote{history\_object}}}
\end{fulllineitems}

\index{history\_type (checklisten.models.HistoricalCheckliste attribute)@\spxentry{history\_type}\spxextra{checklisten.models.HistoricalCheckliste attribute}}

\begin{fulllineitems}
\phantomsection\label{\detokenize{masterCodeDoc:checklisten.models.HistoricalCheckliste.history_type}}\pysigline{\sphinxbfcode{\sphinxupquote{history\_type}}}
A wrapper for a deferred\sphinxhyphen{}loading field. When the value is read from this
object the first time, the query is executed.

\end{fulllineitems}

\index{history\_user (checklisten.models.HistoricalCheckliste attribute)@\spxentry{history\_user}\spxextra{checklisten.models.HistoricalCheckliste attribute}}

\begin{fulllineitems}
\phantomsection\label{\detokenize{masterCodeDoc:checklisten.models.HistoricalCheckliste.history_user}}\pysigline{\sphinxbfcode{\sphinxupquote{history\_user}}}
Accessor to the related object on the forward side of a many\sphinxhyphen{}to\sphinxhyphen{}one or
one\sphinxhyphen{}to\sphinxhyphen{}one (via ForwardOneToOneDescriptor subclass) relation.

In the example:

\begin{sphinxVerbatim}[commandchars=\\\{\}]
\PYG{k}{class} \PYG{n+nc}{Child}\PYG{p}{(}\PYG{n}{Model}\PYG{p}{)}\PYG{p}{:}
    \PYG{n}{parent} \PYG{o}{=} \PYG{n}{ForeignKey}\PYG{p}{(}\PYG{n}{Parent}\PYG{p}{,} \PYG{n}{related\PYGZus{}name}\PYG{o}{=}\PYG{l+s+s1}{\PYGZsq{}}\PYG{l+s+s1}{children}\PYG{l+s+s1}{\PYGZsq{}}\PYG{p}{)}
\end{sphinxVerbatim}

\sphinxcode{\sphinxupquote{Child.parent}} is a \sphinxcode{\sphinxupquote{ForwardManyToOneDescriptor}} instance.

\end{fulllineitems}

\index{history\_user\_id (checklisten.models.HistoricalCheckliste attribute)@\spxentry{history\_user\_id}\spxextra{checklisten.models.HistoricalCheckliste attribute}}

\begin{fulllineitems}
\phantomsection\label{\detokenize{masterCodeDoc:checklisten.models.HistoricalCheckliste.history_user_id}}\pysigline{\sphinxbfcode{\sphinxupquote{history\_user\_id}}}
\end{fulllineitems}

\index{id (checklisten.models.HistoricalCheckliste attribute)@\spxentry{id}\spxextra{checklisten.models.HistoricalCheckliste attribute}}

\begin{fulllineitems}
\phantomsection\label{\detokenize{masterCodeDoc:checklisten.models.HistoricalCheckliste.id}}\pysigline{\sphinxbfcode{\sphinxupquote{id}}}
A wrapper for a deferred\sphinxhyphen{}loading field. When the value is read from this
object the first time, the query is executed.

\end{fulllineitems}

\index{instance() (checklisten.models.HistoricalCheckliste property)@\spxentry{instance()}\spxextra{checklisten.models.HistoricalCheckliste property}}

\begin{fulllineitems}
\phantomsection\label{\detokenize{masterCodeDoc:checklisten.models.HistoricalCheckliste.instance}}\pysigline{\sphinxbfcode{\sphinxupquote{property }}\sphinxbfcode{\sphinxupquote{instance}}}
\end{fulllineitems}

\index{instance\_type (checklisten.models.HistoricalCheckliste attribute)@\spxentry{instance\_type}\spxextra{checklisten.models.HistoricalCheckliste attribute}}

\begin{fulllineitems}
\phantomsection\label{\detokenize{masterCodeDoc:checklisten.models.HistoricalCheckliste.instance_type}}\pysigline{\sphinxbfcode{\sphinxupquote{instance\_type}}}
alias of {\hyperref[\detokenize{masterCodeDoc:checklisten.models.Checkliste}]{\sphinxcrossref{\sphinxcode{\sphinxupquote{Checkliste}}}}}

\end{fulllineitems}

\index{mitglied (checklisten.models.HistoricalCheckliste attribute)@\spxentry{mitglied}\spxextra{checklisten.models.HistoricalCheckliste attribute}}

\begin{fulllineitems}
\phantomsection\label{\detokenize{masterCodeDoc:checklisten.models.HistoricalCheckliste.mitglied}}\pysigline{\sphinxbfcode{\sphinxupquote{mitglied}}}
Accessor to the related object on the forward side of a many\sphinxhyphen{}to\sphinxhyphen{}one or
one\sphinxhyphen{}to\sphinxhyphen{}one (via ForwardOneToOneDescriptor subclass) relation.

In the example:

\begin{sphinxVerbatim}[commandchars=\\\{\}]
\PYG{k}{class} \PYG{n+nc}{Child}\PYG{p}{(}\PYG{n}{Model}\PYG{p}{)}\PYG{p}{:}
    \PYG{n}{parent} \PYG{o}{=} \PYG{n}{ForeignKey}\PYG{p}{(}\PYG{n}{Parent}\PYG{p}{,} \PYG{n}{related\PYGZus{}name}\PYG{o}{=}\PYG{l+s+s1}{\PYGZsq{}}\PYG{l+s+s1}{children}\PYG{l+s+s1}{\PYGZsq{}}\PYG{p}{)}
\end{sphinxVerbatim}

\sphinxcode{\sphinxupquote{Child.parent}} is a \sphinxcode{\sphinxupquote{ForwardManyToOneDescriptor}} instance.

\end{fulllineitems}

\index{mitglied\_id (checklisten.models.HistoricalCheckliste attribute)@\spxentry{mitglied\_id}\spxextra{checklisten.models.HistoricalCheckliste attribute}}

\begin{fulllineitems}
\phantomsection\label{\detokenize{masterCodeDoc:checklisten.models.HistoricalCheckliste.mitglied_id}}\pysigline{\sphinxbfcode{\sphinxupquote{mitglied\_id}}}
\end{fulllineitems}

\index{next\_record() (checklisten.models.HistoricalCheckliste property)@\spxentry{next\_record()}\spxextra{checklisten.models.HistoricalCheckliste property}}

\begin{fulllineitems}
\phantomsection\label{\detokenize{masterCodeDoc:checklisten.models.HistoricalCheckliste.next_record}}\pysigline{\sphinxbfcode{\sphinxupquote{property }}\sphinxbfcode{\sphinxupquote{next\_record}}}
Get the next history record for the instance. \sphinxtitleref{None} if last.

\end{fulllineitems}

\index{objects (checklisten.models.HistoricalCheckliste attribute)@\spxentry{objects}\spxextra{checklisten.models.HistoricalCheckliste attribute}}

\begin{fulllineitems}
\phantomsection\label{\detokenize{masterCodeDoc:checklisten.models.HistoricalCheckliste.objects}}\pysigline{\sphinxbfcode{\sphinxupquote{objects}}\sphinxbfcode{\sphinxupquote{ = \textless{}django.db.models.manager.Manager object\textgreater{}}}}
\end{fulllineitems}

\index{prev\_record() (checklisten.models.HistoricalCheckliste property)@\spxentry{prev\_record()}\spxextra{checklisten.models.HistoricalCheckliste property}}

\begin{fulllineitems}
\phantomsection\label{\detokenize{masterCodeDoc:checklisten.models.HistoricalCheckliste.prev_record}}\pysigline{\sphinxbfcode{\sphinxupquote{property }}\sphinxbfcode{\sphinxupquote{prev\_record}}}
Get the previous history record for the instance. \sphinxtitleref{None} if first.

\end{fulllineitems}

\index{revert\_url() (checklisten.models.HistoricalCheckliste method)@\spxentry{revert\_url()}\spxextra{checklisten.models.HistoricalCheckliste method}}

\begin{fulllineitems}
\phantomsection\label{\detokenize{masterCodeDoc:checklisten.models.HistoricalCheckliste.revert_url}}\pysiglinewithargsret{\sphinxbfcode{\sphinxupquote{revert\_url}}}{}{}
URL for this change in the default admin site.

\end{fulllineitems}


\end{fulllineitems}

\index{HistoricalChecklisteAufgabe (class in checklisten.models)@\spxentry{HistoricalChecklisteAufgabe}\spxextra{class in checklisten.models}}

\begin{fulllineitems}
\phantomsection\label{\detokenize{masterCodeDoc:checklisten.models.HistoricalChecklisteAufgabe}}\pysiglinewithargsret{\sphinxbfcode{\sphinxupquote{class }}\sphinxcode{\sphinxupquote{checklisten.models.}}\sphinxbfcode{\sphinxupquote{HistoricalChecklisteAufgabe}}}{\emph{\DUrole{n}{id}}, \emph{\DUrole{n}{abgehakt}}, \emph{\DUrole{n}{checkliste}}, \emph{\DUrole{n}{aufgabe}}, \emph{\DUrole{n}{history\_id}}, \emph{\DUrole{n}{history\_date}}, \emph{\DUrole{n}{history\_change\_reason}}, \emph{\DUrole{n}{history\_type}}, \emph{\DUrole{n}{history\_user}}}{}~\index{HistoricalChecklisteAufgabe.DoesNotExist@\spxentry{HistoricalChecklisteAufgabe.DoesNotExist}}

\begin{fulllineitems}
\phantomsection\label{\detokenize{masterCodeDoc:checklisten.models.HistoricalChecklisteAufgabe.DoesNotExist}}\pysigline{\sphinxbfcode{\sphinxupquote{exception }}\sphinxbfcode{\sphinxupquote{DoesNotExist}}}
\end{fulllineitems}

\index{HistoricalChecklisteAufgabe.MultipleObjectsReturned@\spxentry{HistoricalChecklisteAufgabe.MultipleObjectsReturned}}

\begin{fulllineitems}
\phantomsection\label{\detokenize{masterCodeDoc:checklisten.models.HistoricalChecklisteAufgabe.MultipleObjectsReturned}}\pysigline{\sphinxbfcode{\sphinxupquote{exception }}\sphinxbfcode{\sphinxupquote{MultipleObjectsReturned}}}
\end{fulllineitems}

\index{abgehakt (checklisten.models.HistoricalChecklisteAufgabe attribute)@\spxentry{abgehakt}\spxextra{checklisten.models.HistoricalChecklisteAufgabe attribute}}

\begin{fulllineitems}
\phantomsection\label{\detokenize{masterCodeDoc:checklisten.models.HistoricalChecklisteAufgabe.abgehakt}}\pysigline{\sphinxbfcode{\sphinxupquote{abgehakt}}}
A wrapper for a deferred\sphinxhyphen{}loading field. When the value is read from this
object the first time, the query is executed.

\end{fulllineitems}

\index{aufgabe (checklisten.models.HistoricalChecklisteAufgabe attribute)@\spxentry{aufgabe}\spxextra{checklisten.models.HistoricalChecklisteAufgabe attribute}}

\begin{fulllineitems}
\phantomsection\label{\detokenize{masterCodeDoc:checklisten.models.HistoricalChecklisteAufgabe.aufgabe}}\pysigline{\sphinxbfcode{\sphinxupquote{aufgabe}}}
Accessor to the related object on the forward side of a many\sphinxhyphen{}to\sphinxhyphen{}one or
one\sphinxhyphen{}to\sphinxhyphen{}one (via ForwardOneToOneDescriptor subclass) relation.

In the example:

\begin{sphinxVerbatim}[commandchars=\\\{\}]
\PYG{k}{class} \PYG{n+nc}{Child}\PYG{p}{(}\PYG{n}{Model}\PYG{p}{)}\PYG{p}{:}
    \PYG{n}{parent} \PYG{o}{=} \PYG{n}{ForeignKey}\PYG{p}{(}\PYG{n}{Parent}\PYG{p}{,} \PYG{n}{related\PYGZus{}name}\PYG{o}{=}\PYG{l+s+s1}{\PYGZsq{}}\PYG{l+s+s1}{children}\PYG{l+s+s1}{\PYGZsq{}}\PYG{p}{)}
\end{sphinxVerbatim}

\sphinxcode{\sphinxupquote{Child.parent}} is a \sphinxcode{\sphinxupquote{ForwardManyToOneDescriptor}} instance.

\end{fulllineitems}

\index{aufgabe\_id (checklisten.models.HistoricalChecklisteAufgabe attribute)@\spxentry{aufgabe\_id}\spxextra{checklisten.models.HistoricalChecklisteAufgabe attribute}}

\begin{fulllineitems}
\phantomsection\label{\detokenize{masterCodeDoc:checklisten.models.HistoricalChecklisteAufgabe.aufgabe_id}}\pysigline{\sphinxbfcode{\sphinxupquote{aufgabe\_id}}}
\end{fulllineitems}

\index{checkliste (checklisten.models.HistoricalChecklisteAufgabe attribute)@\spxentry{checkliste}\spxextra{checklisten.models.HistoricalChecklisteAufgabe attribute}}

\begin{fulllineitems}
\phantomsection\label{\detokenize{masterCodeDoc:checklisten.models.HistoricalChecklisteAufgabe.checkliste}}\pysigline{\sphinxbfcode{\sphinxupquote{checkliste}}}
Accessor to the related object on the forward side of a many\sphinxhyphen{}to\sphinxhyphen{}one or
one\sphinxhyphen{}to\sphinxhyphen{}one (via ForwardOneToOneDescriptor subclass) relation.

In the example:

\begin{sphinxVerbatim}[commandchars=\\\{\}]
\PYG{k}{class} \PYG{n+nc}{Child}\PYG{p}{(}\PYG{n}{Model}\PYG{p}{)}\PYG{p}{:}
    \PYG{n}{parent} \PYG{o}{=} \PYG{n}{ForeignKey}\PYG{p}{(}\PYG{n}{Parent}\PYG{p}{,} \PYG{n}{related\PYGZus{}name}\PYG{o}{=}\PYG{l+s+s1}{\PYGZsq{}}\PYG{l+s+s1}{children}\PYG{l+s+s1}{\PYGZsq{}}\PYG{p}{)}
\end{sphinxVerbatim}

\sphinxcode{\sphinxupquote{Child.parent}} is a \sphinxcode{\sphinxupquote{ForwardManyToOneDescriptor}} instance.

\end{fulllineitems}

\index{checkliste\_id (checklisten.models.HistoricalChecklisteAufgabe attribute)@\spxentry{checkliste\_id}\spxextra{checklisten.models.HistoricalChecklisteAufgabe attribute}}

\begin{fulllineitems}
\phantomsection\label{\detokenize{masterCodeDoc:checklisten.models.HistoricalChecklisteAufgabe.checkliste_id}}\pysigline{\sphinxbfcode{\sphinxupquote{checkliste\_id}}}
\end{fulllineitems}

\index{get\_default\_history\_user() (checklisten.models.HistoricalChecklisteAufgabe static method)@\spxentry{get\_default\_history\_user()}\spxextra{checklisten.models.HistoricalChecklisteAufgabe static method}}

\begin{fulllineitems}
\phantomsection\label{\detokenize{masterCodeDoc:checklisten.models.HistoricalChecklisteAufgabe.get_default_history_user}}\pysiglinewithargsret{\sphinxbfcode{\sphinxupquote{static }}\sphinxbfcode{\sphinxupquote{get\_default\_history\_user}}}{\emph{\DUrole{n}{instance}}}{}
Returns the user specified by \sphinxtitleref{get\_user} method for manually creating
historical objects

\end{fulllineitems}

\index{get\_history\_type\_display() (checklisten.models.HistoricalChecklisteAufgabe method)@\spxentry{get\_history\_type\_display()}\spxextra{checklisten.models.HistoricalChecklisteAufgabe method}}

\begin{fulllineitems}
\phantomsection\label{\detokenize{masterCodeDoc:checklisten.models.HistoricalChecklisteAufgabe.get_history_type_display}}\pysiglinewithargsret{\sphinxbfcode{\sphinxupquote{get\_history\_type\_display}}}{\emph{*}, \emph{field=\textless{}django.db.models.fields.CharField: history\_type\textgreater{}}}{}
\end{fulllineitems}

\index{get\_next\_by\_history\_date() (checklisten.models.HistoricalChecklisteAufgabe method)@\spxentry{get\_next\_by\_history\_date()}\spxextra{checklisten.models.HistoricalChecklisteAufgabe method}}

\begin{fulllineitems}
\phantomsection\label{\detokenize{masterCodeDoc:checklisten.models.HistoricalChecklisteAufgabe.get_next_by_history_date}}\pysiglinewithargsret{\sphinxbfcode{\sphinxupquote{get\_next\_by\_history\_date}}}{\emph{*}, \emph{field=\textless{}django.db.models.fields.DateTimeField: history\_date\textgreater{}}, \emph{is\_next=True}, \emph{**kwargs}}{}
\end{fulllineitems}

\index{get\_previous\_by\_history\_date() (checklisten.models.HistoricalChecklisteAufgabe method)@\spxentry{get\_previous\_by\_history\_date()}\spxextra{checklisten.models.HistoricalChecklisteAufgabe method}}

\begin{fulllineitems}
\phantomsection\label{\detokenize{masterCodeDoc:checklisten.models.HistoricalChecklisteAufgabe.get_previous_by_history_date}}\pysiglinewithargsret{\sphinxbfcode{\sphinxupquote{get\_previous\_by\_history\_date}}}{\emph{*}, \emph{field=\textless{}django.db.models.fields.DateTimeField: history\_date\textgreater{}}, \emph{is\_next=False}, \emph{**kwargs}}{}
\end{fulllineitems}

\index{history\_change\_reason (checklisten.models.HistoricalChecklisteAufgabe attribute)@\spxentry{history\_change\_reason}\spxextra{checklisten.models.HistoricalChecklisteAufgabe attribute}}

\begin{fulllineitems}
\phantomsection\label{\detokenize{masterCodeDoc:checklisten.models.HistoricalChecklisteAufgabe.history_change_reason}}\pysigline{\sphinxbfcode{\sphinxupquote{history\_change\_reason}}}
A wrapper for a deferred\sphinxhyphen{}loading field. When the value is read from this
object the first time, the query is executed.

\end{fulllineitems}

\index{history\_date (checklisten.models.HistoricalChecklisteAufgabe attribute)@\spxentry{history\_date}\spxextra{checklisten.models.HistoricalChecklisteAufgabe attribute}}

\begin{fulllineitems}
\phantomsection\label{\detokenize{masterCodeDoc:checklisten.models.HistoricalChecklisteAufgabe.history_date}}\pysigline{\sphinxbfcode{\sphinxupquote{history\_date}}}
A wrapper for a deferred\sphinxhyphen{}loading field. When the value is read from this
object the first time, the query is executed.

\end{fulllineitems}

\index{history\_id (checklisten.models.HistoricalChecklisteAufgabe attribute)@\spxentry{history\_id}\spxextra{checklisten.models.HistoricalChecklisteAufgabe attribute}}

\begin{fulllineitems}
\phantomsection\label{\detokenize{masterCodeDoc:checklisten.models.HistoricalChecklisteAufgabe.history_id}}\pysigline{\sphinxbfcode{\sphinxupquote{history\_id}}}
A wrapper for a deferred\sphinxhyphen{}loading field. When the value is read from this
object the first time, the query is executed.

\end{fulllineitems}

\index{history\_object (checklisten.models.HistoricalChecklisteAufgabe attribute)@\spxentry{history\_object}\spxextra{checklisten.models.HistoricalChecklisteAufgabe attribute}}

\begin{fulllineitems}
\phantomsection\label{\detokenize{masterCodeDoc:checklisten.models.HistoricalChecklisteAufgabe.history_object}}\pysigline{\sphinxbfcode{\sphinxupquote{history\_object}}}
\end{fulllineitems}

\index{history\_type (checklisten.models.HistoricalChecklisteAufgabe attribute)@\spxentry{history\_type}\spxextra{checklisten.models.HistoricalChecklisteAufgabe attribute}}

\begin{fulllineitems}
\phantomsection\label{\detokenize{masterCodeDoc:checklisten.models.HistoricalChecklisteAufgabe.history_type}}\pysigline{\sphinxbfcode{\sphinxupquote{history\_type}}}
A wrapper for a deferred\sphinxhyphen{}loading field. When the value is read from this
object the first time, the query is executed.

\end{fulllineitems}

\index{history\_user (checklisten.models.HistoricalChecklisteAufgabe attribute)@\spxentry{history\_user}\spxextra{checklisten.models.HistoricalChecklisteAufgabe attribute}}

\begin{fulllineitems}
\phantomsection\label{\detokenize{masterCodeDoc:checklisten.models.HistoricalChecklisteAufgabe.history_user}}\pysigline{\sphinxbfcode{\sphinxupquote{history\_user}}}
Accessor to the related object on the forward side of a many\sphinxhyphen{}to\sphinxhyphen{}one or
one\sphinxhyphen{}to\sphinxhyphen{}one (via ForwardOneToOneDescriptor subclass) relation.

In the example:

\begin{sphinxVerbatim}[commandchars=\\\{\}]
\PYG{k}{class} \PYG{n+nc}{Child}\PYG{p}{(}\PYG{n}{Model}\PYG{p}{)}\PYG{p}{:}
    \PYG{n}{parent} \PYG{o}{=} \PYG{n}{ForeignKey}\PYG{p}{(}\PYG{n}{Parent}\PYG{p}{,} \PYG{n}{related\PYGZus{}name}\PYG{o}{=}\PYG{l+s+s1}{\PYGZsq{}}\PYG{l+s+s1}{children}\PYG{l+s+s1}{\PYGZsq{}}\PYG{p}{)}
\end{sphinxVerbatim}

\sphinxcode{\sphinxupquote{Child.parent}} is a \sphinxcode{\sphinxupquote{ForwardManyToOneDescriptor}} instance.

\end{fulllineitems}

\index{history\_user\_id (checklisten.models.HistoricalChecklisteAufgabe attribute)@\spxentry{history\_user\_id}\spxextra{checklisten.models.HistoricalChecklisteAufgabe attribute}}

\begin{fulllineitems}
\phantomsection\label{\detokenize{masterCodeDoc:checklisten.models.HistoricalChecklisteAufgabe.history_user_id}}\pysigline{\sphinxbfcode{\sphinxupquote{history\_user\_id}}}
\end{fulllineitems}

\index{id (checklisten.models.HistoricalChecklisteAufgabe attribute)@\spxentry{id}\spxextra{checklisten.models.HistoricalChecklisteAufgabe attribute}}

\begin{fulllineitems}
\phantomsection\label{\detokenize{masterCodeDoc:checklisten.models.HistoricalChecklisteAufgabe.id}}\pysigline{\sphinxbfcode{\sphinxupquote{id}}}
A wrapper for a deferred\sphinxhyphen{}loading field. When the value is read from this
object the first time, the query is executed.

\end{fulllineitems}

\index{instance() (checklisten.models.HistoricalChecklisteAufgabe property)@\spxentry{instance()}\spxextra{checklisten.models.HistoricalChecklisteAufgabe property}}

\begin{fulllineitems}
\phantomsection\label{\detokenize{masterCodeDoc:checklisten.models.HistoricalChecklisteAufgabe.instance}}\pysigline{\sphinxbfcode{\sphinxupquote{property }}\sphinxbfcode{\sphinxupquote{instance}}}
\end{fulllineitems}

\index{instance\_type (checklisten.models.HistoricalChecklisteAufgabe attribute)@\spxentry{instance\_type}\spxextra{checklisten.models.HistoricalChecklisteAufgabe attribute}}

\begin{fulllineitems}
\phantomsection\label{\detokenize{masterCodeDoc:checklisten.models.HistoricalChecklisteAufgabe.instance_type}}\pysigline{\sphinxbfcode{\sphinxupquote{instance\_type}}}
alias of {\hyperref[\detokenize{masterCodeDoc:checklisten.models.ChecklisteAufgabe}]{\sphinxcrossref{\sphinxcode{\sphinxupquote{ChecklisteAufgabe}}}}}

\end{fulllineitems}

\index{next\_record() (checklisten.models.HistoricalChecklisteAufgabe property)@\spxentry{next\_record()}\spxextra{checklisten.models.HistoricalChecklisteAufgabe property}}

\begin{fulllineitems}
\phantomsection\label{\detokenize{masterCodeDoc:checklisten.models.HistoricalChecklisteAufgabe.next_record}}\pysigline{\sphinxbfcode{\sphinxupquote{property }}\sphinxbfcode{\sphinxupquote{next\_record}}}
Get the next history record for the instance. \sphinxtitleref{None} if last.

\end{fulllineitems}

\index{objects (checklisten.models.HistoricalChecklisteAufgabe attribute)@\spxentry{objects}\spxextra{checklisten.models.HistoricalChecklisteAufgabe attribute}}

\begin{fulllineitems}
\phantomsection\label{\detokenize{masterCodeDoc:checklisten.models.HistoricalChecklisteAufgabe.objects}}\pysigline{\sphinxbfcode{\sphinxupquote{objects}}\sphinxbfcode{\sphinxupquote{ = \textless{}django.db.models.manager.Manager object\textgreater{}}}}
\end{fulllineitems}

\index{prev\_record() (checklisten.models.HistoricalChecklisteAufgabe property)@\spxentry{prev\_record()}\spxextra{checklisten.models.HistoricalChecklisteAufgabe property}}

\begin{fulllineitems}
\phantomsection\label{\detokenize{masterCodeDoc:checklisten.models.HistoricalChecklisteAufgabe.prev_record}}\pysigline{\sphinxbfcode{\sphinxupquote{property }}\sphinxbfcode{\sphinxupquote{prev\_record}}}
Get the previous history record for the instance. \sphinxtitleref{None} if first.

\end{fulllineitems}

\index{revert\_url() (checklisten.models.HistoricalChecklisteAufgabe method)@\spxentry{revert\_url()}\spxextra{checklisten.models.HistoricalChecklisteAufgabe method}}

\begin{fulllineitems}
\phantomsection\label{\detokenize{masterCodeDoc:checklisten.models.HistoricalChecklisteAufgabe.revert_url}}\pysiglinewithargsret{\sphinxbfcode{\sphinxupquote{revert\_url}}}{}{}
URL for this change in the default admin site.

\end{fulllineitems}


\end{fulllineitems}

\index{HistoricalChecklisteRecht (class in checklisten.models)@\spxentry{HistoricalChecklisteRecht}\spxextra{class in checklisten.models}}

\begin{fulllineitems}
\phantomsection\label{\detokenize{masterCodeDoc:checklisten.models.HistoricalChecklisteRecht}}\pysiglinewithargsret{\sphinxbfcode{\sphinxupquote{class }}\sphinxcode{\sphinxupquote{checklisten.models.}}\sphinxbfcode{\sphinxupquote{HistoricalChecklisteRecht}}}{\emph{\DUrole{n}{id}}, \emph{\DUrole{n}{abgehakt}}, \emph{\DUrole{n}{checkliste}}, \emph{\DUrole{n}{recht}}, \emph{\DUrole{n}{history\_id}}, \emph{\DUrole{n}{history\_date}}, \emph{\DUrole{n}{history\_change\_reason}}, \emph{\DUrole{n}{history\_type}}, \emph{\DUrole{n}{history\_user}}}{}~\index{HistoricalChecklisteRecht.DoesNotExist@\spxentry{HistoricalChecklisteRecht.DoesNotExist}}

\begin{fulllineitems}
\phantomsection\label{\detokenize{masterCodeDoc:checklisten.models.HistoricalChecklisteRecht.DoesNotExist}}\pysigline{\sphinxbfcode{\sphinxupquote{exception }}\sphinxbfcode{\sphinxupquote{DoesNotExist}}}
\end{fulllineitems}

\index{HistoricalChecklisteRecht.MultipleObjectsReturned@\spxentry{HistoricalChecklisteRecht.MultipleObjectsReturned}}

\begin{fulllineitems}
\phantomsection\label{\detokenize{masterCodeDoc:checklisten.models.HistoricalChecklisteRecht.MultipleObjectsReturned}}\pysigline{\sphinxbfcode{\sphinxupquote{exception }}\sphinxbfcode{\sphinxupquote{MultipleObjectsReturned}}}
\end{fulllineitems}

\index{abgehakt (checklisten.models.HistoricalChecklisteRecht attribute)@\spxentry{abgehakt}\spxextra{checklisten.models.HistoricalChecklisteRecht attribute}}

\begin{fulllineitems}
\phantomsection\label{\detokenize{masterCodeDoc:checklisten.models.HistoricalChecklisteRecht.abgehakt}}\pysigline{\sphinxbfcode{\sphinxupquote{abgehakt}}}
A wrapper for a deferred\sphinxhyphen{}loading field. When the value is read from this
object the first time, the query is executed.

\end{fulllineitems}

\index{checkliste (checklisten.models.HistoricalChecklisteRecht attribute)@\spxentry{checkliste}\spxextra{checklisten.models.HistoricalChecklisteRecht attribute}}

\begin{fulllineitems}
\phantomsection\label{\detokenize{masterCodeDoc:checklisten.models.HistoricalChecklisteRecht.checkliste}}\pysigline{\sphinxbfcode{\sphinxupquote{checkliste}}}
Accessor to the related object on the forward side of a many\sphinxhyphen{}to\sphinxhyphen{}one or
one\sphinxhyphen{}to\sphinxhyphen{}one (via ForwardOneToOneDescriptor subclass) relation.

In the example:

\begin{sphinxVerbatim}[commandchars=\\\{\}]
\PYG{k}{class} \PYG{n+nc}{Child}\PYG{p}{(}\PYG{n}{Model}\PYG{p}{)}\PYG{p}{:}
    \PYG{n}{parent} \PYG{o}{=} \PYG{n}{ForeignKey}\PYG{p}{(}\PYG{n}{Parent}\PYG{p}{,} \PYG{n}{related\PYGZus{}name}\PYG{o}{=}\PYG{l+s+s1}{\PYGZsq{}}\PYG{l+s+s1}{children}\PYG{l+s+s1}{\PYGZsq{}}\PYG{p}{)}
\end{sphinxVerbatim}

\sphinxcode{\sphinxupquote{Child.parent}} is a \sphinxcode{\sphinxupquote{ForwardManyToOneDescriptor}} instance.

\end{fulllineitems}

\index{checkliste\_id (checklisten.models.HistoricalChecklisteRecht attribute)@\spxentry{checkliste\_id}\spxextra{checklisten.models.HistoricalChecklisteRecht attribute}}

\begin{fulllineitems}
\phantomsection\label{\detokenize{masterCodeDoc:checklisten.models.HistoricalChecklisteRecht.checkliste_id}}\pysigline{\sphinxbfcode{\sphinxupquote{checkliste\_id}}}
\end{fulllineitems}

\index{get\_default\_history\_user() (checklisten.models.HistoricalChecklisteRecht static method)@\spxentry{get\_default\_history\_user()}\spxextra{checklisten.models.HistoricalChecklisteRecht static method}}

\begin{fulllineitems}
\phantomsection\label{\detokenize{masterCodeDoc:checklisten.models.HistoricalChecklisteRecht.get_default_history_user}}\pysiglinewithargsret{\sphinxbfcode{\sphinxupquote{static }}\sphinxbfcode{\sphinxupquote{get\_default\_history\_user}}}{\emph{\DUrole{n}{instance}}}{}
Returns the user specified by \sphinxtitleref{get\_user} method for manually creating
historical objects

\end{fulllineitems}

\index{get\_history\_type\_display() (checklisten.models.HistoricalChecklisteRecht method)@\spxentry{get\_history\_type\_display()}\spxextra{checklisten.models.HistoricalChecklisteRecht method}}

\begin{fulllineitems}
\phantomsection\label{\detokenize{masterCodeDoc:checklisten.models.HistoricalChecklisteRecht.get_history_type_display}}\pysiglinewithargsret{\sphinxbfcode{\sphinxupquote{get\_history\_type\_display}}}{\emph{*}, \emph{field=\textless{}django.db.models.fields.CharField: history\_type\textgreater{}}}{}
\end{fulllineitems}

\index{get\_next\_by\_history\_date() (checklisten.models.HistoricalChecklisteRecht method)@\spxentry{get\_next\_by\_history\_date()}\spxextra{checklisten.models.HistoricalChecklisteRecht method}}

\begin{fulllineitems}
\phantomsection\label{\detokenize{masterCodeDoc:checklisten.models.HistoricalChecklisteRecht.get_next_by_history_date}}\pysiglinewithargsret{\sphinxbfcode{\sphinxupquote{get\_next\_by\_history\_date}}}{\emph{*}, \emph{field=\textless{}django.db.models.fields.DateTimeField: history\_date\textgreater{}}, \emph{is\_next=True}, \emph{**kwargs}}{}
\end{fulllineitems}

\index{get\_previous\_by\_history\_date() (checklisten.models.HistoricalChecklisteRecht method)@\spxentry{get\_previous\_by\_history\_date()}\spxextra{checklisten.models.HistoricalChecklisteRecht method}}

\begin{fulllineitems}
\phantomsection\label{\detokenize{masterCodeDoc:checklisten.models.HistoricalChecklisteRecht.get_previous_by_history_date}}\pysiglinewithargsret{\sphinxbfcode{\sphinxupquote{get\_previous\_by\_history\_date}}}{\emph{*}, \emph{field=\textless{}django.db.models.fields.DateTimeField: history\_date\textgreater{}}, \emph{is\_next=False}, \emph{**kwargs}}{}
\end{fulllineitems}

\index{history\_change\_reason (checklisten.models.HistoricalChecklisteRecht attribute)@\spxentry{history\_change\_reason}\spxextra{checklisten.models.HistoricalChecklisteRecht attribute}}

\begin{fulllineitems}
\phantomsection\label{\detokenize{masterCodeDoc:checklisten.models.HistoricalChecklisteRecht.history_change_reason}}\pysigline{\sphinxbfcode{\sphinxupquote{history\_change\_reason}}}
A wrapper for a deferred\sphinxhyphen{}loading field. When the value is read from this
object the first time, the query is executed.

\end{fulllineitems}

\index{history\_date (checklisten.models.HistoricalChecklisteRecht attribute)@\spxentry{history\_date}\spxextra{checklisten.models.HistoricalChecklisteRecht attribute}}

\begin{fulllineitems}
\phantomsection\label{\detokenize{masterCodeDoc:checklisten.models.HistoricalChecklisteRecht.history_date}}\pysigline{\sphinxbfcode{\sphinxupquote{history\_date}}}
A wrapper for a deferred\sphinxhyphen{}loading field. When the value is read from this
object the first time, the query is executed.

\end{fulllineitems}

\index{history\_id (checklisten.models.HistoricalChecklisteRecht attribute)@\spxentry{history\_id}\spxextra{checklisten.models.HistoricalChecklisteRecht attribute}}

\begin{fulllineitems}
\phantomsection\label{\detokenize{masterCodeDoc:checklisten.models.HistoricalChecklisteRecht.history_id}}\pysigline{\sphinxbfcode{\sphinxupquote{history\_id}}}
A wrapper for a deferred\sphinxhyphen{}loading field. When the value is read from this
object the first time, the query is executed.

\end{fulllineitems}

\index{history\_object (checklisten.models.HistoricalChecklisteRecht attribute)@\spxentry{history\_object}\spxextra{checklisten.models.HistoricalChecklisteRecht attribute}}

\begin{fulllineitems}
\phantomsection\label{\detokenize{masterCodeDoc:checklisten.models.HistoricalChecklisteRecht.history_object}}\pysigline{\sphinxbfcode{\sphinxupquote{history\_object}}}
\end{fulllineitems}

\index{history\_type (checklisten.models.HistoricalChecklisteRecht attribute)@\spxentry{history\_type}\spxextra{checklisten.models.HistoricalChecklisteRecht attribute}}

\begin{fulllineitems}
\phantomsection\label{\detokenize{masterCodeDoc:checklisten.models.HistoricalChecklisteRecht.history_type}}\pysigline{\sphinxbfcode{\sphinxupquote{history\_type}}}
A wrapper for a deferred\sphinxhyphen{}loading field. When the value is read from this
object the first time, the query is executed.

\end{fulllineitems}

\index{history\_user (checklisten.models.HistoricalChecklisteRecht attribute)@\spxentry{history\_user}\spxextra{checklisten.models.HistoricalChecklisteRecht attribute}}

\begin{fulllineitems}
\phantomsection\label{\detokenize{masterCodeDoc:checklisten.models.HistoricalChecklisteRecht.history_user}}\pysigline{\sphinxbfcode{\sphinxupquote{history\_user}}}
Accessor to the related object on the forward side of a many\sphinxhyphen{}to\sphinxhyphen{}one or
one\sphinxhyphen{}to\sphinxhyphen{}one (via ForwardOneToOneDescriptor subclass) relation.

In the example:

\begin{sphinxVerbatim}[commandchars=\\\{\}]
\PYG{k}{class} \PYG{n+nc}{Child}\PYG{p}{(}\PYG{n}{Model}\PYG{p}{)}\PYG{p}{:}
    \PYG{n}{parent} \PYG{o}{=} \PYG{n}{ForeignKey}\PYG{p}{(}\PYG{n}{Parent}\PYG{p}{,} \PYG{n}{related\PYGZus{}name}\PYG{o}{=}\PYG{l+s+s1}{\PYGZsq{}}\PYG{l+s+s1}{children}\PYG{l+s+s1}{\PYGZsq{}}\PYG{p}{)}
\end{sphinxVerbatim}

\sphinxcode{\sphinxupquote{Child.parent}} is a \sphinxcode{\sphinxupquote{ForwardManyToOneDescriptor}} instance.

\end{fulllineitems}

\index{history\_user\_id (checklisten.models.HistoricalChecklisteRecht attribute)@\spxentry{history\_user\_id}\spxextra{checklisten.models.HistoricalChecklisteRecht attribute}}

\begin{fulllineitems}
\phantomsection\label{\detokenize{masterCodeDoc:checklisten.models.HistoricalChecklisteRecht.history_user_id}}\pysigline{\sphinxbfcode{\sphinxupquote{history\_user\_id}}}
\end{fulllineitems}

\index{id (checklisten.models.HistoricalChecklisteRecht attribute)@\spxentry{id}\spxextra{checklisten.models.HistoricalChecklisteRecht attribute}}

\begin{fulllineitems}
\phantomsection\label{\detokenize{masterCodeDoc:checklisten.models.HistoricalChecklisteRecht.id}}\pysigline{\sphinxbfcode{\sphinxupquote{id}}}
A wrapper for a deferred\sphinxhyphen{}loading field. When the value is read from this
object the first time, the query is executed.

\end{fulllineitems}

\index{instance() (checklisten.models.HistoricalChecklisteRecht property)@\spxentry{instance()}\spxextra{checklisten.models.HistoricalChecklisteRecht property}}

\begin{fulllineitems}
\phantomsection\label{\detokenize{masterCodeDoc:checklisten.models.HistoricalChecklisteRecht.instance}}\pysigline{\sphinxbfcode{\sphinxupquote{property }}\sphinxbfcode{\sphinxupquote{instance}}}
\end{fulllineitems}

\index{instance\_type (checklisten.models.HistoricalChecklisteRecht attribute)@\spxentry{instance\_type}\spxextra{checklisten.models.HistoricalChecklisteRecht attribute}}

\begin{fulllineitems}
\phantomsection\label{\detokenize{masterCodeDoc:checklisten.models.HistoricalChecklisteRecht.instance_type}}\pysigline{\sphinxbfcode{\sphinxupquote{instance\_type}}}
alias of {\hyperref[\detokenize{masterCodeDoc:checklisten.models.ChecklisteRecht}]{\sphinxcrossref{\sphinxcode{\sphinxupquote{ChecklisteRecht}}}}}

\end{fulllineitems}

\index{next\_record() (checklisten.models.HistoricalChecklisteRecht property)@\spxentry{next\_record()}\spxextra{checklisten.models.HistoricalChecklisteRecht property}}

\begin{fulllineitems}
\phantomsection\label{\detokenize{masterCodeDoc:checklisten.models.HistoricalChecklisteRecht.next_record}}\pysigline{\sphinxbfcode{\sphinxupquote{property }}\sphinxbfcode{\sphinxupquote{next\_record}}}
Get the next history record for the instance. \sphinxtitleref{None} if last.

\end{fulllineitems}

\index{objects (checklisten.models.HistoricalChecklisteRecht attribute)@\spxentry{objects}\spxextra{checklisten.models.HistoricalChecklisteRecht attribute}}

\begin{fulllineitems}
\phantomsection\label{\detokenize{masterCodeDoc:checklisten.models.HistoricalChecklisteRecht.objects}}\pysigline{\sphinxbfcode{\sphinxupquote{objects}}\sphinxbfcode{\sphinxupquote{ = \textless{}django.db.models.manager.Manager object\textgreater{}}}}
\end{fulllineitems}

\index{prev\_record() (checklisten.models.HistoricalChecklisteRecht property)@\spxentry{prev\_record()}\spxextra{checklisten.models.HistoricalChecklisteRecht property}}

\begin{fulllineitems}
\phantomsection\label{\detokenize{masterCodeDoc:checklisten.models.HistoricalChecklisteRecht.prev_record}}\pysigline{\sphinxbfcode{\sphinxupquote{property }}\sphinxbfcode{\sphinxupquote{prev\_record}}}
Get the previous history record for the instance. \sphinxtitleref{None} if first.

\end{fulllineitems}

\index{recht (checklisten.models.HistoricalChecklisteRecht attribute)@\spxentry{recht}\spxextra{checklisten.models.HistoricalChecklisteRecht attribute}}

\begin{fulllineitems}
\phantomsection\label{\detokenize{masterCodeDoc:checklisten.models.HistoricalChecklisteRecht.recht}}\pysigline{\sphinxbfcode{\sphinxupquote{recht}}}
Accessor to the related object on the forward side of a many\sphinxhyphen{}to\sphinxhyphen{}one or
one\sphinxhyphen{}to\sphinxhyphen{}one (via ForwardOneToOneDescriptor subclass) relation.

In the example:

\begin{sphinxVerbatim}[commandchars=\\\{\}]
\PYG{k}{class} \PYG{n+nc}{Child}\PYG{p}{(}\PYG{n}{Model}\PYG{p}{)}\PYG{p}{:}
    \PYG{n}{parent} \PYG{o}{=} \PYG{n}{ForeignKey}\PYG{p}{(}\PYG{n}{Parent}\PYG{p}{,} \PYG{n}{related\PYGZus{}name}\PYG{o}{=}\PYG{l+s+s1}{\PYGZsq{}}\PYG{l+s+s1}{children}\PYG{l+s+s1}{\PYGZsq{}}\PYG{p}{)}
\end{sphinxVerbatim}

\sphinxcode{\sphinxupquote{Child.parent}} is a \sphinxcode{\sphinxupquote{ForwardManyToOneDescriptor}} instance.

\end{fulllineitems}

\index{recht\_id (checklisten.models.HistoricalChecklisteRecht attribute)@\spxentry{recht\_id}\spxextra{checklisten.models.HistoricalChecklisteRecht attribute}}

\begin{fulllineitems}
\phantomsection\label{\detokenize{masterCodeDoc:checklisten.models.HistoricalChecklisteRecht.recht_id}}\pysigline{\sphinxbfcode{\sphinxupquote{recht\_id}}}
\end{fulllineitems}

\index{revert\_url() (checklisten.models.HistoricalChecklisteRecht method)@\spxentry{revert\_url()}\spxextra{checklisten.models.HistoricalChecklisteRecht method}}

\begin{fulllineitems}
\phantomsection\label{\detokenize{masterCodeDoc:checklisten.models.HistoricalChecklisteRecht.revert_url}}\pysiglinewithargsret{\sphinxbfcode{\sphinxupquote{revert\_url}}}{}{}
URL for this change in the default admin site.

\end{fulllineitems}


\end{fulllineitems}



\subsection{Views}
\label{\detokenize{masterCodeDoc:id10}}\phantomsection\label{\detokenize{masterCodeDoc:module-checklisten.views}}\index{module@\spxentry{module}!checklisten.views@\spxentry{checklisten.views}}\index{checklisten.views@\spxentry{checklisten.views}!module@\spxentry{module}}\index{abhaken() (in module checklisten.views)@\spxentry{abhaken()}\spxextra{in module checklisten.views}}

\begin{fulllineitems}
\phantomsection\label{\detokenize{masterCodeDoc:checklisten.views.abhaken}}\pysiglinewithargsret{\sphinxcode{\sphinxupquote{checklisten.views.}}\sphinxbfcode{\sphinxupquote{abhaken}}}{\emph{\DUrole{n}{request}}}{}
This view is responsible for checking or unchecking a task from the checklist.
It first checks if the user is allowed to check a task, i.e. is auhenticated and superuser.
Next, the parameters task\_type and task\_id are fetched from the request. They are used to determine if an Aufgabe or a Recht was checked, and to get the correct task.
Finally, the task is checked or unchecked depending on its current state and the changes are saved.
\begin{quote}\begin{description}
\item[{Parameters}] \leavevmode
\sphinxstyleliteralstrong{\sphinxupquote{request}} \textendash{} The HTTP request that triggered the view, including parameters task\_type and task\_id.

\item[{Returns}] \leavevmode
An empty HttpResponse if the operation was successful.

\item[{Returns}] \leavevmode
An HttpResponse indicitang the error if an error has occured or if the user is not allowed to perform the operation.

\end{description}\end{quote}

\end{fulllineitems}

\index{erstellen() (in module checklisten.views)@\spxentry{erstellen()}\spxextra{in module checklisten.views}}

\begin{fulllineitems}
\phantomsection\label{\detokenize{masterCodeDoc:checklisten.views.erstellen}}\pysiglinewithargsret{\sphinxcode{\sphinxupquote{checklisten.views.}}\sphinxbfcode{\sphinxupquote{erstellen}}}{\emph{\DUrole{n}{request}}}{}
This view is responsible for creating a new checklist.
It first checks if the user is allowed to create a new checklist (i.e. is authenticated and superuser).
Next, the Mitglied’s and Funktion’s IDs as well as whether general tasks shall be included are fetched from the request.
The view then tries to find the Mitglied and Funktion with the specified ID, and returns an error message if at least one of them could not be found.
After that, the view checks if a checklist for this Mitglied and Funktion already exists. If that is the case, an error message is shown to the user.
Finally, the new checklist is created and all Aufgaben and Rechte are added to it according to the specified parameters in the request.
\begin{quote}\begin{description}
\item[{Parameters}] \leavevmode
\sphinxstyleliteralstrong{\sphinxupquote{request}} \textendash{} The HTTP request that triggered the view, including parameters mitgliedSelect, funktionSelect and generalTasksCheckbox.

\item[{Returns}] \leavevmode
A HttpResponse, if an error has occured, indicating the error to the user.

\item[{Returns}] \leavevmode
A redirect to /checklisten if creating the checklist was successful or a checklist for the same Mitglied and Funktion already exists.

\end{description}\end{quote}

\end{fulllineitems}

\index{get\_funktionen() (in module checklisten.views)@\spxentry{get\_funktionen()}\spxextra{in module checklisten.views}}

\begin{fulllineitems}
\phantomsection\label{\detokenize{masterCodeDoc:checklisten.views.get_funktionen}}\pysiglinewithargsret{\sphinxcode{\sphinxupquote{checklisten.views.}}\sphinxbfcode{\sphinxupquote{get\_funktionen}}}{\emph{\DUrole{n}{request}}}{}
This view is responsible for getting all Funktionen associated to a Mitglied that was selected in the create checklist modal.
First, it checks if the user is allowed to get a list of Funktionen for a Mitglied in this context (i.e. the user is authenticated and superuser).
Next, all Funktionen for the specified mitglied\_id are determined and returned through a rendered template of select options to populate the dropdown in the create checklist modal.
\begin{quote}\begin{description}
\item[{Parameters}] \leavevmode
\sphinxstyleliteralstrong{\sphinxupquote{request}} \textendash{} The HTTP request that triggered the view, including the mitglied\_id to get the Funktionen for.

\item[{Returns}] \leavevmode
An HttpResponse indicating the error if an error occured.

\item[{Returns}] \leavevmode
The rendered select options to populate the dropdown with if everything was successful.

\end{description}\end{quote}

\end{fulllineitems}

\index{loeschen() (in module checklisten.views)@\spxentry{loeschen()}\spxextra{in module checklisten.views}}

\begin{fulllineitems}
\phantomsection\label{\detokenize{masterCodeDoc:checklisten.views.loeschen}}\pysiglinewithargsret{\sphinxcode{\sphinxupquote{checklisten.views.}}\sphinxbfcode{\sphinxupquote{loeschen}}}{\emph{\DUrole{n}{request}}}{}
This view is responsible for deleting an existing checklist.
First, it checks whether the user is allowed to delete the checklist (i.e. is authenticated and superuser).
Next, the checkliste specified in the request’s parameter checkliste\_id is deleted. 
Since all ForeignKey relations in other models are set to cascade if the checklist is deleted (i.e. in ChecklisteRecht and AufgabeRecht), only the checklist itself needs to be deleted explicitly.
\begin{quote}\begin{description}
\item[{Parameters}] \leavevmode
\sphinxstyleliteralstrong{\sphinxupquote{request}} \textendash{} The HTTP request that triggered the view, including parameter checkliste\_id.

\item[{Returns}] \leavevmode
An empty HttpResponse if deleting the checklist was successful.

\item[{Returns}] \leavevmode
An HttpResponse indicating an error if one occured or a user is not allowed to to delete a checklist.

\end{description}\end{quote}

\end{fulllineitems}

\index{main\_screen() (in module checklisten.views)@\spxentry{main\_screen()}\spxextra{in module checklisten.views}}

\begin{fulllineitems}
\phantomsection\label{\detokenize{masterCodeDoc:checklisten.views.main_screen}}\pysiglinewithargsret{\sphinxcode{\sphinxupquote{checklisten.views.}}\sphinxbfcode{\sphinxupquote{main\_screen}}}{\emph{\DUrole{n}{request}}}{}
This view renders all existing checklists. Furthermore, it provides the 20 latest Mitglieder to the modal used for creating a new checklist.
If the user is not authenticated, an error message will be displayed and the user is redirected to the login page.
\begin{quote}\begin{description}
\item[{Parameters}] \leavevmode
\sphinxstyleliteralstrong{\sphinxupquote{request}} \textendash{} The HTTP request that triggered the view.

\item[{Returns}] \leavevmode
The rendered main\_screen view, or a redirect to the login page if the user is not authenticated.

\end{description}\end{quote}

\end{fulllineitems}



\subsection{Templates}
\label{\detokenize{masterCodeDoc:id11}}
All templates can be found under \sphinxcode{\sphinxupquote{checklisten/templates/checklisten}}.


\subsubsection{main\_screen.html}
\label{\detokenize{masterCodeDoc:id12}}
This template contains the main page of the app. Each Checkliste is represented by a card and consists of three main elements:
\begin{itemize}
\item {} 
The header contains the name of the Mitglied and, if specified, the Funktion that the Checkliste was created for.

\item {} 
The general tasks section only exists if specified while creating the Checkliste and displays all ChecklisteAufgabe objects associated with the Checkliste.

\item {} 
The permissions section only exists if a Funktion was selected for this Checkliste and displays all CheckllisteRecht objects associated with the Checkliste.

\end{itemize}

Tasks can only be checked if the user is administrator; otherwise, the checkboxes are disabled. The same goes for creating a new Checkliste; otherwise, the create button is not visible.
The template also contains JavaScript methods used to make AJAX requests to the server, e.g. to complete task or delete a Checkliste. Refer to the source code comments for more information.


\subsubsection{\_createModal.html}
\label{\detokenize{masterCodeDoc:createmodal-html}}
This template contains the modal that is shown when creating a new Checkliste. The modal includes:
\begin{itemize}
\item {} 
A dropdown to select the Mitglied that the Checkliste shall be created for. Please not that only the latest 20 Mitglieder are shown to improve performance.

\item {} 
A dropdown to select the Funktion that the Checkliste shall be created for. It only contains Funktionen belonging to the Mitglied selected beforehand.

\item {} 
A checkbox to determine whether the set of general tasks (Aufgaben) shall be included in the Checkliste. Will be force\sphinxhyphen{}checked and disabled if no Funktion is selected to prevent empty checklists from being created.

\end{itemize}

The template also contains JavaScript methods to populate the Funktionen dropdown via an AJAX request to the server, as well as to correctly disable and check the general tasks checkbox if no Funktion was selected. Refer to the source code comments for more information.


\subsubsection{\_deleteModal.html}
\label{\detokenize{masterCodeDoc:deletemodal-html}}
This template only contains a confirmation modal if the user is trying to delete a Checkliste to prevent accidental deletions.


\subsubsection{\_funktionSelectOptions.html}
\label{\detokenize{masterCodeDoc:funktionselectoptions-html}}
This template is used to generate the Funktion dropdown options for the create modal. It is only used by the \sphinxcode{\sphinxupquote{get\_funktionen}} view and is not referenced directly in other templates.


\subsection{Template Tags}
\label{\detokenize{masterCodeDoc:id13}}\phantomsection\label{\detokenize{masterCodeDoc:module-checklisten.templatetags.t_checklisten.get_perms}}\index{module@\spxentry{module}!checklisten.templatetags.t\_checklisten.get\_perms@\spxentry{checklisten.templatetags.t\_checklisten.get\_perms}}\index{checklisten.templatetags.t\_checklisten.get\_perms@\spxentry{checklisten.templatetags.t\_checklisten.get\_perms}!module@\spxentry{module}}\index{get\_perms() (in module checklisten.templatetags.t\_checklisten.get\_perms)@\spxentry{get\_perms()}\spxextra{in module checklisten.templatetags.t\_checklisten.get\_perms}}

\begin{fulllineitems}
\phantomsection\label{\detokenize{masterCodeDoc:checklisten.templatetags.t_checklisten.get_perms.get_perms}}\pysiglinewithargsret{\sphinxcode{\sphinxupquote{checklisten.templatetags.t\_checklisten.get\_perms.}}\sphinxbfcode{\sphinxupquote{get\_perms}}}{\emph{\DUrole{n}{checklist\_id}}}{}
Gets all ChecklisteRecht objects that belong to the Checkliste identified by checklist\_id.
\begin{quote}\begin{description}
\item[{Parameters}] \leavevmode
\sphinxstyleliteralstrong{\sphinxupquote{checklist\_id}} (\sphinxstyleliteralemphasis{\sphinxupquote{int}}) \textendash{} The ID of the Checkliste to fetch the ChecklisteRecht objects for.

\item[{Returns}] \leavevmode
A QuerySet containing all ChecklisteRecht objects for the specified Checkliste.

\item[{Return type}] \leavevmode
QuerySet

\end{description}\end{quote}

\end{fulllineitems}

\phantomsection\label{\detokenize{masterCodeDoc:module-checklisten.templatetags.t_checklisten.get_tasks}}\index{module@\spxentry{module}!checklisten.templatetags.t\_checklisten.get\_tasks@\spxentry{checklisten.templatetags.t\_checklisten.get\_tasks}}\index{checklisten.templatetags.t\_checklisten.get\_tasks@\spxentry{checklisten.templatetags.t\_checklisten.get\_tasks}!module@\spxentry{module}}\index{get\_tasks() (in module checklisten.templatetags.t\_checklisten.get\_tasks)@\spxentry{get\_tasks()}\spxextra{in module checklisten.templatetags.t\_checklisten.get\_tasks}}

\begin{fulllineitems}
\phantomsection\label{\detokenize{masterCodeDoc:checklisten.templatetags.t_checklisten.get_tasks.get_tasks}}\pysiglinewithargsret{\sphinxcode{\sphinxupquote{checklisten.templatetags.t\_checklisten.get\_tasks.}}\sphinxbfcode{\sphinxupquote{get\_tasks}}}{\emph{\DUrole{n}{checklist\_id}}}{}
Gets all ChecklisteAufgabe objects that belong to the Checkliste identified by checklist\_id.
\begin{quote}\begin{description}
\item[{Parameters}] \leavevmode
\sphinxstyleliteralstrong{\sphinxupquote{checklist\_id}} (\sphinxstyleliteralemphasis{\sphinxupquote{int}}) \textendash{} The ID of the Checkliste to fetch the ChecklisteAufgabe objects for.

\item[{Returns}] \leavevmode
A QuerySet containing all ChecklisteAufgabe objects for the specified Checkliste.

\item[{Return type}] \leavevmode
QuerySet

\end{description}\end{quote}

\end{fulllineitems}



\section{Importscripts}
\label{\detokenize{masterCodeDoc:importscripts}}
You can use the following script/scripts to import some data in your \sphinxstyleemphasis{db.sqlite3}
Tese script/scripts are used in the functional tests.


\subsection{How to use:}
\label{\detokenize{masterCodeDoc:how-to-use}}\begin{quote}

Change to the directory \sphinxcode{\sphinxupquote{cd importscrips}} and execute \sphinxstyleemphasis{main.py} \sphinxcode{\sphinxupquote{python3 ./main.py}}
\end{quote}


\subsection{CSV Data:}
\label{\detokenize{masterCodeDoc:csv-data}}\begin{itemize}
\item {} \begin{description}
\item[{ReferateUnterbereicheAemter.csv}] \leavevmode\begin{itemize}
\item {} 
contains basic data for the tables: “Organisationseinheit, Unterbereich und Funktion”

\end{itemize}

\end{description}

\end{itemize}


\subsection{Functions:}
\label{\detokenize{masterCodeDoc:functions}}\begin{quote}


\begin{fulllineitems}
\pysiglinewithargsret{\sphinxcode{\sphinxupquote{importscripts.main.}}\sphinxbfcode{\sphinxupquote{importAemter}}}{\emph{\DUrole{n}{file}}}{}
\sphinxstylestrong{!WARNING!}
This function clears following Tables in your Database:
\begin{itemize}
\item {} 
Organisationseinheit

\item {} 
Unterbereich

\item {} 
Funktion

\end{itemize}

To use this function you need to have a file.csv with following structure:
\begin{itemize}
\item {} 
delimiter = ‘,’

\item {} 
organisationseinheit,unterbereich,funktion,max\_members

\item {} 
First line is a heading and will not be imported

\end{itemize}
\begin{quote}\begin{description}
\item[{Parameters}] \leavevmode
\sphinxstyleliteralstrong{\sphinxupquote{file}} (\sphinxstyleliteralemphasis{\sphinxupquote{TextIO}}) \textendash{} File containing the data to be imported

\item[{Returns}] \leavevmode
No return Value

\end{description}\end{quote}

\end{fulllineitems}

\end{quote}


\section{Tests}
\label{\detokenize{masterCodeDoc:tests}}
This section contains the description of the functional UI tests created with the Selenium testing framework. The Pagination
and the Module historie are passively tested in the other Tests.


\subsection{Sources}
\label{\detokenize{masterCodeDoc:sources}}\begin{itemize}
\item {} 
\sphinxhref{https://django-testing-docs.readthedocs.io/en/latest/index.html}{django\sphinxhyphen{}testing\sphinxhyphen{}docs}

\end{itemize}


\subsection{Dependencies}
\label{\detokenize{masterCodeDoc:dependencies}}

\subsubsection{selenium}
\label{\detokenize{masterCodeDoc:selenium}}\begin{itemize}
\item {} 
Installation: \sphinxcode{\sphinxupquote{pip install selenium}}

\item {} 
Documentation: \sphinxhref{https://www.selenium.dev/documentation/en/}{www.selenium.dev}

\end{itemize}


\subsubsection{Webdriver for Firefox}
\label{\detokenize{masterCodeDoc:webdriver-for-firefox}}
Link to Mozilla’s site of the Gecko\sphinxhyphen{}Webdriver \sphinxhref{https://github.com/mozilla/geckodriver}{geckodriver}.


\subsection{MyTestCase}
\label{\detokenize{masterCodeDoc:mytestcase}}\begin{quote}


\begin{fulllineitems}
\pysiglinewithargsret{\sphinxbfcode{\sphinxupquote{class }}\sphinxcode{\sphinxupquote{tests.MyTestCase.}}\sphinxbfcode{\sphinxupquote{MyTestCase}}}{\emph{\DUrole{n}{methodName}\DUrole{o}{=}\DUrole{default_value}{\textquotesingle{}runTest\textquotesingle{}}}}{}
Setup and Teardown funktions are specified here.
The following Testcases inherit from this class.

All testcases inheriting from this class are testing the User Interface.


\begin{fulllineitems}
\pysiglinewithargsret{\sphinxbfcode{\sphinxupquote{setUp}}}{}{}
This function is called before every testcase.

It sets up the webdriver and creates 1 admin and 1 user.
You can adjust the webdriver by changing the \sphinxstyleemphasis{options} parameter.
The Importscripts from the folder \sphinxstyleemphasis{importscripts} are also called here.

The Webdriver Instance is stored in \sphinxstylestrong{self.browser}.
\begin{quote}\begin{description}
\item[{Parameters}] \leavevmode
\sphinxstyleliteralstrong{\sphinxupquote{self}} \textendash{} 

\item[{Returns}] \leavevmode
No return Value

\end{description}\end{quote}

\end{fulllineitems}



\begin{fulllineitems}
\pysiglinewithargsret{\sphinxbfcode{\sphinxupquote{tearDown}}}{}{}
This function is called after every testcase.

The Webdriver Instance that is stored in \sphinxstylestrong{self.browser} will be closed.
:param self:
:type self:
:return: No return Value

\end{fulllineitems}


\end{fulllineitems}

\end{quote}


\subsection{Helper Functions}
\label{\detokenize{masterCodeDoc:helper-functions}}\begin{quote}


\begin{fulllineitems}
\pysiglinewithargsret{\sphinxcode{\sphinxupquote{tests.MyFuncLogin.}}\sphinxbfcode{\sphinxupquote{loginAsLukasAdmin}}}{\emph{\DUrole{n}{self}}}{}
Opens a Browser instance and login as admin with the account testlukasadmin.
\begin{quote}\begin{description}
\item[{Parameters}] \leavevmode
\sphinxstyleliteralstrong{\sphinxupquote{self}} \textendash{} 

\item[{Returns}] \leavevmode
No return Value

\end{description}\end{quote}

\end{fulllineitems}



\begin{fulllineitems}
\pysiglinewithargsret{\sphinxcode{\sphinxupquote{tests.MyFuncLogin.}}\sphinxbfcode{\sphinxupquote{loginAsLukasUser}}}{\emph{\DUrole{n}{self}}}{}
Opens a Browser instance and login as user with the account testlukas.
\begin{quote}\begin{description}
\item[{Parameters}] \leavevmode
\sphinxstyleliteralstrong{\sphinxupquote{self}} \textendash{} 

\item[{Returns}] \leavevmode
No return Value

\end{description}\end{quote}

\end{fulllineitems}



\begin{fulllineitems}
\pysiglinewithargsret{\sphinxcode{\sphinxupquote{tests.MyFuncMitglieder.}}\sphinxbfcode{\sphinxupquote{addMitglied}}}{\emph{\DUrole{n}{self}}}{}
Add a member with default parameters over the “mitglieder/erstellen” view.
You need to be on the “mitglieder/” site to call this function.
\begin{quote}\begin{description}
\item[{Parameters}] \leavevmode
\sphinxstyleliteralstrong{\sphinxupquote{self}} \textendash{} 

\item[{Returns}] \leavevmode
No return Value

\end{description}\end{quote}

\end{fulllineitems}



\begin{fulllineitems}
\pysiglinewithargsret{\sphinxcode{\sphinxupquote{tests.MyFuncMitglieder.}}\sphinxbfcode{\sphinxupquote{addMitgliedWithParameters}}}{\emph{\DUrole{n}{self}}, \emph{\DUrole{n}{vorname}}, \emph{\DUrole{n}{nachname}}, \emph{\DUrole{n}{spitzname}}}{}
Add a member with default parameters over the “mitglieder/erstellen” view.
You can change firstname, lastname and username.
You need to be on the “mitglieder/” site to call this function.
\begin{quote}\begin{description}
\item[{Parameters}] \leavevmode\begin{itemize}
\item {} 
\sphinxstyleliteralstrong{\sphinxupquote{self}} \textendash{} 

\item {} 
\sphinxstyleliteralstrong{\sphinxupquote{vorname}} (\sphinxstyleliteralemphasis{\sphinxupquote{string}}) \textendash{} Firstname of the Member, that you want to create.

\item {} 
\sphinxstyleliteralstrong{\sphinxupquote{nachname}} (\sphinxstyleliteralemphasis{\sphinxupquote{string}}) \textendash{} Lastname of the Member, that you want to create.

\item {} 
\sphinxstyleliteralstrong{\sphinxupquote{spitzname}} (\sphinxstyleliteralemphasis{\sphinxupquote{string}}) \textendash{} Username of the Member, that you want to create.

\end{itemize}

\item[{Returns}] \leavevmode
No return Value

\end{description}\end{quote}

\end{fulllineitems}



\begin{fulllineitems}
\pysiglinewithargsret{\sphinxcode{\sphinxupquote{tests.MyFuncAemter.}}\sphinxbfcode{\sphinxupquote{createAmt}}}{\emph{\DUrole{n}{self}}, \emph{\DUrole{n}{organisationseinheit}}, \emph{\DUrole{n}{unterbereich}}, \emph{\DUrole{n}{funktion}}}{}
Erstellen eines Amtes über die GUI, benötigt ist ein login als AdminPanel
Ausgang ist, dass der User Angemeldet ist und sich in der Mitglieder sicht
befindet.
\begin{quote}\begin{description}
\item[{Parameters}] \leavevmode\begin{itemize}
\item {} 
\sphinxstyleliteralstrong{\sphinxupquote{self}} \textendash{} 

\item {} 
\sphinxstyleliteralstrong{\sphinxupquote{organisationseinheit}} (\sphinxstyleliteralemphasis{\sphinxupquote{string}}) \textendash{} Organisationseinheit, dem das Funktion zugeordnet werden soll

\item {} 
\sphinxstyleliteralstrong{\sphinxupquote{unterbereich}} (\sphinxstyleliteralemphasis{\sphinxupquote{string}}) \textendash{} Unterbereich des Referats

\item {} 
\sphinxstyleliteralstrong{\sphinxupquote{funktion}} (\sphinxstyleliteralemphasis{\sphinxupquote{string}}) \textendash{} Angabe des Namens, der das neue Funktion erhalten soll

\end{itemize}

\item[{Returns}] \leavevmode
No return Value

\end{description}\end{quote}

\end{fulllineitems}



\begin{fulllineitems}
\pysiglinewithargsret{\sphinxcode{\sphinxupquote{tests.MyFuncAemter.}}\sphinxbfcode{\sphinxupquote{createReferat}}}{\emph{\DUrole{n}{self}}, \emph{\DUrole{n}{organisationseinheit}}}{}
Erstellen eines Referates über die GUI, benötigt ist ein login als AdminPanel
Ausgang ist, dass der User Angemeldet ist und sich in der Mitglieder sicht
befindet.
\begin{quote}\begin{description}
\item[{Parameters}] \leavevmode\begin{itemize}
\item {} 
\sphinxstyleliteralstrong{\sphinxupquote{self}} \textendash{} 

\item {} 
\sphinxstyleliteralstrong{\sphinxupquote{organisationseinheit}} (\sphinxstyleliteralemphasis{\sphinxupquote{string}}) \textendash{} Name, wie das neue Organisationseinheit heißen soll

\end{itemize}

\item[{Returns}] \leavevmode
No return Value

\end{description}\end{quote}

\end{fulllineitems}



\begin{fulllineitems}
\pysiglinewithargsret{\sphinxcode{\sphinxupquote{tests.MyFuncAemter.}}\sphinxbfcode{\sphinxupquote{createUnterbereich}}}{\emph{\DUrole{n}{self}}, \emph{\DUrole{n}{organisationseinheit}}, \emph{\DUrole{n}{unterbereich}}}{}
Erstellen eines Unterbereiches über die GUI, benötigt ist ein login als AdminPanel
Ausgang ist, dass der User Angemeldet ist und sich in der Mitglieder sicht
befindet.
\begin{quote}\begin{description}
\item[{Parameters}] \leavevmode\begin{itemize}
\item {} 
\sphinxstyleliteralstrong{\sphinxupquote{self}} \textendash{} 

\item {} 
\sphinxstyleliteralstrong{\sphinxupquote{organisationseinheit}} (\sphinxstyleliteralemphasis{\sphinxupquote{string}}) \textendash{} Organisationseinheit, dem der Unterbereich zugeordnet werden soll

\item {} 
\sphinxstyleliteralstrong{\sphinxupquote{unterbereich}} (\sphinxstyleliteralemphasis{\sphinxupquote{string}}) \textendash{} Name des Unterbereichs

\end{itemize}

\item[{Returns}] \leavevmode
No return Value

\end{description}\end{quote}

\end{fulllineitems}

\end{quote}


\subsection{Login Module Tests}
\label{\detokenize{masterCodeDoc:login-module-tests}}
The System under test (SUT) is the “login” module.


\subsubsection{Testcase 001}
\label{\detokenize{masterCodeDoc:testcase-001}}\begin{quote}


\begin{fulllineitems}
\pysiglinewithargsret{\sphinxbfcode{\sphinxupquote{class }}\sphinxcode{\sphinxupquote{tests.test\_001\_admin.}}\sphinxbfcode{\sphinxupquote{TestAdmin}}}{\emph{\DUrole{n}{methodName}\DUrole{o}{=}\DUrole{default_value}{\textquotesingle{}runTest\textquotesingle{}}}}{}~

\begin{fulllineitems}
\pysiglinewithargsret{\sphinxbfcode{\sphinxupquote{test\_login\_superuser}}}{}{}
This is a “positive” Systemtest as Blackboxtest.
Here we want to check if you can login as Admin and all sites are displayed correctly.

\end{fulllineitems}


\end{fulllineitems}

\end{quote}


\subsubsection{Testcase 002}
\label{\detokenize{masterCodeDoc:testcase-002}}\begin{quote}


\begin{fulllineitems}
\pysiglinewithargsret{\sphinxbfcode{\sphinxupquote{class }}\sphinxcode{\sphinxupquote{tests.test\_002\_user.}}\sphinxbfcode{\sphinxupquote{TestUser}}}{\emph{\DUrole{n}{methodName}\DUrole{o}{=}\DUrole{default_value}{\textquotesingle{}runTest\textquotesingle{}}}}{}~

\begin{fulllineitems}
\pysiglinewithargsret{\sphinxbfcode{\sphinxupquote{test\_login\_user}}}{}{}
This is a “positive” Systemtest as Blackboxtest.
Here we want to check if you can login as User and all sites are displayed correctly.
But that you can not reach Admin\sphinxhyphen{}only content.

\end{fulllineitems}


\end{fulllineitems}

\end{quote}


\subsection{Mitglieder Module Tests}
\label{\detokenize{masterCodeDoc:mitglieder-module-tests}}
The System under test (SUT) is the “mitglieder” module.


\subsubsection{Testcase 003}
\label{\detokenize{masterCodeDoc:testcase-003}}\begin{quote}

TODO: under Construction
\end{quote}


\subsubsection{Testcase 004}
\label{\detokenize{masterCodeDoc:testcase-004}}\begin{quote}


\begin{fulllineitems}
\pysiglinewithargsret{\sphinxbfcode{\sphinxupquote{class }}\sphinxcode{\sphinxupquote{tests.test\_004\_mitgliedHinzufuegen.}}\sphinxbfcode{\sphinxupquote{TestMitgliedHinzufuegen}}}{\emph{\DUrole{n}{methodName}\DUrole{o}{=}\DUrole{default_value}{\textquotesingle{}runTest\textquotesingle{}}}}{}~

\begin{fulllineitems}
\pysiglinewithargsret{\sphinxbfcode{\sphinxupquote{test\_1MitgliedHinzufügen\_AsSuperuser}}}{}{}
This is a “positive” Systemtest as Blackboxtest.
Here we want to check if you can add a new Member as Admin and if the
Member is displayed correctly in the table.

Steps:
\begin{itemize}
\item {} 
login as Admin

\item {} 
add a Member

\end{itemize}

\end{fulllineitems}



\begin{fulllineitems}
\pysiglinewithargsret{\sphinxbfcode{\sphinxupquote{test\_50MitgliederHinzufügen\_AsSuperuser\_lookAsUser}}}{}{}
This is a complex “positive” Systemtest as Blackboxtest.
Here we want to check if you can add a multiple new Members (50) as Admin and if the
Member is displayed correctly in the table.
We also want to check if the Pagination is working correctly.

Steps:
\begin{itemize}
\item {} 
login as Admin

\item {} 
add a Member (as loop)

\item {} 
check Pagination

\end{itemize}

\end{fulllineitems}


\end{fulllineitems}

\end{quote}


\subsubsection{Testcase 005}
\label{\detokenize{masterCodeDoc:testcase-005}}\begin{quote}


\begin{fulllineitems}
\pysiglinewithargsret{\sphinxbfcode{\sphinxupquote{class }}\sphinxcode{\sphinxupquote{tests.test\_005\_mitgliedEntfernen.}}\sphinxbfcode{\sphinxupquote{TestMitgliedEntfernen}}}{\emph{\DUrole{n}{methodName}\DUrole{o}{=}\DUrole{default_value}{\textquotesingle{}runTest\textquotesingle{}}}}{}~

\begin{fulllineitems}
\pysiglinewithargsret{\sphinxbfcode{\sphinxupquote{test\_1MitgliedEntfernen\_AsSuperuser}}}{}{}
This is a “positive” Systemtest as Blackboxtest.
Here we want to check if you can delete a new Member as Admin and if the
Member is deleted correctly in the table.

Steps:
\begin{itemize}
\item {} 
login as Admin

\item {} 
add a Member

\item {} 
delete Member

\end{itemize}

\end{fulllineitems}


\end{fulllineitems}

\end{quote}


\subsubsection{Testcase 006}
\label{\detokenize{masterCodeDoc:testcase-006}}\begin{quote}


\begin{fulllineitems}
\pysiglinewithargsret{\sphinxbfcode{\sphinxupquote{class }}\sphinxcode{\sphinxupquote{tests.test\_006\_mitgliedAendern.}}\sphinxbfcode{\sphinxupquote{TestMitgliedAendern}}}{\emph{\DUrole{n}{methodName}\DUrole{o}{=}\DUrole{default_value}{\textquotesingle{}runTest\textquotesingle{}}}}{}~

\begin{fulllineitems}
\pysiglinewithargsret{\sphinxbfcode{\sphinxupquote{test\_1MitgliedAendern\_AsSuperuser}}}{}{}
This is a “positive” Systemtest as Blackboxtest.
Here we want to check if you can change information of a Member as Admin and if the
Member is changed correctly in the table.

Steps:
\begin{itemize}
\item {} 
login as Admin

\item {} 
add a Member

\item {} 
change Member Data

\item {} 
check if everything is correct displayed

\end{itemize}

\end{fulllineitems}


\end{fulllineitems}

\end{quote}


\subsection{Aemter Module Tests}
\label{\detokenize{masterCodeDoc:aemter-module-tests}}
The System under test (SUT) is the “aemter” module.


\subsubsection{Testcase 007}
\label{\detokenize{masterCodeDoc:testcase-007}}\begin{quote}


\begin{fulllineitems}
\pysiglinewithargsret{\sphinxbfcode{\sphinxupquote{class }}\sphinxcode{\sphinxupquote{tests.test\_007\_aemtHinzufuegen.}}\sphinxbfcode{\sphinxupquote{TestAemtHinzufuegen}}}{\emph{\DUrole{n}{methodName}\DUrole{o}{=}\DUrole{default_value}{\textquotesingle{}runTest\textquotesingle{}}}}{}~

\begin{fulllineitems}
\pysiglinewithargsret{\sphinxbfcode{\sphinxupquote{test\_1FunktionHinzufuegen\_AsSuperuser}}}{}{}
This is a “positive” Systemtest as Blackboxtest.
Here we want to check if you can add a funktion as Admin.

Steps:
\begin{itemize}
\item {} 
login as Admin

\item {} 
add a funktion

\end{itemize}

\end{fulllineitems}



\begin{fulllineitems}
\pysiglinewithargsret{\sphinxbfcode{\sphinxupquote{test\_1OrganisationseinheitHinzufuegen\_AsSuperuser}}}{}{}
This is a “positive” Systemtest as Blackboxtest.
Here we want to check if you can add a “Organisationseinheit” as Admin.

Steps:
\begin{itemize}
\item {} 
login as Admin

\item {} 
add a “Organisationseinheit”

\end{itemize}

\end{fulllineitems}



\begin{fulllineitems}
\pysiglinewithargsret{\sphinxbfcode{\sphinxupquote{test\_1UnterbereichHinzufuegen\_AsSuperuser}}}{}{}
This is a “positive” Systemtest as Blackboxtest.
Here we want to check if you can add a “unterbereich” as Admin.

Steps:
\begin{itemize}
\item {} 
login as Admin

\item {} 
add a “unterbereich”

\end{itemize}

\end{fulllineitems}



\begin{fulllineitems}
\pysiglinewithargsret{\sphinxbfcode{\sphinxupquote{test\_ReferatUnterbereichAmtHinzufuegen\_AsSuperuser}}}{}{}
This is a complex “positive” Systemtest as Blackboxtest.
Here we want to check if you can add a organisationseinheit, unterbereich and funktion as Admin.
We also check if we can add a new Member with the new data and if everything is displayed correctly
in “/aemter/”.

Steps:
\begin{itemize}
\item {} 
login as Admin

\item {} 
add a “organisationseinheit”

\item {} 
add a “unterbereich”

\item {} 
add a funktion

\item {} 
create a Member

\item {} 
navigate to aemteruebersicht “/aemter/”

\end{itemize}

\end{fulllineitems}


\end{fulllineitems}

\end{quote}


\subsubsection{Testcase 008}
\label{\detokenize{masterCodeDoc:testcase-008}}\begin{quote}


\begin{fulllineitems}
\pysiglinewithargsret{\sphinxbfcode{\sphinxupquote{class }}\sphinxcode{\sphinxupquote{tests.test\_008\_aemtEntfernen.}}\sphinxbfcode{\sphinxupquote{TestAemtEntfernen}}}{\emph{\DUrole{n}{methodName}\DUrole{o}{=}\DUrole{default_value}{\textquotesingle{}runTest\textquotesingle{}}}}{}~

\begin{fulllineitems}
\pysiglinewithargsret{\sphinxbfcode{\sphinxupquote{test\_1AemtEntfernen\_AsSuperuser}}}{}{}
This is a “positive” Systemtest as Blackboxtest.
Here we want to check if you can delete a funktion as Admin.

Steps:
\begin{itemize}
\item {} 
login as Admin

\item {} 
add a funktion

\item {} 
delete the funktion

\end{itemize}

\end{fulllineitems}



\begin{fulllineitems}
\pysiglinewithargsret{\sphinxbfcode{\sphinxupquote{test\_1ReferatEntfernen\_AsSuperuser}}}{}{}
This is a “positive” Systemtest as Blackboxtest.
Here we want to check if you can delete a “Organisationseinheit” as Admin.

Steps:
\begin{itemize}
\item {} 
login as Admin

\item {} 
add a “Organisationseinheit”

\item {} 
delete the “Organisationseinheit”

\end{itemize}

\end{fulllineitems}



\begin{fulllineitems}
\pysiglinewithargsret{\sphinxbfcode{\sphinxupquote{test\_1UnterbereichEntfernen\_AsSuperuser}}}{}{}
This is a “positive” Systemtest as Blackboxtest.
Here we want to check if you can delete a “unterbereich” as Admin.

Steps:
\begin{itemize}
\item {} 
login as Admin

\item {} 
add a “unterbereich”

\item {} 
delete the “unterbereich”

\end{itemize}

\end{fulllineitems}


\end{fulllineitems}

\end{quote}


\subsubsection{Testcase 009}
\label{\detokenize{masterCodeDoc:testcase-009}}\begin{quote}


\begin{fulllineitems}
\pysiglinewithargsret{\sphinxbfcode{\sphinxupquote{class }}\sphinxcode{\sphinxupquote{tests.test\_009\_aemtAendern.}}\sphinxbfcode{\sphinxupquote{TestAemtAendern}}}{\emph{\DUrole{n}{methodName}\DUrole{o}{=}\DUrole{default_value}{\textquotesingle{}runTest\textquotesingle{}}}}{}~

\begin{fulllineitems}
\pysiglinewithargsret{\sphinxbfcode{\sphinxupquote{test\_1AmtBezeichnungAendern}}}{}{}
This is a “positive” Systemtest as Blackboxtest.
Here we want to check if you can change information of a funktion as Admin.
We change the name of the funktion by appending a “\_1”.

\end{fulllineitems}



\begin{fulllineitems}
\pysiglinewithargsret{\sphinxbfcode{\sphinxupquote{test\_1AmtReferatAendern}}}{}{}
This is a “positive” Systemtest as Blackboxtest.
Here we want to check if you can change the “organisationseinheit” of a funktion as Admin.

\end{fulllineitems}



\begin{fulllineitems}
\pysiglinewithargsret{\sphinxbfcode{\sphinxupquote{test\_1AmtWorkloadAendern}}}{}{}
This is a “positive” Systemtest as Blackboxtest.
Here we want to check if you can change the workload of a funktion as Admin.

\end{fulllineitems}



\begin{fulllineitems}
\pysiglinewithargsret{\sphinxbfcode{\sphinxupquote{test\_1OrganisationseinheitBezeichnungAendern}}}{}{}
This is a “positive” Systemtest as Blackboxtest.
Here we want to check if you can change information of a “organisationseinheit” as Admin.
We change the name of the “organisationseinheit” by appending a “\_1”.

\end{fulllineitems}



\begin{fulllineitems}
\pysiglinewithargsret{\sphinxbfcode{\sphinxupquote{test\_1UnterbereichBezeichnungAendern}}}{}{}
This is a “positive” Systemtest as Blackboxtest.
Here we want to check if you can change information of a “unterbereich” as Admin.
We change the name of the “unterbereich” by appending a “\_1”.

\end{fulllineitems}



\begin{fulllineitems}
\pysiglinewithargsret{\sphinxbfcode{\sphinxupquote{test\_1UnterbereichReferatAendern}}}{}{}
This is a “positive” Systemtest as Blackboxtest.
Here we want to check if you can change the “organisationseinheit” of a “unterbereich” as Admin.

\end{fulllineitems}


\end{fulllineitems}

\end{quote}


\subsection{Latest test coverage report}
\label{\detokenize{masterCodeDoc:latest-test-coverage-report}}
Commands to generate a coverage report:
\begin{quote}
\begin{quote}

\begin{DUlineblock}{0em}
\item[] \sphinxcode{\sphinxupquote{coverage run \sphinxhyphen{}\sphinxhyphen{}source=aemter,bin,checklisten,historie,login,mitglieder,tests \sphinxhyphen{}\sphinxhyphen{}omit=*/migrations/* ./manage.py test}}
\item[] \sphinxcode{\sphinxupquote{coverage report}}
\end{DUlineblock}
\end{quote}
\sphinxSetupCaptionForVerbatim{coverage report}
\def\sphinxLiteralBlockLabel{\label{\detokenize{masterCodeDoc:id14}}}
\begin{sphinxVerbatim}[commandchars=\\\{\}]
Name                                           Stmts   Miss  Cover
\PYGZhy{}\PYGZhy{}\PYGZhy{}\PYGZhy{}\PYGZhy{}\PYGZhy{}\PYGZhy{}\PYGZhy{}\PYGZhy{}\PYGZhy{}\PYGZhy{}\PYGZhy{}\PYGZhy{}\PYGZhy{}\PYGZhy{}\PYGZhy{}\PYGZhy{}\PYGZhy{}\PYGZhy{}\PYGZhy{}\PYGZhy{}\PYGZhy{}\PYGZhy{}\PYGZhy{}\PYGZhy{}\PYGZhy{}\PYGZhy{}\PYGZhy{}\PYGZhy{}\PYGZhy{}\PYGZhy{}\PYGZhy{}\PYGZhy{}\PYGZhy{}\PYGZhy{}\PYGZhy{}\PYGZhy{}\PYGZhy{}\PYGZhy{}\PYGZhy{}\PYGZhy{}\PYGZhy{}\PYGZhy{}\PYGZhy{}\PYGZhy{}\PYGZhy{}\PYGZhy{}\PYGZhy{}\PYGZhy{}\PYGZhy{}\PYGZhy{}\PYGZhy{}\PYGZhy{}\PYGZhy{}\PYGZhy{}\PYGZhy{}\PYGZhy{}\PYGZhy{}\PYGZhy{}\PYGZhy{}\PYGZhy{}\PYGZhy{}\PYGZhy{}\PYGZhy{}\PYGZhy{}\PYGZhy{}
aemter/\PYGZus{}\PYGZus{}init\PYGZus{}\PYGZus{}.py                                 \PYG{l+m}{0}      \PYG{l+m}{0}   \PYG{l+m}{100}\PYGZpc{}
aemter/admin.py                                   \PYG{l+m}{26}      \PYG{l+m}{0}   \PYG{l+m}{100}\PYGZpc{}
aemter/apps.py                                     \PYG{l+m}{3}      \PYG{l+m}{0}   \PYG{l+m}{100}\PYGZpc{}
aemter/models.py                                  \PYG{l+m}{35}      \PYG{l+m}{2}    \PYG{l+m}{94}\PYGZpc{}
aemter/tests/\PYGZus{}\PYGZus{}init\PYGZus{}\PYGZus{}.py                           \PYG{l+m}{0}      \PYG{l+m}{0}   \PYG{l+m}{100}\PYGZpc{}
aemter/tests/test\PYGZus{}apps.py                          \PYG{l+m}{7}      \PYG{l+m}{0}   \PYG{l+m}{100}\PYGZpc{}
aemter/tests/test\PYGZus{}models.py                       \PYG{l+m}{17}      \PYG{l+m}{0}   \PYG{l+m}{100}\PYGZpc{}
aemter/tests/test\PYGZus{}urls.py                          \PYG{l+m}{8}      \PYG{l+m}{0}   \PYG{l+m}{100}\PYGZpc{}
aemter/tests/test\PYGZus{}views.py                        \PYG{l+m}{24}      \PYG{l+m}{0}   \PYG{l+m}{100}\PYGZpc{}
aemter/urls.py                                     \PYG{l+m}{7}      \PYG{l+m}{0}   \PYG{l+m}{100}\PYGZpc{}
aemter/views.py                                   \PYG{l+m}{22}      \PYG{l+m}{0}   \PYG{l+m}{100}\PYGZpc{}
bin/\PYGZus{}\PYGZus{}init\PYGZus{}\PYGZus{}.py                                    \PYG{l+m}{0}      \PYG{l+m}{0}   \PYG{l+m}{100}\PYGZpc{}
bin/asgi.py                                        \PYG{l+m}{4}      \PYG{l+m}{4}     \PYG{l+m}{0}\PYGZpc{}
bin/settings.py                                   \PYG{l+m}{19}      \PYG{l+m}{0}   \PYG{l+m}{100}\PYGZpc{}
bin/urls.py                                        \PYG{l+m}{3}      \PYG{l+m}{0}   \PYG{l+m}{100}\PYGZpc{}
bin/wsgi.py                                        \PYG{l+m}{4}      \PYG{l+m}{4}     \PYG{l+m}{0}\PYGZpc{}
checklisten/\PYGZus{}\PYGZus{}init\PYGZus{}\PYGZus{}.py                            \PYG{l+m}{0}      \PYG{l+m}{0}   \PYG{l+m}{100}\PYGZpc{}
checklisten/admin.py                               \PYG{l+m}{1}      \PYG{l+m}{0}   \PYG{l+m}{100}\PYGZpc{}
checklisten/apps.py                                \PYG{l+m}{3}      \PYG{l+m}{0}   \PYG{l+m}{100}\PYGZpc{}
checklisten/models.py                             \PYG{l+m}{10}      \PYG{l+m}{0}   \PYG{l+m}{100}\PYGZpc{}
checklisten/tests/\PYGZus{}\PYGZus{}init\PYGZus{}\PYGZus{}.py                      \PYG{l+m}{0}      \PYG{l+m}{0}   \PYG{l+m}{100}\PYGZpc{}
checklisten/tests/test\PYGZus{}apps.py                     \PYG{l+m}{7}      \PYG{l+m}{0}   \PYG{l+m}{100}\PYGZpc{}
checklisten/urls.py                                \PYG{l+m}{7}      \PYG{l+m}{0}   \PYG{l+m}{100}\PYGZpc{}
checklisten/views.py                               \PYG{l+m}{8}      \PYG{l+m}{5}    \PYG{l+m}{38}\PYGZpc{}
historie/\PYGZus{}\PYGZus{}init\PYGZus{}\PYGZus{}.py                               \PYG{l+m}{0}      \PYG{l+m}{0}   \PYG{l+m}{100}\PYGZpc{}
historie/apps.py                                   \PYG{l+m}{3}      \PYG{l+m}{0}   \PYG{l+m}{100}\PYGZpc{}
historie/models.py                                 \PYG{l+m}{4}      \PYG{l+m}{0}   \PYG{l+m}{100}\PYGZpc{}
historie/templatetags/\PYGZus{}\PYGZus{}init\PYGZus{}\PYGZus{}.py                  \PYG{l+m}{0}      \PYG{l+m}{0}   \PYG{l+m}{100}\PYGZpc{}
historie/templatetags/get\PYGZus{}associated\PYGZus{}data.py      \PYG{l+m}{25}     \PYG{l+m}{19}    \PYG{l+m}{24}\PYGZpc{}
historie/templatetags/to\PYGZus{}class\PYGZus{}name.py             \PYG{l+m}{5}      \PYG{l+m}{0}   \PYG{l+m}{100}\PYGZpc{}
historie/tests/\PYGZus{}\PYGZus{}init\PYGZus{}\PYGZus{}.py                         \PYG{l+m}{0}      \PYG{l+m}{0}   \PYG{l+m}{100}\PYGZpc{}
historie/tests/test\PYGZus{}apps.py                        \PYG{l+m}{7}      \PYG{l+m}{0}   \PYG{l+m}{100}\PYGZpc{}
historie/tests/test\PYGZus{}urls.py                        \PYG{l+m}{8}      \PYG{l+m}{0}   \PYG{l+m}{100}\PYGZpc{}
historie/tests/test\PYGZus{}views.py                      \PYG{l+m}{22}      \PYG{l+m}{0}   \PYG{l+m}{100}\PYGZpc{}
historie/urls.py                                   \PYG{l+m}{4}      \PYG{l+m}{0}   \PYG{l+m}{100}\PYGZpc{}
historie/views.py                                \PYG{l+m}{101}     \PYG{l+m}{54}    \PYG{l+m}{47}\PYGZpc{}
login/\PYGZus{}\PYGZus{}init\PYGZus{}\PYGZus{}.py                                  \PYG{l+m}{0}      \PYG{l+m}{0}   \PYG{l+m}{100}\PYGZpc{}
login/apps.py                                      \PYG{l+m}{3}      \PYG{l+m}{0}   \PYG{l+m}{100}\PYGZpc{}
login/tests/\PYGZus{}\PYGZus{}init\PYGZus{}\PYGZus{}.py                            \PYG{l+m}{0}      \PYG{l+m}{0}   \PYG{l+m}{100}\PYGZpc{}
login/tests/test\PYGZus{}apps.py                           \PYG{l+m}{7}      \PYG{l+m}{0}   \PYG{l+m}{100}\PYGZpc{}
login/tests/test\PYGZus{}urls.py                          \PYG{l+m}{11}      \PYG{l+m}{0}   \PYG{l+m}{100}\PYGZpc{}
login/tests/test\PYGZus{}views.py                         \PYG{l+m}{35}      \PYG{l+m}{0}   \PYG{l+m}{100}\PYGZpc{}
login/urls.py                                      \PYG{l+m}{4}      \PYG{l+m}{0}   \PYG{l+m}{100}\PYGZpc{}
login/views.py                                    \PYG{l+m}{27}      \PYG{l+m}{2}    \PYG{l+m}{93}\PYGZpc{}
mitglieder/\PYGZus{}\PYGZus{}init\PYGZus{}\PYGZus{}.py                             \PYG{l+m}{0}      \PYG{l+m}{0}   \PYG{l+m}{100}\PYGZpc{}
mitglieder/admin.py                                \PYG{l+m}{2}      \PYG{l+m}{0}   \PYG{l+m}{100}\PYGZpc{}
mitglieder/apps.py                                 \PYG{l+m}{3}      \PYG{l+m}{0}   \PYG{l+m}{100}\PYGZpc{}
mitglieder/models.py                              \PYG{l+m}{42}      \PYG{l+m}{1}    \PYG{l+m}{98}\PYGZpc{}
mitglieder/tests/\PYGZus{}\PYGZus{}init\PYGZus{}\PYGZus{}.py                       \PYG{l+m}{0}      \PYG{l+m}{0}   \PYG{l+m}{100}\PYGZpc{}
mitglieder/tests/test\PYGZus{}apps.py                      \PYG{l+m}{7}      \PYG{l+m}{0}   \PYG{l+m}{100}\PYGZpc{}
mitglieder/tests/test\PYGZus{}models.py                   \PYG{l+m}{17}      \PYG{l+m}{0}   \PYG{l+m}{100}\PYGZpc{}
mitglieder/tests/test\PYGZus{}urls.py                     \PYG{l+m}{47}      \PYG{l+m}{0}   \PYG{l+m}{100}\PYGZpc{}
mitglieder/tests/test\PYGZus{}views.py                    \PYG{l+m}{51}      \PYG{l+m}{0}   \PYG{l+m}{100}\PYGZpc{}
mitglieder/urls.py                                 \PYG{l+m}{4}      \PYG{l+m}{0}   \PYG{l+m}{100}\PYGZpc{}
mitglieder/views.py                              \PYG{l+m}{205}     \PYG{l+m}{66}    \PYG{l+m}{68}\PYGZpc{}
tests/MyFuncAemter.py                             \PYG{l+m}{34}      \PYG{l+m}{0}   \PYG{l+m}{100}\PYGZpc{}
tests/MyFuncLogin.py                              \PYG{l+m}{33}      \PYG{l+m}{8}    \PYG{l+m}{76}\PYGZpc{}
tests/MyFuncMitglieder.py                        \PYG{l+m}{103}     \PYG{l+m}{14}    \PYG{l+m}{86}\PYGZpc{}
tests/MyTestCase.py                               \PYG{l+m}{31}      \PYG{l+m}{4}    \PYG{l+m}{87}\PYGZpc{}
tests/\PYGZus{}\PYGZus{}init\PYGZus{}\PYGZus{}.py                                  \PYG{l+m}{0}      \PYG{l+m}{0}   \PYG{l+m}{100}\PYGZpc{}
tests/test\PYGZus{}001\PYGZus{}admin.py                            \PYG{l+m}{6}      \PYG{l+m}{0}   \PYG{l+m}{100}\PYGZpc{}
tests/test\PYGZus{}002\PYGZus{}user.py                             \PYG{l+m}{6}      \PYG{l+m}{0}   \PYG{l+m}{100}\PYGZpc{}
tests/test\PYGZus{}003\PYGZus{}multiuser.py                        \PYG{l+m}{1}      \PYG{l+m}{0}   \PYG{l+m}{100}\PYGZpc{}
tests/test\PYGZus{}004\PYGZus{}mitgliedHinzufuegen.py             \PYG{l+m}{38}      \PYG{l+m}{4}    \PYG{l+m}{89}\PYGZpc{}
tests/test\PYGZus{}005\PYGZus{}mitgliedEntfernen.py               \PYG{l+m}{13}      \PYG{l+m}{0}   \PYG{l+m}{100}\PYGZpc{}
tests/test\PYGZus{}006\PYGZus{}mitgliedAendern.py                 \PYG{l+m}{13}      \PYG{l+m}{0}   \PYG{l+m}{100}\PYGZpc{}
tests/test\PYGZus{}007\PYGZus{}aemtHinzufuegen.py                 \PYG{l+m}{63}      \PYG{l+m}{0}   \PYG{l+m}{100}\PYGZpc{}
tests/test\PYGZus{}008\PYGZus{}aemtEntfernen.py                   \PYG{l+m}{47}      \PYG{l+m}{0}   \PYG{l+m}{100}\PYGZpc{}
tests/test\PYGZus{}009\PYGZus{}aemtAendern.py                     \PYG{l+m}{92}      \PYG{l+m}{0}   \PYG{l+m}{100}\PYGZpc{}
\PYGZhy{}\PYGZhy{}\PYGZhy{}\PYGZhy{}\PYGZhy{}\PYGZhy{}\PYGZhy{}\PYGZhy{}\PYGZhy{}\PYGZhy{}\PYGZhy{}\PYGZhy{}\PYGZhy{}\PYGZhy{}\PYGZhy{}\PYGZhy{}\PYGZhy{}\PYGZhy{}\PYGZhy{}\PYGZhy{}\PYGZhy{}\PYGZhy{}\PYGZhy{}\PYGZhy{}\PYGZhy{}\PYGZhy{}\PYGZhy{}\PYGZhy{}\PYGZhy{}\PYGZhy{}\PYGZhy{}\PYGZhy{}\PYGZhy{}\PYGZhy{}\PYGZhy{}\PYGZhy{}\PYGZhy{}\PYGZhy{}\PYGZhy{}\PYGZhy{}\PYGZhy{}\PYGZhy{}\PYGZhy{}\PYGZhy{}\PYGZhy{}\PYGZhy{}\PYGZhy{}\PYGZhy{}\PYGZhy{}\PYGZhy{}\PYGZhy{}\PYGZhy{}\PYGZhy{}\PYGZhy{}\PYGZhy{}\PYGZhy{}\PYGZhy{}\PYGZhy{}\PYGZhy{}\PYGZhy{}\PYGZhy{}\PYGZhy{}\PYGZhy{}\PYGZhy{}\PYGZhy{}\PYGZhy{}
TOTAL                                           \PYG{l+m}{1339}    \PYG{l+m}{187}    \PYG{l+m}{86}\PYGZpc{}
\end{sphinxVerbatim}
\end{quote}


\section{Commands}
\label{\detokenize{masterCodeDoc:commands}}

\subsection{delete\_old\_historie}
\label{\detokenize{masterCodeDoc:module-bin.management.commands.delete_old_historie}}\label{\detokenize{masterCodeDoc:delete-old-historie}}\index{module@\spxentry{module}!bin.management.commands.delete\_old\_historie@\spxentry{bin.management.commands.delete\_old\_historie}}\index{bin.management.commands.delete\_old\_historie@\spxentry{bin.management.commands.delete\_old\_historie}!module@\spxentry{module}}\index{Command (class in bin.management.commands.delete\_old\_historie)@\spxentry{Command}\spxextra{class in bin.management.commands.delete\_old\_historie}}

\begin{fulllineitems}
\phantomsection\label{\detokenize{masterCodeDoc:bin.management.commands.delete_old_historie.Command}}\pysiglinewithargsret{\sphinxbfcode{\sphinxupquote{class }}\sphinxcode{\sphinxupquote{bin.management.commands.delete\_old\_historie.}}\sphinxbfcode{\sphinxupquote{Command}}}{\emph{\DUrole{n}{stdout}\DUrole{o}{=}\DUrole{default_value}{None}}, \emph{\DUrole{n}{stderr}\DUrole{o}{=}\DUrole{default_value}{None}}, \emph{\DUrole{n}{no\_color}\DUrole{o}{=}\DUrole{default_value}{False}}, \emph{\DUrole{n}{force\_color}\DUrole{o}{=}\DUrole{default_value}{False}}}{}~\index{handle() (bin.management.commands.delete\_old\_historie.Command method)@\spxentry{handle()}\spxextra{bin.management.commands.delete\_old\_historie.Command method}}

\begin{fulllineitems}
\phantomsection\label{\detokenize{masterCodeDoc:bin.management.commands.delete_old_historie.Command.handle}}\pysiglinewithargsret{\sphinxbfcode{\sphinxupquote{handle}}}{\emph{\DUrole{o}{*}\DUrole{n}{args}}, \emph{\DUrole{o}{**}\DUrole{n}{options}}}{}
This command deletes all entries from the Historie of Mitglied, MitgliedAmt and MitgliedMail that
\begin{itemize}
\item {} 
are at least 1 year old if the referenced Mitglied does not exist in the database anymore

\item {} 
are at least 5 years old otherwise

\end{itemize}

Please note that 1 year is equivalent to 365 days, so leap years are not accounted for.

It can be called using \sphinxcode{\sphinxupquote{python manage.py delete\_old\_historie}} and can thus be automated, for example by setting up a cronjob.

\end{fulllineitems}

\index{help (bin.management.commands.delete\_old\_historie.Command attribute)@\spxentry{help}\spxextra{bin.management.commands.delete\_old\_historie.Command attribute}}

\begin{fulllineitems}
\phantomsection\label{\detokenize{masterCodeDoc:bin.management.commands.delete_old_historie.Command.help}}\pysigline{\sphinxbfcode{\sphinxupquote{help}}\sphinxbfcode{\sphinxupquote{ = \textquotesingle{}Deletes all entries concerning Mitglieder from the Historie older than 5 years. If the associated Mitglied is not in the database anymore, the entry will be deleted if it is older than 1 year.\textquotesingle{}}}}
\end{fulllineitems}


\end{fulllineitems}



\subsection{clean\_duplicate\_history}
\label{\detokenize{masterCodeDoc:clean-duplicate-history}}
django\sphinxhyphen{}simple\sphinxhyphen{}history includes a command that can be used to remove all duplicate entries in the Historie.
For example, an entry is created in MitgliedMail and MitgliedAmt when the associated Mitglied is changed, even though no changes
have been made to MitgliedMail and MitgliedAmt. To get rid of these entries, you can use:

\sphinxcode{\sphinxupquote{python manage.py clean\_duplicate\_history {[}\sphinxhyphen{}m ...{]} \sphinxhyphen{}\sphinxhyphen{}auto}}

The \sphinxhyphen{}m flag is optional and is used to specify the amount of minutes to go back when searching for duplicate entries. This command can
be automated, e.g. by running a cronjob.


\renewcommand{\indexname}{Python Module Index}
\begin{sphinxtheindex}
\let\bigletter\sphinxstyleindexlettergroup
\bigletter{a}
\item\relax\sphinxstyleindexentry{aemter.models}\sphinxstyleindexpageref{masterCodeDoc:\detokenize{module-aemter.models}}
\item\relax\sphinxstyleindexentry{aemter.views}\sphinxstyleindexpageref{masterCodeDoc:\detokenize{module-aemter.views}}
\indexspace
\bigletter{b}
\item\relax\sphinxstyleindexentry{bin.management.commands.delete\_old\_historie}\sphinxstyleindexpageref{masterCodeDoc:\detokenize{module-bin.management.commands.delete_old_historie}}
\indexspace
\bigletter{c}
\item\relax\sphinxstyleindexentry{checklisten.models}\sphinxstyleindexpageref{masterCodeDoc:\detokenize{module-checklisten.models}}
\item\relax\sphinxstyleindexentry{checklisten.templatetags.t\_checklisten.get\_perms}\sphinxstyleindexpageref{masterCodeDoc:\detokenize{module-checklisten.templatetags.t_checklisten.get_perms}}
\item\relax\sphinxstyleindexentry{checklisten.templatetags.t\_checklisten.get\_tasks}\sphinxstyleindexpageref{masterCodeDoc:\detokenize{module-checklisten.templatetags.t_checklisten.get_tasks}}
\item\relax\sphinxstyleindexentry{checklisten.views}\sphinxstyleindexpageref{masterCodeDoc:\detokenize{module-checklisten.views}}
\indexspace
\bigletter{h}
\item\relax\sphinxstyleindexentry{historie.templatetags.t\_historie.get\_associated\_data}\sphinxstyleindexpageref{masterCodeDoc:\detokenize{module-historie.templatetags.t_historie.get_associated_data}}
\item\relax\sphinxstyleindexentry{historie.templatetags.t\_historie.to\_class\_name}\sphinxstyleindexpageref{masterCodeDoc:\detokenize{module-historie.templatetags.t_historie.to_class_name}}
\item\relax\sphinxstyleindexentry{historie.views}\sphinxstyleindexpageref{masterCodeDoc:\detokenize{module-historie.views}}
\indexspace
\bigletter{l}
\item\relax\sphinxstyleindexentry{login.views}\sphinxstyleindexpageref{masterCodeDoc:\detokenize{module-login.views}}
\indexspace
\bigletter{m}
\item\relax\sphinxstyleindexentry{mitglieder.views}\sphinxstyleindexpageref{masterCodeDoc:\detokenize{module-mitglieder.views}}
\end{sphinxtheindex}

\renewcommand{\indexname}{Index}
\printindex
\end{document}